\documentclass{article}
\usepackage{stmaryrd}
\usepackage{graphicx}
\usepackage[utf8]{inputenc}

\usepackage[french]{babel}
\usepackage[T1]{fontenc}
\usepackage{hyperref}
\usepackage{verbatim}

\usepackage{color, soul}

\usepackage{pgfplots}
\pgfplotsset{compat=1.15}
\usepackage{mathrsfs}

\usepackage{amsmath}
\usepackage{amsfonts}
\usepackage{amssymb}
\usepackage{tkz-tab}
\author{Destinés à la 1\textsuperscript{ère}S2\\Au Lycée de Dindéferlo}
\title{\textbf{Barycentre}}
\date{\today}
\usepackage{tikz}
\usetikzlibrary{arrows, shapes.geometric, fit}

% Commande pour la couleur d'accentuation
\newcommand{\myul}[2][black]{\setulcolor{#1}\ul{#2}\setulcolor{black}}
\newcommand\tab[1][1cm]{\hspace*{#1}}

\begin{document}
\maketitle
\newpage

\section*{\underline{\textbf{\textcolor{red}{I.Barycentre de plusieurs points}}}}
	La notion de barycentre de $2$ ou $3$ points pondérés, étudiée en Seconde, peut être étendue à $4$, $5$, $\ldots$, $n$ points. 
	
	Dans l'exposé qui suit, nous présentons les résultats avec $4$ points, mais ils restent valables avec un nombre quelconque de points.
	\subsection*{\underline{\textbf{\textcolor{red}{1) Barycentre de 3 points }}}}
	Soient $(A\;,\ \alpha)\ ;\ (B\;,\ \beta)\ et \ (C\;,\ \gamma).$ 3 points pondérés avec $\alpha$, $\beta$ et $\gamma$ des réels tels que $\alpha+\beta+\gamma\neq 0$. Le barycentre du système: $(A\;,\ \alpha)\ ;\ (B\;,\ \beta)\ et \ (C\;,\ \gamma).$ est l'unique point $G$ du plan tel que $$\alpha\overrightarrow{GA}+\beta\overrightarrow{GB}+\gamma\overrightarrow{GC}=\overrightarrow{0}\quad$$

On note
\begin{table}[h!]
\textbf{G=bar}
\begin{tabular}{|c|c|c|}
\textbf{A} & \textbf{B} & \textbf{C} \\ \hline
$\alpha$        & $\beta$           & $\gamma$          \\
\end{tabular}
\end{table}

\textbf{NB} Pour la contruction on utilise la relation $$\overrightarrow{AG}=\dfrac{\beta}{\alpha+\beta+\gamma}\overrightarrow{AB}+\dfrac{\gamma}{\alpha+\beta+\gamma}\overrightarrow{AC}$$

\textbf{Exemple 1}

Contruire $G=\{(A\;,\ 2)\ ;\ (B\;,\ -1)\ ;\ (C\;,\ 2) \}.$

	\subsection*{\underline{\textbf{\textcolor{red}{2) Barycentre de 4 points }}}}
	
	Soient $\alpha$, $\beta$, $\gamma$ et $\delta$ des réels tel que $\alpha+\beta+\gamma+\delta\neq 0$ et $A$, $B$, $C$, $D$ des points du plan. 
	
	Il existe un unique point $G$ tel que :
	
	$$\alpha\overrightarrow{GA}+\beta\overrightarrow{GB}+\gamma\overrightarrow{GC}+\delta\overrightarrow{GD}=\overrightarrow{0}\quad(1)$$
	$$\overrightarrow{AG}=\dfrac{\beta}{\alpha+\beta+\gamma+\delta}\overrightarrow{AB}+\dfrac{\gamma}{\alpha+\beta+\gamma+\delta}\overrightarrow{AC}+\dfrac{\delta} {\alpha+\beta+\gamma+\delta}\overrightarrow{AD}\quad(2)$$
	Le point $G$ est appelé barycentre du système de points pondérés ${(A\;,\ \alpha)\ ;\ (B\;,\ \beta)\ ;\ (C\;,\ \gamma)\ ;\ (D\;,\ \delta)}.$
	
	Démonstration du passage de (1) à (2):
	
	$\begin{array}{rcl} \alpha\overrightarrow{GA}+\beta\overrightarrow{GB}+\gamma\overrightarrow{GC}+\delta\overrightarrow{GD}=\vec{0} & \Leftrightarrow & \alpha\overrightarrow{GA}+\beta\left(\overrightarrow{GA}+\overrightarrow{AB}\right)+\gamma\left(\overrightarrow{GA}+\overrightarrow{AC}\right)+\delta\left(\overrightarrow{GA}+\overrightarrow{AD}\right)=\vec{0}\\\\ & \Leftrightarrow & (\alpha+\beta+\gamma+\delta)\overrightarrow{GA}+\beta\overrightarrow{AB}+\gamma\overrightarrow{AC}+\delta\overrightarrow{AD}=\overrightarrow{0}\\\\ & \Leftrightarrow & (\alpha+\beta+\gamma+\delta)\overrightarrow{AG}=\beta\overrightarrow{AB}+\gamma\overrightarrow{AC}+\delta\overrightarrow{AD}\\\\ & \Leftrightarrow & \overrightarrow{AG}=\dfrac{\beta}{\alpha+\beta+\gamma+\delta}\overrightarrow{AB}+\dfrac{\gamma}{\alpha+\beta+\gamma+\delta}\overrightarrow{AC}+\dfrac{\delta} {\alpha+\beta+\gamma+\delta}\overrightarrow{AD}\quad(2)\end{array}$
	
	Soit $\overrightarrow{V}$ le vecteur au second membre de $(2).$
	
	$\overrightarrow{V}$ est un vecteur fixe puisque $A$, $B$, $C$, $D$ et $\alpha$, $\beta$, $\gamma$ et $\delta$ sont donnés.
	
	D'après l'Axiome d'EUCLIDE, cf. paragraphe 1, il existe un unique point $G$ tel que :
	
	$$\overrightarrow{AG}=\overrightarrow{V}$$
	
	\textbf{N.B}

	La relation $(2)$ de la démonstration précédente permet de construire vectoriellement le barycentre de plusieurs points.

	On a aussi les relations analogues :
	
		$\overrightarrow{BG}=\dfrac{\alpha}{\alpha+\beta+\gamma+\delta}\overrightarrow{BA}+\dfrac{\gamma}{\alpha+\beta+\gamma+\delta}\overrightarrow{BC}+\dfrac{\delta}{\alpha+\beta+\gamma+\delta}\overrightarrow{BD}$
		
		$\overrightarrow{CG}=\dfrac{\alpha}{\alpha+\beta+\gamma+\delta}\overrightarrow{CA}+\dfrac{\beta}{\alpha+\beta+\gamma+\delta}\overrightarrow{CB}+\dfrac{\delta}{\alpha+\beta+\gamma+\delta}\overrightarrow{CD}$
		
		\subsection*{\underline{\textbf{\textcolor{red}{2) Homogénéité du barycentre}}}}
		
		Le barycentre de plusieurs points reste inchangé lorsqu'on multiplie tous les coefficients par un même nombre non nul.
		
		En effet soit $G$ le barycentre du système ${(A\;,\ \alpha)\ ;\ (B\;,\ \beta)\ ;\ (C\;,\ \gamma)\ ;\ (D\;,\ \delta)}$ et $k$ un réel non nul. 

On a :

$\begin{array}{rcl} \alpha\overrightarrow{GA}+\beta\overrightarrow{GB}+\gamma\overrightarrow{GC}+\delta\overrightarrow{GD}=\vec{0}&\Rightarrow & k\left(\alpha\overrightarrow{GA}+\beta\overrightarrow{GB}+\gamma\overrightarrow{GC}\delta\overrightarrow{GD}\right)=\vec{0}\\\\&\Rightarrow & (k\alpha)\overrightarrow{GA}+(k\beta)\overrightarrow{GB}+(k\gamma)\overrightarrow{GC}+(k\delta)\overrightarrow{GD}=\vec{0}\end{array}$

Et la somme $(k\alpha+k\beta+k\gamma+k\delta)=k(\alpha+\beta+\gamma+\delta)$ est non nulle par hypothèse.

Donc $G$ est aussi le barycentre du système ${(A\;,\ k\alpha)\ ;\ (B\;,\ k\beta)\ ;\ (C\;,\ k\gamma)\ ;\ (D\;,\ k\delta)}.$

Cas particulier :

Si tous les coefficients sont égaux et non nuls (i.e. $\alpha=\beta=\gamma=\delta$ et $\alpha\neq 0)$, le barycentre du système ${(A\;,\ \alpha)\ ;\ (B\;,\ \alpha)\ ;\ (C\;,\ \alpha)\ ;\ (D\;,\ \alpha)}$ est aussi celui de ${(A\;,\ 1)\ ;\ (B\;,\ 1)\ ;\ (C\;,\ 1)\ ;\ (D\;,\ 1)}.$

\textbf{N.B.} : Selon la configuration des points du système, le barycentre se situe :
\begin{itemize}
    \item au milieu du segment \([AB]\), dans le cas d'un système de deux points distincts ;
    \item au centre de gravité du triangle \(\triangle ABC\), dans le cas d'un système de trois points non alignés ;
    \item au point de concours des diagonales, dans le cas d'un système de quatre points formant un parallélogramme.
\end{itemize}
\subsection*{\underline{\textbf{\textcolor{red}{3) Réduction du vecteur}}}}
$\overrightarrow{V_{M}}=\alpha\overrightarrow{MA}+\beta\overrightarrow{MB}+\gamma\overrightarrow{MC}+\delta\overrightarrow{MD}$, $M$ point quelconque du plan

\textbf{$1^{er}$ cas :} Si $(\alpha+\beta+\gamma+\delta)\neq 0$ :

Alors $G$ le barycentre du système ${(A\;,\ \alpha)\ ;\ (B\;,\ \beta)\ ;\ (C\;,\ \gamma)\ ;\ (D\;,\ \delta)}.$

$G$ existe d'après le théorème du 1)

On peut écrire d'après la relation de CHASLES :

$\begin{array}{rcl}\overrightarrow{V_{M}}&=&\alpha\left(\overrightarrow{MG}+\overrightarrow{GA}\right)+\beta\left(\overrightarrow{MG}+\overrightarrow{GB}\right)+\gamma\left(\overrightarrow{MG}+\overrightarrow{GC}\right)+\delta\left(\overrightarrow{MG}+\overrightarrow{GD}\right)\\ \\&=&(\alpha+\beta+\gamma+\delta)\overrightarrow{MG}+\underbrace{\alpha\overrightarrow{GA}+\beta\overrightarrow{GB}+\gamma\overrightarrow{GC}+\delta\overrightarrow{GD}}_{=\overrightarrow{0}\text{ par définition de }G}\end{array}$

Ainsi, dans ce cas, le vecteur $\overrightarrow{V_{M}}$ se réduit à :

$$\overrightarrow{V_{M}}=(\alpha+\beta+\gamma+\delta)\overrightarrow{MG}.$$

On retiendra que : si le système ${(A\;,\ \alpha)\ ;\ (B\;,\ \beta)\ ;\ (C\;,\ \gamma)\ ;\ (D\;,\ \delta)}$ a un barycentre, alors pour tout point $M$ du plan, on a :

$$\alpha\overrightarrow{MA}+\beta\overrightarrow{MB}+\gamma\overrightarrow{MC}+\delta\overrightarrow{MD}=(\alpha+\beta+\gamma+\delta)\overrightarrow{MG}$$

\textbf{$2^{er}$ cas :} Si $(\alpha+\beta+\gamma+\delta)=0$ :

Soit $N$ un autre point du plan.

On peut écrire :

$\begin{array}{rcl}\overrightarrow{V_{M}}&=&\alpha\left(\overrightarrow{MN}+\overrightarrow{NA}\right)+\beta\left(\overrightarrow{MN}+\overrightarrow{NB}\right)+\gamma\left(\overrightarrow{MN}+\overrightarrow{NC}\right)+\delta\left(\overrightarrow{MN}+\overrightarrow{ND}\right)\\ \\&=&\underbrace{(\alpha+\beta+\gamma+\delta)\overrightarrow{MN}}_{=\overrightarrow{0}\text{ car par hypothèse }(\alpha+\beta+\gamma+\delta)=0}+\underbrace{\alpha\overrightarrow{NA}+\beta\overrightarrow{NB}+\gamma\overrightarrow{NC}+\delta\overrightarrow{ND}}_{=\overrightarrow{V_{N}}}\end{array}$

Ainsi, dans ce cas, si $M$ et $N$ sont deux points quelconques du plan, on a : $\overrightarrow{V_{M}}=\overrightarrow{V_{N}}.$

Donc $\overrightarrow{V_{M}}$ est un vecteur constant.

Il est égal, par exemple, à $\beta\overrightarrow{AB}+\gamma\overrightarrow{AC}+\delta\overrightarrow{AD}$ ($M$ remplacé par $A).$
\subsection*{\underline{\textbf{\textcolor{red}{4) Associativité du barycentre}}}}

Soient \( A \), \( B \), \( C \), \( D \) quatre points du plan, et \( G \) le barycentre du système :  
\[
G = \{(A, -1), (B, 2), (C, -1), (D, 3)\}.
\]

\paragraph{Existence de \( G \) :}  
Le barycentre \( G \) existe car la somme des coefficients est non nulle :  
\[
-1 + 2 - 1 + 3 = 3 \neq 0.
\]

Nous considérons deux sous-systèmes pondérés :  
- \( G_1 = \{(A, -1), (B, 2)\} \), de barycentre \( I \).  
- \( G_2 = \{(C, -1), (D, 3)\} \), de barycentre \( J \).

D’après la propriété 3) (vue précédemment), pour tout point \( M \) du plan, on a :  
\[
-\overrightarrow{MA} + 2\overrightarrow{MB} = \overrightarrow{MI} \quad (1),
\]
\[
-\overrightarrow{MC} + 3\overrightarrow{MD} = 2\overrightarrow{MJ} \quad (2).
\]

En remplaçant \( M \) par \( G \) dans les relations (1) et (2), et en additionnant les deux équations, on obtient :  
\[
-\overrightarrow{GA} + 2\overrightarrow{GB} - \overrightarrow{GC} + 3\overrightarrow{GD} = \overrightarrow{GI} + 2\overrightarrow{GJ}.
\]

Par définition de \( G \), le vecteur de gauche est nul :  
\[
\overrightarrow{GI} + 2\overrightarrow{GJ} = \vec{0}.
\]

Cela montre que \( G \) est le barycentre du système \( S' = \{(I, 1), (J, 2)\} \).  
On observe que les coefficients \( 1 \) et \( 2 \) correspondent respectivement aux sommes des coefficients des systèmes \( G_1 \) et \( G_2 \).

\paragraph{Cas du barycentre \( K \) d’un sous-système étendu :}  
Considérons le système \( G_3 = \{(B, 2), (C, -1), (D, 3)\} \), dont le barycentre est \( K \).  
Pour tout point \( M \) du plan, la propriété 3) donne :  
\[
2\overrightarrow{MB} - \overrightarrow{MC} + 3\overrightarrow{MD} = 4\overrightarrow{MK}.
\]

En remplaçant \( M \) par \( G \), on obtient :  
\[
2\overrightarrow{GB} - \overrightarrow{GC} + 3\overrightarrow{GD} = 4\overrightarrow{GK}.
\]

Ajoutons \( -\overrightarrow{GA} \) à chaque membre :  
\[
-\overrightarrow{GA} + 2\overrightarrow{GB} - \overrightarrow{GC} + 3\overrightarrow{GD} = -\overrightarrow{GA} + 4\overrightarrow{GK}.
\]

Par définition de \( G \), le vecteur de gauche est nul, donc :  
\[
-\overrightarrow{GA} + 4\overrightarrow{GK} = \vec{0}.
\]

Ainsi, \( G \) est aussi le barycentre du système \( G'' = \{(A, -1), (K, 4)\} \).

\paragraph{Conclusion :}  
Ces calculs montrent que, dans la recherche du barycentre de plusieurs points, on peut regrouper certains points en sous-systèmes, remplacer ces sous-systèmes par leurs barycentres partiels, et affecter à ces barycentres la somme des coefficients des points regroupés (à condition que cette somme soit non nulle).

		
\subsection*{\underline{\textbf{\textcolor{red}{5) Contruction du barycentre de 4 points}}}}

Soient \( A \), \( B \), \( C \), \( D \) quatre points du plan et \( G \) le barycentre du système :  
\[
G = \{(A, m_A), (B, m_B), (C, m_C), (D, m_D)\},
\]
où \( m_A, m_B, m_C, m_D \) sont des coefficients réels tels que leur somme est non nulle :  
\[
m_A + m_B + m_C + m_D \neq 0.
\]

**Procédé utilisant l’associativité :**

1. **Étape 1 : Construction de barycentres partiels**
  
   - Regroupez les points \( A \) et \( B \) pour former un sous-système \( G_1 = \{(A, m_A), (B, m_B)\} \).  
     \[
     \overrightarrow{AG_{1}} = \frac{m_A}{{m_A + m_B}} \overrightarrow{AB}.
     \]
   - Regroupez ensuite \( C \) et \( D \) pour former un autre sous-système \( G_2 = \{(C, m_C), (D, m_D)\} \).  
     \[
     \overrightarrow{CG_{2}} = \frac{m_A}{{m_A + m_B}} \overrightarrow{CD}.
     \]

2. **Étape 2 : Barycentre global des barycentres partiels**  

   Le barycentre \( G \) du système initial s’obtient comme barycentre des deux points pondérés \( G_1 \) et \( G_2 \) :  
   \[
   G = \{(G_{1}, m_A + m_B), (G_{2}, m_C + m_D)\}.
   \]
   Ainsi :  
   \[
   \overrightarrow{G_{1}G} = \frac{(m_A + m_B) }{m_A + m_B+m_C + m_D}\overrightarrow{G_{1}G_{2}}
   \]

\section*{Exercice 4 \hfill (5 points)}

\begin{enumerate}
    \item $ABCD$ est un carré de centre $O$. Faire une figure avec $AB = 4 \, \text{cm}$.
    
    \item Soit le système de points pondérés suivants : $\{(A, 2); (B, 3); (C, 1); (D, -1)\}$.
    \begin{enumerate}
        \item Justifier qu’il existe un point $G$ barycentre de ce système. \hfill \textbf{0,25 pt}
        
        \item En déduire une relation vectorielle caractérisant le barycentre $G$. \hfill \textbf{0,5 pt}
        
        \item Placer les points $H$ et $K$ tels que :
        \[
        \overrightarrow{AH} = \frac{1}{3} \overrightarrow{AC} \quad \text{et} \quad \overrightarrow{BK} = -\frac{1}{2} \overrightarrow{BD}.
        \]
        \hfill \textbf{0,5 pt}
        
        \item Écrire le point $H$ comme barycentre de $A$ et $C$ puis $K$ comme barycentre de $B$ et $D$. \hfill \textbf{0,25 pt + 0,25 pt}
        
        \item Montrer que les points $G$, $H$ et $K$ sont alignés. \hfill \textbf{0,5 pt}
        
        \item Construire $G$. \hfill \textbf{0,75 pt}
    \end{enumerate}
\end{enumerate}

\subsection*{\underline{\textbf{\textcolor{red}{6) Décomposer un vecteur de l’espace}}}}

\paragraph{Décomposition dans une base orthonormée :}

Soit \( \vec{v} \) un vecteur de l’espace défini par deux points \( A(x_A, y_A, z_A) \) et \( B(x_B, y_B, z_B) \).  
Dans une base orthonormée \((\vec{i}, \vec{j}, \vec{k})\), la décomposition de \( \overrightarrow{AB} \) est donnée par :  
\[
\overrightarrow{AB} = (x_B - x_A) \vec{i} + (y_B - y_A) \vec{j} + (z_B - z_A) \vec{k}.
\]

**Exemple :**  
Si \( A(1, 2, 3) \) et \( B(4, 6, 5) \), alors :  
\[
\overrightarrow{AB} = (4 - 1)\vec{i} + (6 - 2)\vec{j} + (5 - 3)\vec{k} = 3\vec{i} + 4\vec{j} + 2\vec{k}.
\]

\paragraph{Norme d’un vecteur dans l’espace :}

La norme de \( \overrightarrow{AB} \) est donnée par :  
\[
\|\overrightarrow{AB}\| = \sqrt{(x_B - x_A)^2 + (y_B - y_A)^2 + (z_B - z_A)^2}.
\]

**Exemple :**  
Avec les mêmes points \( A(1, 2, 3) \) et \( B(4, 6, 5) \), on obtient :  
\[
\|\overrightarrow{AB}\| = \sqrt{(4-1)^2 + (6-2)^2 + (5-3)^2} = \sqrt{9 + 16 + 4} = \sqrt{29}.
\]

\paragraph{Projection sur des axes :}

Le vecteur \( \overrightarrow{AB} \) peut être projeté sur chacun des axes de l’espace :

- Projection sur l’axe \( Ox \) : \( (x_B - x_A)\vec{i} \),

- Projection sur l’axe \( Oy \) : \( (y_B - y_A)\vec{j} \),

- Projection sur l’axe \( Oz \) : \( (z_B - z_A)\vec{k} \).

Ainsi, la décomposition de \( \overrightarrow{AB} \) dans l’espace est la somme de ses projections sur \( Ox \), \( Oy \), et \( Oz \).

\paragraph{Décomposition dans un plan de l’espace :}

La projection de \( \overrightarrow{AB} \) sur un plan de l’espace se fait en annulant la composante perpendiculaire à ce plan :  
- **Projection sur \( Oxy \) :**  
  Annulez la composante selon \( z \) :  
  \[
  \overrightarrow{AB}_{xy} = (x_B - x_A)\vec{i} + (y_B - y_A)\vec{j}.
  \]
- **Projection sur \( Oxz \) :**  
  Annulez la composante selon \( y \) :  
  \[
  \overrightarrow{AB}_{xz} = (x_B - x_A)\vec{i} + (z_B - z_A)\vec{k}.
  \]
- **Projection sur \( Oyz \) :**  
  Annulez la composante selon \( x \) :  
  \[
  \overrightarrow{AB}_{yz} = (y_B - y_A)\vec{j} + (z_B - z_A)\vec{k}.
  \]

**Exemple :**  
Avec \( A(1, 2, 3) \) et \( B(4, 6, 5) \) :  

- \( \overrightarrow{AB}_{xy} = 3\vec{i} + 4\vec{j} \),  

- \( \overrightarrow{AB}_{xz} = 3\vec{i} + 2\vec{k} \),  

- \( \overrightarrow{AB}_{yz} = 4\vec{j} + 2\vec{k} \).

\subsection*{\underline{\textbf{\textcolor{red}{7) Propriété : Trois points et leur barycentre sont coplanaires}}}}

Soient \( A, B, C \) trois points non alignés de l’espace et \( G \) leur barycentre avec les coefficients \( m_A, m_B, m_C \).  
La propriété stipule que \( A, B, C \) et \( G \) sont toujours coplanaires.

\paragraph{Démonstration :}  

1. Par définition, le barycentre \( G \) est donné par :  
   \[
   \overrightarrow{OG} = \frac{m_A\overrightarrow{OA} + m_B\overrightarrow{OB} + m_C\overrightarrow{OC}}{m_A + m_B + m_C},
   \]
   où \( m_A + m_B + m_C \neq 0 \).

2. Le plan défini par \( A, B, C \) est déterminé par les vecteurs \( \overrightarrow{AB} \) et \( \overrightarrow{AC} \), c'est-à-dire que tout point \( P \) appartenant à ce plan peut être exprimé comme :  
   \[
   \overrightarrow{OP} = \overrightarrow{OA} + \lambda \overrightarrow{AB} + \mu \overrightarrow{AC}, \quad \lambda, \mu \in \mathbb{R}.
   \]

3. Puisque \( \overrightarrow{OG} \) est une combinaison linéaire des vecteurs \( \overrightarrow{OA}, \overrightarrow{OB}, \overrightarrow{OC} \), il peut s'écrire comme une combinaison des vecteurs \( \overrightarrow{OA}, \overrightarrow{AB}, \overrightarrow{AC} \), qui forment une base du plan \( (A, B, C) \).

4. Par conséquent, le vecteur \( \overrightarrow{OG} \) appartient au plan \( (A, B, C) \), ce qui montre que \( G \) est coplanaire avec \( A, B, C \).

\paragraph{Conséquence :}  
Pour quatre points \( A, B, C, D \) dans l’espace, si \( G \) est le barycentre de \( (A, B, C) \), alors \( G \) et \( D \) seront coplanaires si et seulement si \( D \) appartient au plan \( (A, B, C) \).

\paragraph{Applications :}  

1. **Construction géométrique :**  

   Si \( A, B, C \) sont donnés dans l’espace avec des coefficients \( m_A, m_B, m_C \), le barycentre \( G \) peut être construit comme suit :  
   
   - Trouver d’abord un point \( I \), barycentre de \( A \) et \( B \), tel que :  
     \[
     \overrightarrow{OI} = \frac{m_A \overrightarrow{OA} + m_B \overrightarrow{OB}}{m_A + m_B}.
     \]
   - Ensuite, calculer \( G \) comme barycentre de \( I \) et \( C \) :  
     \[
     \overrightarrow{OG} = \frac{(m_A + m_B)\overrightarrow{OI} + m_C \overrightarrow{OC}}{m_A + m_B + m_C}.
     \]

3. **Problèmes en physique :**  
   Cette propriété est souvent utilisée pour analyser les forces ou les centres de gravité dans des systèmes où des points massiques sont répartis dans un plan.


\subsection*{\underline{\textbf{\textcolor{red}{8) Centre d’inertie d’une plaque homogène}}}}
\section*{Propriété : Coplanarité des points et de leur barycentre}

\textbf{Énoncé de la propriété :} \\
Dans l’espace, trois points \( A \), \( B \) et \( C \), ainsi que leur barycentre \( G \), sont toujours coplanaires. \\

\textbf{Justification :}
\begin{itemize}
    \item Les trois points \( A \), \( B \), et \( C \) définissent un plan unique (s'ils ne sont pas alignés).
    \item Le barycentre \( G \) des points \( A \), \( B \), et \( C \) est défini comme une combinaison linéaire des vecteurs associés à ces points. Cela implique que \( G \) est un point de ce plan, quelle que soit la répartition des masses associées à \( A \), \( B \), et \( C \).
\end{itemize}

Ainsi, \( A \), \( B \), \( C \), et \( G \) appartiennent nécessairement au même plan.
		
\end{document}