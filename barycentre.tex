\documentclass{article}
\usepackage{stmaryrd}
\usepackage{graphicx}
\usepackage[utf8]{inputenc}

\usepackage[french]{babel}
\usepackage[T1]{fontenc}
\usepackage{hyperref}
\usepackage{verbatim}

\usepackage{color, soul}

\usepackage{pgfplots}
\pgfplotsset{compat=1.15}
\usepackage{mathrsfs}

\usepackage{amsmath}
\usepackage{amsfonts}
\usepackage{amssymb}
\usepackage{tkz-tab}
\author{Destinés à la 1\textsuperscript{ère}S2\\Au Lycée de Dindéferlo}
\title{\textbf{Barycentre}}
\date{\today}
\usepackage{tikz}
\usetikzlibrary{arrows, shapes.geometric, fit}

% Commande pour la couleur d'accentuation
\newcommand{\myul}[2][black]{\setulcolor{#1}\ul{#2}\setulcolor{black}}
\newcommand\tab[1][1cm]{\hspace*{#1}}

\begin{document}
\maketitle
\newpage

\section*{\underline{\textbf{\textcolor{red}{I.Barycentre de plusieurs points}}}}
	La notion de barycentre de $2$ ou $3$ points pondérés, étudiée en Seconde, peut être étendue à $4$, $5$, $\ldots$, $n$ points. 
	
	Dans l'exposé qui suit, nous présentons les résultats avec $4$ points, mais ils restent valables avec un nombre quelconque de points.
	\subsection*{\underline{\textbf{\textcolor{red}{1) Barycentre de 3 points }}}}
	Soient $(A\;,\ \alpha)\ ;\ (B\;,\ \beta)\ et \ (C\;,\ \gamma).$ 3 points pondérés avec $\alpha$, $\beta$ et $\gamma$ des réels tels que $\alpha+\beta+\gamma\neq 0$. Le barycentre du système: $(A\;,\ \alpha)\ ;\ (B\;,\ \beta)\ et \ (C\;,\ \gamma).$ est l'unique point $G$ du plan tel que $$\alpha\overrightarrow{GA}+\beta\overrightarrow{GB}+\gamma\overrightarrow{GC}=\overrightarrow{0}\quad$$

On note
\begin{table}[h!]
\textbf{G=bar}
\begin{tabular}{|c|c|c|}
\textbf{A} & \textbf{B} & \textbf{C} \\ \hline
$\alpha$        & $\beta$           & $\gamma$          \\
\end{tabular}
\end{table}

\textbf{NB} Pour la contruction on utilise la relation $$\overrightarrow{AG}=\dfrac{\beta}{\alpha+\beta+\gamma}\overrightarrow{AB}+\dfrac{\gamma}{\alpha+\beta+\gamma}\overrightarrow{AC}$$

\textbf{Exemple 1}

Contruire $G=\{(A\;,\ 2)\ ;\ (B\;,\ -1)\ ;\ (C\;,\ 2) \}.$

	\subsection*{\underline{\textbf{\textcolor{red}{2) Barycentre de 4 points }}}}
	
	Soient $\alpha$, $\beta$, $\gamma$ et $\delta$ des réels tel que $\alpha+\beta+\gamma+\delta\neq 0$ et $A$, $B$, $C$, $D$ des points du plan. 
	
	Il existe un unique point $G$ tel que :
	
	$$\alpha\overrightarrow{GA}+\beta\overrightarrow{GB}+\gamma\overrightarrow{GC}+\delta\overrightarrow{GD}=\overrightarrow{0}\quad(1)$$
	$$\overrightarrow{AG}=\dfrac{\beta}{\alpha+\beta+\gamma+\delta}\overrightarrow{AB}+\dfrac{\gamma}{\alpha+\beta+\gamma+\delta}\overrightarrow{AC}+\dfrac{\delta} {\alpha+\beta+\gamma+\delta}\overrightarrow{AD}\quad(2)$$
	Le point $G$ est appelé barycentre du système de points pondérés ${(A\;,\ \alpha)\ ;\ (B\;,\ \beta)\ ;\ (C\;,\ \gamma)\ ;\ (D\;,\ \delta)}.$
	
	Démonstration du passage de (1) à (2):
	
	$\begin{array}{rcl} \alpha\overrightarrow{GA}+\beta\overrightarrow{GB}+\gamma\overrightarrow{GC}+\delta\overrightarrow{GD}=\vec{0} & \Leftrightarrow & \alpha\overrightarrow{GA}+\beta\left(\overrightarrow{GA}+\overrightarrow{AB}\right)+\gamma\left(\overrightarrow{GA}+\overrightarrow{AC}\right)+\delta\left(\overrightarrow{GA}+\overrightarrow{AD}\right)=\vec{0}\\\\ & \Leftrightarrow & (\alpha+\beta+\gamma+\delta)\overrightarrow{GA}+\beta\overrightarrow{AB}+\gamma\overrightarrow{AC}+\delta\overrightarrow{AD}=\overrightarrow{0}\\\\ & \Leftrightarrow & (\alpha+\beta+\gamma+\delta)\overrightarrow{AG}=\beta\overrightarrow{AB}+\gamma\overrightarrow{AC}+\delta\overrightarrow{AD}\\\\ & \Leftrightarrow & \overrightarrow{AG}=\dfrac{\beta}{\alpha+\beta+\gamma+\delta}\overrightarrow{AB}+\dfrac{\gamma}{\alpha+\beta+\gamma+\delta}\overrightarrow{AC}+\dfrac{\delta} {\alpha+\beta+\gamma+\delta}\overrightarrow{AD}\quad(2)\end{array}$
	
	Soit $\overrightarrow{V}$ le vecteur au second membre de $(2).$
	
	$\overrightarrow{V}$ est un vecteur fixe puisque $A$, $B$, $C$, $D$ et $\alpha$, $\beta$, $\gamma$ et $\delta$ sont donnés.
	
	D'après l'Axiome d'EUCLIDE, cf. paragraphe 1, il existe un unique point $G$ tel que :
	
	$$\overrightarrow{AG}=\overrightarrow{V}$$
	
	\textbf{N.B}

	La relation $(2)$ de la démonstration précédente permet de construire vectoriellement le barycentre de plusieurs points.

	On a aussi les relations analogues :
	
		$\overrightarrow{BG}=\dfrac{\alpha}{\alpha+\beta+\gamma+\delta}\overrightarrow{BA}+\dfrac{\gamma}{\alpha+\beta+\gamma+\delta}\overrightarrow{BC}+\dfrac{\delta}{\alpha+\beta+\gamma+\delta}\overrightarrow{BD}$
		
		$\overrightarrow{CG}=\dfrac{\alpha}{\alpha+\beta+\gamma+\delta}\overrightarrow{CA}+\dfrac{\beta}{\alpha+\beta+\gamma+\delta}\overrightarrow{CB}+\dfrac{\delta}{\alpha+\beta+\gamma+\delta}\overrightarrow{CD}$
		
		\subsection*{\underline{\textbf{\textcolor{red}{2) Homogénéité du barycentre}}}}
		
		Le barycentre de plusieurs points reste inchangé lorsqu'on multiplie tous les coefficients par un même nombre non nul.
		
		En effet soit $G$ le barycentre du système ${(A\;,\ \alpha)\ ;\ (B\;,\ \beta)\ ;\ (C\;,\ \gamma)\ ;\ (D\;,\ \delta)}$ et $k$ un réel non nul. 

On a :

$\begin{array}{rcl} \alpha\overrightarrow{GA}+\beta\overrightarrow{GB}+\gamma\overrightarrow{GC}+\delta\overrightarrow{GD}=\vec{0}&\Rightarrow & k\left(\alpha\overrightarrow{GA}+\beta\overrightarrow{GB}+\gamma\overrightarrow{GC}\delta\overrightarrow{GD}\right)=\vec{0}\\\\&\Rightarrow & (k\alpha)\overrightarrow{GA}+(k\beta)\overrightarrow{GB}+(k\gamma)\overrightarrow{GC}+(k\delta)\overrightarrow{GD}=\vec{0}\end{array}$

Et la somme $(k\alpha+k\beta+k\gamma+k\delta)=k(\alpha+\beta+\gamma+\delta)$ est non nulle par hypothèse.

Donc $G$ est aussi le barycentre du système ${(A\;,\ k\alpha)\ ;\ (B\;,\ k\beta)\ ;\ (C\;,\ k\gamma)\ ;\ (D\;,\ k\delta)}.$

Cas particulier :

Si tous les coefficients sont égaux et non nuls (i.e. $\alpha=\beta=\gamma=\delta$ et $\alpha\neq 0)$, le barycentre du système ${(A\;,\ \alpha)\ ;\ (B\;,\ \alpha)\ ;\ (C\;,\ \alpha)\ ;\ (D\;,\ \alpha)}$ est aussi celui de ${(A\;,\ 1)\ ;\ (B\;,\ 1)\ ;\ (C\;,\ 1)\ ;\ (D\;,\ 1)}.$

\textbf{N.B.} : Selon la configuration des points du système, le barycentre se situe :
\begin{itemize}
    \item au milieu du segment \([AB]\), dans le cas d'un système de deux points distincts ;
    \item au centre de gravité du triangle \(\triangle ABC\), dans le cas d'un système de trois points non alignés ;
    \item au point de concours des diagonales, dans le cas d'un système de quatre points formant un parallélogramme.
\end{itemize}
\subsection*{\underline{\textbf{\textcolor{red}{3) Réduction du vecteur}}}}
$\overrightarrow{V_{M}}=\alpha\overrightarrow{MA}+\beta\overrightarrow{MB}+\gamma\overrightarrow{MC}+\delta\overrightarrow{MD}$, $M$ point quelconque du plan

\textbf{$1^{er}$ cas :} Si $(\alpha+\beta+\gamma+\delta)\neq 0$ :

Alors $G$ le barycentre du système ${(A\;,\ \alpha)\ ;\ (B\;,\ \beta)\ ;\ (C\;,\ \gamma)\ ;\ (D\;,\ \delta)}.$

$G$ existe d'après le théorème du 1)

On peut écrire d'après la relation de CHASLES :

$\begin{array}{rcl}\overrightarrow{V_{M}}&=&\alpha\left(\overrightarrow{MG}+\overrightarrow{GA}\right)+\beta\left(\overrightarrow{MG}+\overrightarrow{GB}\right)+\gamma\left(\overrightarrow{MG}+\overrightarrow{GC}\right)+\delta\left(\overrightarrow{MG}+\overrightarrow{GD}\right)\\ \\&=&(\alpha+\beta+\gamma+\delta)\overrightarrow{MG}+\underbrace{\alpha\overrightarrow{GA}+\beta\overrightarrow{GB}+\gamma\overrightarrow{GC}+\delta\overrightarrow{GD}}_{=\overrightarrow{0}\text{ par définition de }G}\end{array}$

Ainsi, dans ce cas, le vecteur $\overrightarrow{V_{M}}$ se réduit à :

$$\overrightarrow{V_{M}}=(\alpha+\beta+\gamma+\delta)\overrightarrow{MG}.$$

On retiendra que : si le système ${(A\;,\ \alpha)\ ;\ (B\;,\ \beta)\ ;\ (C\;,\ \gamma)\ ;\ (D\;,\ \delta)}$ a un barycentre, alors pour tout point $M$ du plan, on a :

$$\alpha\overrightarrow{MA}+\beta\overrightarrow{MB}+\gamma\overrightarrow{MC}+\delta\overrightarrow{MD}=(\alpha+\beta+\gamma+\delta)\overrightarrow{MG}$$

\textbf{$2^{er}$ cas :} Si $(\alpha+\beta+\gamma+\delta)=0$ :

Soit $N$ un autre point du plan.

On peut écrire :

$\begin{array}{rcl}\overrightarrow{V_{M}}&=&\alpha\left(\overrightarrow{MN}+\overrightarrow{NA}\right)+\beta\left(\overrightarrow{MN}+\overrightarrow{NB}\right)+\gamma\left(\overrightarrow{MN}+\overrightarrow{NC}\right)+\delta\left(\overrightarrow{MN}+\overrightarrow{ND}\right)\\ \\&=&\underbrace{(\alpha+\beta+\gamma+\delta)\overrightarrow{MN}}_{=\overrightarrow{0}\text{ car par hypothèse }(\alpha+\beta+\gamma+\delta)=0}+\underbrace{\alpha\overrightarrow{NA}+\beta\overrightarrow{NB}+\gamma\overrightarrow{NC}+\delta\overrightarrow{ND}}_{=\overrightarrow{V_{N}}}\end{array}$

Ainsi, dans ce cas, si $M$ et $N$ sont deux points quelconques du plan, on a : $\overrightarrow{V_{M}}=\overrightarrow{V_{N}}.$

Donc $\overrightarrow{V_{M}}$ est un vecteur constant.

Il est égal, par exemple, à $\beta\overrightarrow{AB}+\gamma\overrightarrow{AC}+\delta\overrightarrow{AD}$ ($M$ remplacé par $A).$
\subsection*{\underline{\textbf{\textcolor{red}{4) Associativité du barycentre}}}}

Soient \( A \), \( B \), \( C \), \( D \) quatre points du plan, et \( G \) le barycentre du système :  
\[
G = \{(A, -1), (B, 2), (C, -1), (D, 3)\}.
\]

\paragraph{Existence de \( G \) :}  
Le barycentre \( G \) existe car la somme des coefficients est non nulle :  
\[
-1 + 2 - 1 + 3 = 3 \neq 0.
\]

Nous considérons deux sous-systèmes pondérés :  
- \( G_1 = \{(A, -1), (B, 2)\} \), de barycentre \( I \).  
- \( G_2 = \{(C, -1), (D, 3)\} \), de barycentre \( J \).

D’après la propriété 3) (vue précédemment), pour tout point \( M \) du plan, on a :  
\[
-\overrightarrow{MA} + 2\overrightarrow{MB} = \overrightarrow{MI} \quad (1),
\]
\[
-\overrightarrow{MC} + 3\overrightarrow{MD} = 2\overrightarrow{MJ} \quad (2).
\]

En remplaçant \( M \) par \( G \) dans les relations (1) et (2), et en additionnant les deux équations, on obtient :  
\[
-\overrightarrow{GA} + 2\overrightarrow{GB} - \overrightarrow{GC} + 3\overrightarrow{GD} = \overrightarrow{GI} + 2\overrightarrow{GJ}.
\]

Par définition de \( G \), le vecteur de gauche est nul :  
\[
\overrightarrow{GI} + 2\overrightarrow{GJ} = \vec{0}.
\]

Cela montre que \( G \) est le barycentre du système \( S' = \{(I, 1), (J, 2)\} \).  
On observe que les coefficients \( 1 \) et \( 2 \) correspondent respectivement aux sommes des coefficients des systèmes \( G_1 \) et \( G_2 \).

\paragraph{Cas du barycentre \( K \) d’un sous-système étendu :}  
Considérons le système \( G_3 = \{(B, 2), (C, -1), (D, 3)\} \), dont le barycentre est \( K \).  
Pour tout point \( M \) du plan, la propriété 3) donne :  
\[
2\overrightarrow{MB} - \overrightarrow{MC} + 3\overrightarrow{MD} = 4\overrightarrow{MK}.
\]

En remplaçant \( M \) par \( G \), on obtient :  
\[
2\overrightarrow{GB} - \overrightarrow{GC} + 3\overrightarrow{GD} = 4\overrightarrow{GK}.
\]

Ajoutons \( -\overrightarrow{GA} \) à chaque membre :  
\[
-\overrightarrow{GA} + 2\overrightarrow{GB} - \overrightarrow{GC} + 3\overrightarrow{GD} = -\overrightarrow{GA} + 4\overrightarrow{GK}.
\]

Par définition de \( G \), le vecteur de gauche est nul, donc :  
\[
-\overrightarrow{GA} + 4\overrightarrow{GK} = \vec{0}.
\]

Ainsi, \( G \) est aussi le barycentre du système \( G'' = \{(A, -1), (K, 4)\} \).

\paragraph{Conclusion :}  
Ces calculs montrent que, dans la recherche du barycentre de plusieurs points, on peut regrouper certains points en sous-systèmes, remplacer ces sous-systèmes par leurs barycentres partiels, et affecter à ces barycentres la somme des coefficients des points regroupés (à condition que cette somme soit non nulle).

		
\subsection*{\underline{\textbf{\textcolor{red}{5) Contruction du barycentre de 4 points}}}}

Soient \( A \), \( B \), \( C \), \( D \) quatre points du plan et \( G \) le barycentre du système :  
\[
G = \{(A, m_A), (B, m_B), (C, m_C), (D, m_D)\},
\]
où \( m_A, m_B, m_C, m_D \) sont des coefficients réels tels que leur somme est non nulle :  
\[
m_A + m_B + m_C + m_D \neq 0.
\]

**Procédé utilisant l’associativité :**

1. **Étape 1 : Construction de barycentres partiels**
  
   - Regroupez les points \( A \) et \( B \) pour former un sous-système \( G_1 = \{(A, m_A), (B, m_B)\} \).  
     \[
     \overrightarrow{AG_{1}} = \frac{m_A}{{m_A + m_B}} \overrightarrow{AB}.
     \]
   - Regroupez ensuite \( C \) et \( D \) pour former un autre sous-système \( G_2 = \{(C, m_C), (D, m_D)\} \).  
     \[
     \overrightarrow{CG_{2}} = \frac{m_A}{{m_A + m_B}} \overrightarrow{CD}.
     \]

2. **Étape 2 : Barycentre global des barycentres partiels**  

   Le barycentre \( G \) du système initial s’obtient comme barycentre des deux points pondérés \( I \) et \( J \) :  
   \[
   G = \{(G_{1}, m_A + m_B), (G_{2}, m_C + m_D)\}.
   \]
   Ainsi :  
   \[
   \overrightarrow{G_{1}G} = \frac{(m_A + m_B) }{m_A + m_B+m_C + m_D}\overrightarrow{G_{1}G_{2}}
   \]

Exemple 2

G bar (A,2),(B,1),(C,-3),(D,1)

$G_{1}$ bar (A,2),(B,1)

$G_{2}$ bar (C,-3),(D,1).

donc G bar ($G_{1}$,3),($G_{2}$,-2)

\subsection*{\underline{\textbf{\textcolor{red}{6) Décomposer un vecteur de l’espace}}}}

\paragraph{Décomposition dans une base orthonormée :}

Soit \( \vec{v} \) un vecteur de l’espace défini par deux points \( A(x_A, y_A, z_A) \) et \( B(x_B, y_B, z_B) \).  
Dans une base orthonormée \((\vec{i}, \vec{j}, \vec{k})\), la décomposition de \( \overrightarrow{AB} \) est donnée par :  
\[
\overrightarrow{AB} = (x_B - x_A) \vec{i} + (y_B - y_A) \vec{j} + (z_B - z_A) \vec{k}.
\]

**Exemple :**  
Si \( A(1, 2, 3) \) et \( B(4, 6, 5) \), alors :  
\[
\overrightarrow{AB} = (4 - 1)\vec{i} + (6 - 2)\vec{j} + (5 - 3)\vec{k} = 3\vec{i} + 4\vec{j} + 2\vec{k}.
\]

\paragraph{Norme d’un vecteur dans l’espace :}

La norme de \( \overrightarrow{AB} \) est donnée par :  
\[
\|\overrightarrow{AB}\| = \sqrt{(x_B - x_A)^2 + (y_B - y_A)^2 + (z_B - z_A)^2}.
\]

**Exemple :**  
Avec les mêmes points \( A(1, 2, 3) \) et \( B(4, 6, 5) \), on obtient :  
\[
\|\overrightarrow{AB}\| = \sqrt{(4-1)^2 + (6-2)^2 + (5-3)^2} = \sqrt{9 + 16 + 4} = \sqrt{29}.
\]

\paragraph{Projection sur des axes :}

Le vecteur \( \overrightarrow{AB} \) peut être projeté sur chacun des axes de l’espace :

- Projection sur l’axe \( Ox \) : \( (x_B - x_A)\vec{i} \),

- Projection sur l’axe \( Oy \) : \( (y_B - y_A)\vec{j} \),

- Projection sur l’axe \( Oz \) : \( (z_B - z_A)\vec{k} \).

Ainsi, la décomposition de \( \overrightarrow{AB} \) dans l’espace est la somme de ses projections sur \( Ox \), \( Oy \), et \( Oz \).

\paragraph{Décomposition dans un plan de l’espace :}

La projection de \( \overrightarrow{AB} \) sur un plan de l’espace se fait en annulant la composante perpendiculaire à ce plan :  
- **Projection sur \( Oxy \) :**  
  Annulez la composante selon \( z \) :  
  \[
  \overrightarrow{AB}_{xy} = (x_B - x_A)\vec{i} + (y_B - y_A)\vec{j}.
  \]
- **Projection sur \( Oxz \) :**  
  Annulez la composante selon \( y \) :  
  \[
  \overrightarrow{AB}_{xz} = (x_B - x_A)\vec{i} + (z_B - z_A)\vec{k}.
  \]
- **Projection sur \( Oyz \) :**  
  Annulez la composante selon \( x \) :  
  \[
  \overrightarrow{AB}_{yz} = (y_B - y_A)\vec{j} + (z_B - z_A)\vec{k}.
  \]

**Exemple :**  
Avec \( A(1, 2, 3) \) et \( B(4, 6, 5) \) :  
- \( \overrightarrow{AB}_{xy} = 3\vec{i} + 4\vec{j} \),  
- \( \overrightarrow{AB}_{xz} = 3\vec{i} + 2\vec{k} \),  
- \( \overrightarrow{AB}_{yz} = 4\vec{j} + 2\vec{k} \).

\paragraph{Applications :}

1. **Produit scalaire dans l’espace :**  
   Le produit scalaire entre deux vecteurs \( \vec{u} = u_x\vec{i} + u_y\vec{j} + u_z\vec{k} \) et \( \vec{v} = v_x\vec{i} + v_y\vec{j} + v_z\vec{k} \) est :  
   \[
   \vec{u} \cdot \vec{v} = u_x v_x + u_y v_y + u_z v_z.
   \]

2. **Applications physiques :**  
   La décomposition des vecteurs est essentielle pour résoudre des problèmes de mécanique, comme la détermination des forces agissant sur un point matériel ou la projection d’un champ vectoriel dans un référentiel donné.

\paragraph{Conclusion :}  
La décomposition d’un vecteur de l’espace permet de comprendre sa direction et sa norme, ainsi que ses projections sur les axes et les plans. Cela constitue une base fondamentale pour les calculs géométriques et physiques.


\subsection*{\underline{\textbf{\textcolor{red}{7) Propriété : Trois points et leur barycentre sont coplanaires}}}}

Soient \( A, B, C \) trois points non alignés de l’espace et \( G \) leur barycentre avec les coefficients \( m_A, m_B, m_C \).  
La propriété stipule que \( A, B, C \) et \( G \) sont toujours coplanaires.

\paragraph{Démonstration :}  

1. Par définition, le barycentre \( G \) est donné par :  
   \[
   \overrightarrow{OG} = \frac{m_A\overrightarrow{OA} + m_B\overrightarrow{OB} + m_C\overrightarrow{OC}}{m_A + m_B + m_C},
   \]
   où \( m_A + m_B + m_C \neq 0 \).

2. Le plan défini par \( A, B, C \) est déterminé par les vecteurs \( \overrightarrow{AB} \) et \( \overrightarrow{AC} \), c'est-à-dire que tout point \( P \) appartenant à ce plan peut être exprimé comme :  
   \[
   \overrightarrow{OP} = \overrightarrow{OA} + \lambda \overrightarrow{AB} + \mu \overrightarrow{AC}, \quad \lambda, \mu \in \mathbb{R}.
   \]

3. Puisque \( \overrightarrow{OG} \) est une combinaison linéaire des vecteurs \( \overrightarrow{OA}, \overrightarrow{OB}, \overrightarrow{OC} \), il peut s'écrire comme une combinaison des vecteurs \( \overrightarrow{OA}, \overrightarrow{AB}, \overrightarrow{AC} \), qui forment une base du plan \( (A, B, C) \).

4. Par conséquent, le vecteur \( \overrightarrow{OG} \) appartient au plan \( (A, B, C) \), ce qui montre que \( G \) est coplanaire avec \( A, B, C \).

\paragraph{Conséquence :}  
Pour quatre points \( A, B, C, D \) dans l’espace, si \( G \) est le barycentre de \( (A, B, C) \), alors \( G \) et \( D \) seront coplanaires si et seulement si \( D \) appartient au plan \( (A, B, C) \).

\paragraph{Applications :}  

1. **Construction géométrique :**  
   Si \( A, B, C \) sont donnés dans l’espace avec des coefficients \( m_A, m_B, m_C \), le barycentre \( G \) peut être construit comme suit :  
   
   - Trouver d’abord un point \( I \), barycentre de \( A \) et \( B \), tel que :  
     \[
     \overrightarrow{OI} = \frac{m_A \overrightarrow{OA} + m_B \overrightarrow{OB}}{m_A + m_B}.
     \]
   - Ensuite, calculer \( G \) comme barycentre de \( I \) et \( C \) :  
     \[
     \overrightarrow{OG} = \frac{(m_A + m_B)\overrightarrow{OI} + m_C \overrightarrow{OC}}{m_A + m_B + m_C}.
     \]

2. **Étude de coplanarité d’un point avec un plan donné :**  
   Si \( D \) est un point extérieur, la condition de coplanarité de \( D \) avec \( A, B, C \) peut être vérifiée en calculant le volume mixte des vecteurs \( \overrightarrow{AB}, \overrightarrow{AC}, \overrightarrow{AD} \).  
   Le point \( D \) est coplanaire avec \( A, B, C \) si :  
   \[
   \det(\overrightarrow{AB}, \overrightarrow{AC}, \overrightarrow{AD}) = 0.
   \]

3. **Problèmes en physique :**  
   Cette propriété est souvent utilisée pour analyser les forces ou les centres de gravité dans des systèmes où des points massiques sont répartis dans un plan.

\paragraph{Conclusion :}  
La propriété de coplanarité du barycentre avec les points du système est une conséquence directe de sa définition comme combinaison linéaire des vecteurs position des points pondérés. Elle est un outil fondamental pour simplifier les problèmes géométriques et physiques dans l’espace.


\subsection*{\underline{\textbf{\textcolor{red}{8) Centre d’inertie d’une plaque homogène}}}}

\paragraph{Cas 1 : Plaque rectangulaire}  
Pour une plaque homogène rectangulaire de dimensions \( a \times b \), le centre d’inertie se situe au centre géométrique, soit :  
\[
(x_G, y_G) = \left(\frac{a}{2}, \frac{b}{2}\right).
\]

\paragraph{Cas 2 : Disque évidé d’un autre disque}  
Considérons un disque de rayon \( R \), homogène, évidé d’un disque concentrique de rayon \( r \).  
1. La densité surfacique est uniforme.  
2. Le centre d’inertie reste au centre géométrique des disques, soit \( (x_G, y_G) = (0, 0) \), car la répartition de masse est symétrique.

\paragraph{Cas 3 : Disque évidé d’une figure simple (excentrée)}  
Supposons un disque \( D \) de rayon \( R \) évidé d’un disque \( D' \) de rayon \( r \), avec les centres des deux disques situés respectivement en \( O \) et \( O' \).  
1. On calcule les masses des deux parties :  
   \[
   m_D = \rho \pi R^2, \quad m_{D'} = \rho \pi r^2.
   \]
2. Le barycentre du système se calcule comme :  
   \[
   \overrightarrow{OG} = \frac{m_D \overrightarrow{O} - m_{D'} \overrightarrow{O'}}{m_D - m_{D'}}.
   \]
   La contribution de \( D' \) est soustraite, car la masse est enlevée.

\paragraph{Résumé :}  
- Le centre d’inertie d’une plaque homogène simple est son centre géométrique. 
 
- Pour des figures évidées, le centre d’inertie dépend de la position et de la masse relative des parties évidées.

\subsection*{\underline{\textbf{\textcolor{red}{Transformation d'écriture}}}}
\section*{\underline{\textbf{\textcolor{red}{II. Géométrie analytique (droites et repères)}}}}
		Une base du plan est un couple $(\vec{u}\;,\ \vec{v})$ de vecteurs non colinéaires.
		
		Le triplet $(O\;,\ \vec{u}\;,\ \vec{v})$ où $O$ est un point du plan et $(\vec{u}\;,\ \vec{v})$ une base est un repère cartésien du plan.
		
		Soit $(\vec{i}\;,\ \vec{j})$ une base du plan.
		
		Soit $\vec{u}$ un vecteur quelconque.
		
		Il existe un couple unique $(x\;,\ y)$ de réels tels que :
		
		$\vec{u}=x\vec{i}+y\vec{j}.$
		
		Ces réels $x$ et $y$ sont appelés coordonnées de $\vec{u}$ dans la base $(\vec{i}\;,\ \vec{j}).$
		
		Soit $(O\;,\ \vec{i}\;,\ \vec{j})$ un repère du plan.
		
		Soit $M$ un point quelconque.
		
		Il existe un couple unique $(x\;,\ y)$ de réels tels que :
		
		$\overrightarrow{OM}=x\vec{i}+y\vec{j}.$
		
		Ces réels $x$ et $y$ sont appelés coordonnées de $M$ dans le repère $(O\;,\ \vec{i}\;,\ \vec{j}).$
		
		
		Condition de colinéarité :
		
		Soit $\vec{u}(x\;,\ y)$ et $\vec{v}(x'\;,\ y')$ dans une base $(\vec{i}\;,\ \vec{j}).$
		
		Alors $\vec{u}$ et $\vec{v}$ sont colinéaires si et seulement si $xy'-yx'=0$ $\ (\Leftrightarrow$ dét$(\vec{u}\;,\ \vec{v})=0).$
		
		Soit dans le plan $\mathcal{P}$ muni du repère $(O\;,\ \vec{i}\;,\ \vec{j})$ les points $A\left(x_{A}\;,\ y_{A}\right)$ et $B\left(x_{B}\;,\ y_{B}\right).$
		
		Alors $\overrightarrow{AB}$ a pour coordonnées $\overrightarrow{AB}\left(x_{B}-x_{A}\;,\ y_{B}-y_{A}\right)$ et le milieu $I$ du segment $[AB]$ a pour coordonnées $I\left(\dfrac{x_{A}+x_{B}}{2}\;;\ \dfrac{y_{A}+y_{B}}{2}\right).$
		
		Représentations analytique d'une droite
		
		Soit $(O\;,\ \vec{i}\;,\ \vec{j})$ un repère du plan.
		
		Soit $\mathcal{D}$ la droite passant par $A(x_{0}\;;\ y_{0})$ et de vecteur directeur $\overrightarrow{u}(\alpha\;,\ \beta).$
		
		Un point $M(x\;,\ y)$ appartient à $\mathcal{D}$ si et seulement si $\overrightarrow{AM}$ et $\overrightarrow{u}$ sont colinéaires , c'est-à-dire s'il existe un réel $t$ tel que : 
		
		$\overrightarrow{AM}=t\overrightarrow{AB}$, ce qui se traduit par le système suivant :
		
		$$\left\lbrace\begin{array}{lcl} x& =& x_{A}+t\alpha\\ \\ y& =& y_{A}+t\beta \end{array}\right.$$
		
		appelé système d'équations paramétriques de la droite $\mathcal{D}.$
		
		Soit $\mathcal{D}$ la droite passant par deux points $A$ et $B.$ 
		
		Un point $M(x\;,\ y)$ appartient à $\mathcal{D}$ si et seulement si $\overrightarrow{AM}$ et $\overrightarrow{AB}$ sont colinéaires, ce qui équivaut à dét $ \left(\overrightarrow{AM}\;,\ \overrightarrow{AB}\right)=0$ et se traduit par une relation de la forme : 
		
		$ax+by+c=0$ appelé équation cartésienne de la droite $\mathcal{D}.$
		
		Si $b\neq 0$, cette équation peut se mettre sous la forme $y=mx+p$ et s'appelle alors équation réduite de la droite $\mathcal{D}$ : 
		
		$m$ s'appelle le coefficient directeur de $\mathcal{D}$ (si le repère $(O\;,\ \vec{i}\;,\ \vec{j})$ est orthogonal, $m$ est aussi appelé pente de la droite $\mathcal{D}).$ 
		
		$p$ est l'ordonnée à l'origine.
		
		Coordonnées d'un vecteur directeur de $\mathcal{D}$ :
		
		$\bullet\ $ équation cartésienne $ax+by+c=0$ : $\overrightarrow{u}(-b\;;\ a).$
		
		$\bullet\ $ système d'équations paramétriques $$\left\lbrace\begin{array}{lcl} x& = & x_{A}+t\alpha\\ \\ y& =& y_{A}+t\beta \end{array}\qquad \vec{u}(\alpha\;,\ \beta)\right.$$
		
		$\bullet\ $ équation réduite $y=mx+p$ : $\ \vec{u}(1\;,\ m).$
		
		Conditions de parallélisme
		
		$\left.\begin{array}{lcl} \mathcal{D}\ :\ ax+by+c& = &0\\ \mathcal{D'}\ :\ a'x+b'y+c'& = &0 \end{array}\right\rbrace\qquad\mathcal{D}-\mathcal{D'}\Leftrightarrow ab'-ba'=0$
		
		$\left.\begin{array}{lcl} \mathcal{D}\ :\ y& =& mx+p\\ \mathcal{D'}\ :\ y& =& m'x+p' \end{array}\right\rbrace\qquad\mathcal{D}\times\mathcal{D'}\Leftrightarrow m=m'$
		
\end{document}