\documentclass{article}
\usepackage{stmaryrd}
\usepackage{graphicx}
\usepackage[utf8]{inputenc}

\usepackage[french]{babel}
\usepackage[T1]{fontenc}
\usepackage{hyperref}
\usepackage{verbatim}

\usepackage{color, soul}

\usepackage{pgfplots}
\pgfplotsset{compat=1.15}
\usepackage{mathrsfs}

\usepackage{amsmath}
\usepackage{amsfonts}
\usepackage{amssymb}
\usepackage{tkz-tab}
\author{Destinés à la 1\textsuperscript{ère}S2\\Au Lycée de Dindéferlo}
\title{\textbf{Barycentre}}
\date{\today}
\usepackage{tikz}
\usetikzlibrary{arrows, shapes.geometric, fit}

% Commande pour la couleur d'accentuation
\newcommand{\myul}[2][black]{\setulcolor{#1}\ul{#2}\setulcolor{black}}
\newcommand\tab[1][1cm]{\hspace*{#1}}

\begin{document}
\maketitle
\newpage

\section*{\underline{\textbf{\textcolor{red}{I.Barycentre de plusieurs points}}}}
	La notion de barycentre de $2$ ou $3$ points pondérés, étudiée en Seconde, peut être étendue à $4$, $5$, $\ldots$, $n$ points. 
	
	Dans l'exposé qui suit, nous présentons les résultats avec $4$ points, mais ils restent valables avec un nombre quelconque de points.
	\subsection*{\underline{\textbf{\textcolor{red}{1) Barycentre de 3 points }}}}
	Soient $(A\;,\ \alpha)\ ;\ (B\;,\ \beta)\ et \ (C\;,\ \gamma).$ 3 points pondérés avec $\alpha$, $\beta$ et $\gamma$ des réels tels que $\alpha+\beta+\gamma\neq 0$. Le barycentre du système: $(A\;,\ \alpha)\ ;\ (B\;,\ \beta)\ et \ (C\;,\ \gamma).$ est l'unique point $G$ du plan tel que $$\alpha\overrightarrow{GA}+\beta\overrightarrow{GB}+\gamma\overrightarrow{GC}=\overrightarrow{0}\quad$$

On note
\begin{table}[h!]
\textbf{G=bar}
\begin{tabular}{|c|c|c|}
\textbf{A} & \textbf{B} & \textbf{C} \\ \hline
$\alpha$        & $\beta$           & $\gamma$          \\
\end{tabular}
\end{table}

\textbf{NB} Pour la contruction on utilise la relation $$\overrightarrow{AG}=\dfrac{\beta}{\alpha+\beta+\gamma}\overrightarrow{AB}+\dfrac{\gamma}{\alpha+\beta+\gamma}\overrightarrow{AC}$$

\textbf{Exemple 1}

Contruire $G=\{(A\;,\ 2)\ ;\ (B\;,\ -1)\ ;\ (C\;,\ 2) \}.$

	\subsection*{\underline{\textbf{\textcolor{red}{2) Barycentre de 4 points }}}}
	
	Soient $\alpha$, $\beta$, $\gamma$ et $\delta$ des réels tel que $\alpha+\beta+\gamma+\delta\neq 0$ et $A$, $B$, $C$, $D$ des points du plan. 
	
	Il existe un unique point $G$ tel que :
	
	$$\alpha\overrightarrow{GA}+\beta\overrightarrow{GB}+\gamma\overrightarrow{GC}+\delta\overrightarrow{GD}=\overrightarrow{0}\quad(1)$$
	$$\overrightarrow{AG}=\dfrac{\beta}{\alpha+\beta+\gamma+\delta}\overrightarrow{AB}+\dfrac{\gamma}{\alpha+\beta+\gamma+\delta}\overrightarrow{AC}+\dfrac{\delta} {\alpha+\beta+\gamma+\delta}\overrightarrow{AD}\quad(2)$$
	Le point $G$ est appelé barycentre du système de points pondérés ${(A\;,\ \alpha)\ ;\ (B\;,\ \beta)\ ;\ (C\;,\ \gamma)\ ;\ (D\;,\ \delta)}.$
	
	Démonstration du passage de (1) à (2):
	
	$\begin{array}{rcl} \alpha\overrightarrow{GA}+\beta\overrightarrow{GB}+\gamma\overrightarrow{GC}+\delta\overrightarrow{GD}=\vec{0} & \Leftrightarrow & \alpha\overrightarrow{GA}+\beta\left(\overrightarrow{GA}+\overrightarrow{AB}\right)+\gamma\left(\overrightarrow{GA}+\overrightarrow{AC}\right)+\delta\left(\overrightarrow{GA}+\overrightarrow{AD}\right)=\vec{0}\\\\ & \Leftrightarrow & (\alpha+\beta+\gamma+\delta)\overrightarrow{GA}+\beta\overrightarrow{AB}+\gamma\overrightarrow{AC}+\delta\overrightarrow{AD}=\overrightarrow{0}\\\\ & \Leftrightarrow & (\alpha+\beta+\gamma+\delta)\overrightarrow{AG}=\beta\overrightarrow{AB}+\gamma\overrightarrow{AC}+\delta\overrightarrow{AD}\\\\ & \Leftrightarrow & \overrightarrow{AG}=\dfrac{\beta}{\alpha+\beta+\gamma+\delta}\overrightarrow{AB}+\dfrac{\gamma}{\alpha+\beta+\gamma+\delta}\overrightarrow{AC}+\dfrac{\delta} {\alpha+\beta+\gamma+\delta}\overrightarrow{AD}\quad(2)\end{array}$
	
	Soit $\overrightarrow{V}$ le vecteur au second membre de $(2).$
	
	$\overrightarrow{V}$ est un vecteur fixe puisque $A$, $B$, $C$, $D$ et $\alpha$, $\beta$, $\gamma$ et $\delta$ sont donnés.
	
	D'après l'Axiome d'EUCLIDE, cf. paragraphe 1, il existe un unique point $G$ tel que :
	
	$$\overrightarrow{AG}=\overrightarrow{V}$$
	
	\textbf{N.B}

	La relation $(2)$ de la démonstration précédente permet de construire vectoriellement le barycentre de plusieurs points.

	On a aussi les relations analogues :
	
		$\overrightarrow{BG}=\dfrac{\alpha}{\alpha+\beta+\gamma+\delta}\overrightarrow{BA}+\dfrac{\gamma}{\alpha+\beta+\gamma+\delta}\overrightarrow{BC}+\dfrac{\delta}{\alpha+\beta+\gamma+\delta}\overrightarrow{BD}$
		
		$\overrightarrow{CG}=\dfrac{\alpha}{\alpha+\beta+\gamma+\delta}\overrightarrow{CA}+\dfrac{\beta}{\alpha+\beta+\gamma+\delta}\overrightarrow{CB}+\dfrac{\delta}{\alpha+\beta+\gamma+\delta}\overrightarrow{CD}$
		
		\subsection*{\underline{\textbf{\textcolor{red}{2) Homogénéité du barycentre}}}}
		
		Le barycentre de plusieurs points reste inchangé lorsqu'on multiplie tous les coefficients par un même nombre non nul.
		
		En effet soit $G$ le barycentre du système ${(A\;,\ \alpha)\ ;\ (B\;,\ \beta)\ ;\ (C\;,\ \gamma)\ ;\ (D\;,\ \delta)}$ et $k$ un réel non nul. 

On a :

$\begin{array}{rcl} \alpha\overrightarrow{GA}+\beta\overrightarrow{GB}+\gamma\overrightarrow{GC}+\delta\overrightarrow{GD}=\vec{0}&\Rightarrow & k\left(\alpha\overrightarrow{GA}+\beta\overrightarrow{GB}+\gamma\overrightarrow{GC}\delta\overrightarrow{GD}\right)=\vec{0}\\\\&\Rightarrow & (k\alpha)\overrightarrow{GA}+(k\beta)\overrightarrow{GB}+(k\gamma)\overrightarrow{GC}+(k\delta)\overrightarrow{GD}=\vec{0}\end{array}$

Et la somme $(k\alpha+k\beta+k\gamma+k\delta)=k(\alpha+\beta+\gamma+\delta)$ est non nulle par hypothèse.

Donc $G$ est aussi le barycentre du système ${(A\;,\ k\alpha)\ ;\ (B\;,\ k\beta)\ ;\ (C\;,\ k\gamma)\ ;\ (D\;,\ k\delta)}.$

Cas particulier :

Si tous les coefficients sont égaux et non nuls (i.e. $\alpha=\beta=\gamma=\delta$ et $\alpha\neq 0)$, le barycentre du système ${(A\;,\ \alpha)\ ;\ (B\;,\ \alpha)\ ;\ (C\;,\ \alpha)\ ;\ (D\;,\ \alpha)}$ est aussi celui de ${(A\;,\ 1)\ ;\ (B\;,\ 1)\ ;\ (C\;,\ 1)\ ;\ (D\;,\ 1)}.$

\textbf{N.B.} : Selon la configuration des points du système, le barycentre se situe :
\begin{itemize}
    \item au milieu du segment \([AB]\), dans le cas d'un système de deux points distincts ;
    \item au centre de gravité du triangle \(\triangle ABC\), dans le cas d'un système de trois points non alignés ;
    \item au point de concours des diagonales, dans le cas d'un système de quatre points formant un parallélogramme.
\end{itemize}
\subsection*{\underline{\textbf{\textcolor{red}{3) Réduction du vecteur}}}}
$\overrightarrow{V_{M}}=\alpha\overrightarrow{MA}+\beta\overrightarrow{MB}+\gamma\overrightarrow{MC}+\delta\overrightarrow{MD}$, $M$ point quelconque du plan

\textbf{$1^{er}$ cas :} Si $(\alpha+\beta+\gamma+\delta)\neq 0$ :

Alors $G$ le barycentre du système ${(A\;,\ \alpha)\ ;\ (B\;,\ \beta)\ ;\ (C\;,\ \gamma)\ ;\ (D\;,\ \delta)}.$

$G$ existe d'après le théorème du 1)

On peut écrire d'après la relation de CHASLES :

$\begin{array}{rcl}\overrightarrow{V_{M}}&=&\alpha\left(\overrightarrow{MG}+\overrightarrow{GA}\right)+\beta\left(\overrightarrow{MG}+\overrightarrow{GB}\right)+\gamma\left(\overrightarrow{MG}+\overrightarrow{GC}\right)+\delta\left(\overrightarrow{MG}+\overrightarrow{GD}\right)\\ \\&=&(\alpha+\beta+\gamma+\delta)\overrightarrow{MG}+\underbrace{\alpha\overrightarrow{GA}+\beta\overrightarrow{GB}+\gamma\overrightarrow{GC}+\delta\overrightarrow{GD}}_{=\overrightarrow{0}\text{ par définition de }G}\end{array}$

Ainsi, dans ce cas, le vecteur $\overrightarrow{V_{M}}$ se réduit à :

$$\overrightarrow{V_{M}}=(\alpha+\beta+\gamma+\delta)\overrightarrow{MG}.$$

On retiendra que : si le système ${(A\;,\ \alpha)\ ;\ (B\;,\ \beta)\ ;\ (C\;,\ \gamma)\ ;\ (D\;,\ \delta)}$ a un barycentre, alors pour tout point $M$ du plan, on a :

$$\alpha\overrightarrow{MA}+\beta\overrightarrow{MB}+\gamma\overrightarrow{MC}+\delta\overrightarrow{MD}=(\alpha+\beta+\gamma+\delta)\overrightarrow{MG}$$

\textbf{$2^{er}$ cas :} Si $(\alpha+\beta+\gamma+\delta)=0$ :

Soit $N$ un autre point du plan.

On peut écrire :

$\begin{array}{rcl}\overrightarrow{V_{M}}&=&\alpha\left(\overrightarrow{MN}+\overrightarrow{NA}\right)+\beta\left(\overrightarrow{MN}+\overrightarrow{NB}\right)+\gamma\left(\overrightarrow{MN}+\overrightarrow{NC}\right)+\delta\left(\overrightarrow{MN}+\overrightarrow{ND}\right)\\ \\&=&\underbrace{(\alpha+\beta+\gamma+\delta)\overrightarrow{MN}}_{=\overrightarrow{0}\text{ car par hypothèse }(\alpha+\beta+\gamma+\delta)=0}+\underbrace{\alpha\overrightarrow{NA}+\beta\overrightarrow{NB}+\gamma\overrightarrow{NC}+\delta\overrightarrow{ND}}_{=\overrightarrow{V_{N}}}\end{array}$

Ainsi, dans ce cas, si $M$ et $N$ sont deux points quelconques du plan, on a : $\overrightarrow{V_{M}}=\overrightarrow{V_{N}}.$

Donc $\overrightarrow{V_{M}}$ est un vecteur constant.

Il est égal, par exemple, à $\beta\overrightarrow{AB}+\gamma\overrightarrow{AC}+\delta\overrightarrow{AD}$ ($M$ remplacé par $A).$
\subsection*{\underline{\textbf{\textcolor{red}{4) Associativité du barycentre}}}}

Soient \( A \), \( B \), \( C \), \( D \) quatre points du plan, et \( G \) le barycentre du système :  
\[
G = \{(A, -1), (B, 2), (C, -1), (D, 3)\}.
\]

\paragraph{Existence de \( G \) :}  
Le barycentre \( G \) existe car la somme des coefficients est non nulle :  
\[
-1 + 2 - 1 + 3 = 3 \neq 0.
\]

Nous considérons deux sous-systèmes pondérés :  
- \( G_1 = \{(A, -1), (B, 2)\} \), de barycentre \( I \).  
- \( G_2 = \{(C, -1), (D, 3)\} \), de barycentre \( J \).

D’après la propriété 3) (vue précédemment), pour tout point \( M \) du plan, on a :  
\[
-\overrightarrow{MA} + 2\overrightarrow{MB} = \overrightarrow{MI} \quad (1),
\]
\[
-\overrightarrow{MC} + 3\overrightarrow{MD} = 2\overrightarrow{MJ} \quad (2).
\]

En remplaçant \( M \) par \( G \) dans les relations (1) et (2), et en additionnant les deux équations, on obtient :  
\[
-\overrightarrow{GA} + 2\overrightarrow{GB} - \overrightarrow{GC} + 3\overrightarrow{GD} = \overrightarrow{GI} + 2\overrightarrow{GJ}.
\]

Par définition de \( G \), le vecteur de gauche est nul :  
\[
\overrightarrow{GI} + 2\overrightarrow{GJ} = \vec{0}.
\]

Cela montre que \( G \) est le barycentre du système \( S' = \{(I, 1), (J, 2)\} \).  
On observe que les coefficients \( 1 \) et \( 2 \) correspondent respectivement aux sommes des coefficients des systèmes \( G_1 \) et \( G_2 \).

\paragraph{Cas du barycentre \( K \) d’un sous-système étendu :}  
Considérons le système \( G_3 = \{(B, 2), (C, -1), (D, 3)\} \), dont le barycentre est \( K \).  
Pour tout point \( M \) du plan, la propriété 3) donne :  
\[
2\overrightarrow{MB} - \overrightarrow{MC} + 3\overrightarrow{MD} = 4\overrightarrow{MK}.
\]

En remplaçant \( M \) par \( G \), on obtient :  
\[
2\overrightarrow{GB} - \overrightarrow{GC} + 3\overrightarrow{GD} = 4\overrightarrow{GK}.
\]

Ajoutons \( -\overrightarrow{GA} \) à chaque membre :  
\[
-\overrightarrow{GA} + 2\overrightarrow{GB} - \overrightarrow{GC} + 3\overrightarrow{GD} = -\overrightarrow{GA} + 4\overrightarrow{GK}.
\]

Par définition de \( G \), le vecteur de gauche est nul, donc :  
\[
-\overrightarrow{GA} + 4\overrightarrow{GK} = \vec{0}.
\]

Ainsi, \( G \) est aussi le barycentre du système \( G'' = \{(A, -1), (K, 4)\} \).

\paragraph{Conclusion :}  
Ces calculs montrent que, dans la recherche du barycentre de plusieurs points, on peut regrouper certains points en sous-systèmes, remplacer ces sous-systèmes par leurs barycentres partiels, et affecter à ces barycentres la somme des coefficients des points regroupés (à condition que cette somme soit non nulle).

		
\subsection*{\underline{\textbf{\textcolor{red}{5) Contruction du barycentre de 4 points}}}}

Soient \( A \), \( B \), \( C \), \( D \) quatre points du plan et \( G \) le barycentre du système :  
\[
G = \{(A, m_A), (B, m_B), (C, m_C), (D, m_D)\},
\]
où \( m_A, m_B, m_C, m_D \) sont des coefficients réels tels que leur somme est non nulle :  
\[
m_A + m_B + m_C + m_D \neq 0.
\]

**Procédé utilisant l’associativité :**

1. **Étape 1 : Construction de barycentres partiels**
  
   - Regroupez les points \( A \) et \( B \) pour former un sous-système \( G_1 = \{(A, m_A), (B, m_B)\} \), dont le barycentre est noté \( I \).  
     Par définition, on a :  
     \[
     \overrightarrow{OI} = \frac{m_A \overrightarrow{OA} + m_B \overrightarrow{OB}}{m_A + m_B}.
     \]
   - Regroupez ensuite \( C \) et \( D \) pour former un autre sous-système \( G_2 = \{(C, m_C), (D, m_D)\} \), dont le barycentre est noté \( J \).  
     De même, on a :  
     \[
     \overrightarrow{OJ} = \frac{m_C \overrightarrow{OC} + m_D \overrightarrow{OD}}{m_C + m_D}.
     \]

2. **Étape 2 : Barycentre global des barycentres partiels**  

   Le barycentre \( G \) du système initial s’obtient comme barycentre des deux points pondérés \( I \) et \( J \) :  
   \[
   G = \{(I, m_A + m_B), (J, m_C + m_D)\}.
   \]
   Ainsi :  
   \[
   \overrightarrow{OG} = \frac{(m_A + m_B) \overrightarrow{OI} + (m_C + m_D) \overrightarrow{OJ}}{m_A + m_B + m_C + m_D}.
   \]

**Résumé du procédé :**  

1. Calculez les barycentres partiels \( I \) et \( J \) des sous-systèmes \( G_1 \) et \( G_2 \). 
 
2. Utilisez ces barycentres partiels pour déterminer \( G \) comme barycentre du système réduit \(\{(I, m_A + m_B), (J, m_C + m_D)\}\).

**Remarque :**  

Ce procédé permet de simplifier les calculs en regroupant progressivement les points tout en respectant la propriété d’associativité du barycentre.

Exemple 2

G bar (A,2),(B,1),(C,-3),(D,1)

I bar (A,2),(B,1)

J bar (C,-3),(D,1).

donc G bar (I,3),(J,-2)

\subsection*{\underline{\textbf{\textcolor{red}{6) Décomposer un vecteur de l’espace}}}}

Soit \( \vec{v} \) un vecteur de l’espace défini par deux points \( A(x_A, y_A, z_A) \) et \( B(x_B, y_B, z_B) \).  
La décomposition de \( \vec{v} = \overrightarrow{AB} \) dans une base orthonormée \((\vec{i}, \vec{j}, \vec{k})\) est donnée par :  
\[
\overrightarrow{AB} = (x_B - x_A) \vec{i} + (y_B - y_A) \vec{j} + (z_B - z_A) \vec{k}.
\]

**Exemple** :  
Si \( A(1, 2, 3) \) et \( B(4, 6, 5) \), alors :  
\[
\overrightarrow{AB} = (4-1)\vec{i} + (6-2)\vec{j} + (5-3)\vec{k} = 3\vec{i} + 4\vec{j} + 2\vec{k}.
\]

\textbf{Décomposition en projection sur des plans :}  
La projection de \( \overrightarrow{AB} \) sur les plans de l’espace peut être exprimée comme suit :
- Projection sur le plan \( Oxy \) : \( \vec{v}_{xy} = (x_B - x_A)\vec{i} + (y_B - y_A)\vec{j} \).
- Projection sur le plan \( Oxz \) : \( \vec{v}_{xz} = (x_B - x_A)\vec{i} + (z_B - z_A)\vec{k} \).
- Projection sur le plan \( Oyz \) : \( \vec{v}_{yz} = (y_B - y_A)\vec{j} + (z_B - z_A)\vec{k} \).

---

\subsection*{\underline{\textbf{\textcolor{red}{7) Propriété : Trois points et leur barycentre sont coplanaires}}}}

Soient \( A, B, C \) trois points non alignés de l’espace et \( G \) leur barycentre avec les coefficients \( m_A, m_B, m_C \).  
La propriété stipule que \( A, B, C \) et \( G \) sont toujours coplanaires.

\paragraph{Démonstration :}  
1. Le barycentre \( G \) est défini par :  
   \[
   \overrightarrow{OG} = \frac{m_A\overrightarrow{OA} + m_B\overrightarrow{OB} + m_C\overrightarrow{OC}}{m_A + m_B + m_C}.
   \]

2. Les vecteurs \( \overrightarrow{AB} \) et \( \overrightarrow{AC} \) définissent le plan contenant \( A, B, \) et \( C \).  
   Puisque \( \overrightarrow{OG} \) est une combinaison linéaire de \( \overrightarrow{OA}, \overrightarrow{OB}, \) et \( \overrightarrow{OC} \), \( \overrightarrow{OG} \) appartient au plan de base \( (A, B, C) \).

3. Par conséquent, \( G \) appartient au même plan que \( A, B, \) et \( C \).  

Ainsi, les points \( A, B, C \) et \( G \) sont coplanaires.

---

\subsection*{\underline{\textbf{\textcolor{red}{8) Centre d’inertie d’une plaque homogène}}}}

\paragraph{Cas 1 : Plaque rectangulaire}  
Pour une plaque homogène rectangulaire de dimensions \( a \times b \), le centre d’inertie se situe au centre géométrique, soit :  
\[
(x_G, y_G) = \left(\frac{a}{2}, \frac{b}{2}\right).
\]

\paragraph{Cas 2 : Disque évidé d’un autre disque}  
Considérons un disque de rayon \( R \), homogène, évidé d’un disque concentrique de rayon \( r \).  
1. La densité surfacique est uniforme.  
2. Le centre d’inertie reste au centre géométrique des disques, soit \( (x_G, y_G) = (0, 0) \), car la répartition de masse est symétrique.

\paragraph{Cas 3 : Disque évidé d’une figure simple (excentrée)}  
Supposons un disque \( D \) de rayon \( R \) évidé d’un disque \( D' \) de rayon \( r \), avec les centres des deux disques situés respectivement en \( O \) et \( O' \).  
1. On calcule les masses des deux parties :  
   \[
   m_D = \rho \pi R^2, \quad m_{D'} = \rho \pi r^2.
   \]
2. Le barycentre du système se calcule comme :  
   \[
   \overrightarrow{OG} = \frac{m_D \overrightarrow{O} - m_{D'} \overrightarrow{O'}}{m_D - m_{D'}}.
   \]
   La contribution de \( D' \) est soustraite, car la masse est enlevée.

\paragraph{Résumé :}  
- Le centre d’inertie d’une plaque homogène simple est son centre géométrique.  
- Pour des figures évidées, le centre d’inertie dépend de la position et de la masse relative des parties évidées.

\subsection*{\underline{\textbf{\textcolor{red}{Transformation d'écriture}}}}
\section*{\underline{\textbf{\textcolor{red}{II. Géométrie analytique (droites et repères)}}}}
		Une base du plan est un couple $(\vec{u}\;,\ \vec{v})$ de vecteurs non colinéaires.
		
		Le triplet $(O\;,\ \vec{u}\;,\ \vec{v})$ où $O$ est un point du plan et $(\vec{u}\;,\ \vec{v})$ une base est un repère cartésien du plan.
		
		Soit $(\vec{i}\;,\ \vec{j})$ une base du plan.
		
		Soit $\vec{u}$ un vecteur quelconque.
		
		Il existe un couple unique $(x\;,\ y)$ de réels tels que :
		
		$\vec{u}=x\vec{i}+y\vec{j}.$
		
		Ces réels $x$ et $y$ sont appelés coordonnées de $\vec{u}$ dans la base $(\vec{i}\;,\ \vec{j}).$
		
		Soit $(O\;,\ \vec{i}\;,\ \vec{j})$ un repère du plan.
		
		Soit $M$ un point quelconque.
		
		Il existe un couple unique $(x\;,\ y)$ de réels tels que :
		
		$\overrightarrow{OM}=x\vec{i}+y\vec{j}.$
		
		Ces réels $x$ et $y$ sont appelés coordonnées de $M$ dans le repère $(O\;,\ \vec{i}\;,\ \vec{j}).$
		
		
		Condition de colinéarité :
		
		Soit $\vec{u}(x\;,\ y)$ et $\vec{v}(x'\;,\ y')$ dans une base $(\vec{i}\;,\ \vec{j}).$
		
		Alors $\vec{u}$ et $\vec{v}$ sont colinéaires si et seulement si $xy'-yx'=0$ $\ (\Leftrightarrow$ dét$(\vec{u}\;,\ \vec{v})=0).$
		
		Soit dans le plan $\mathcal{P}$ muni du repère $(O\;,\ \vec{i}\;,\ \vec{j})$ les points $A\left(x_{A}\;,\ y_{A}\right)$ et $B\left(x_{B}\;,\ y_{B}\right).$
		
		Alors $\overrightarrow{AB}$ a pour coordonnées $\overrightarrow{AB}\left(x_{B}-x_{A}\;,\ y_{B}-y_{A}\right)$ et le milieu $I$ du segment $[AB]$ a pour coordonnées $I\left(\dfrac{x_{A}+x_{B}}{2}\;;\ \dfrac{y_{A}+y_{B}}{2}\right).$
		
		Représentations analytique d'une droite
		
		Soit $(O\;,\ \vec{i}\;,\ \vec{j})$ un repère du plan.
		
		Soit $\mathcal{D}$ la droite passant par $A(x_{0}\;;\ y_{0})$ et de vecteur directeur $\overrightarrow{u}(\alpha\;,\ \beta).$
		
		Un point $M(x\;,\ y)$ appartient à $\mathcal{D}$ si et seulement si $\overrightarrow{AM}$ et $\overrightarrow{u}$ sont colinéaires , c'est-à-dire s'il existe un réel $t$ tel que : 
		
		$\overrightarrow{AM}=t\overrightarrow{AB}$, ce qui se traduit par le système suivant :
		
		$$\left\lbrace\begin{array}{lcl} x& =& x_{A}+t\alpha\\ \\ y& =& y_{A}+t\beta \end{array}\right.$$
		
		appelé système d'équations paramétriques de la droite $\mathcal{D}.$
		
		Soit $\mathcal{D}$ la droite passant par deux points $A$ et $B.$ 
		
		Un point $M(x\;,\ y)$ appartient à $\mathcal{D}$ si et seulement si $\overrightarrow{AM}$ et $\overrightarrow{AB}$ sont colinéaires, ce qui équivaut à dét $ \left(\overrightarrow{AM}\;,\ \overrightarrow{AB}\right)=0$ et se traduit par une relation de la forme : 
		
		$ax+by+c=0$ appelé équation cartésienne de la droite $\mathcal{D}.$
		
		Si $b\neq 0$, cette équation peut se mettre sous la forme $y=mx+p$ et s'appelle alors équation réduite de la droite $\mathcal{D}$ : 
		
		$m$ s'appelle le coefficient directeur de $\mathcal{D}$ (si le repère $(O\;,\ \vec{i}\;,\ \vec{j})$ est orthogonal, $m$ est aussi appelé pente de la droite $\mathcal{D}).$ 
		
		$p$ est l'ordonnée à l'origine.
		
		Coordonnées d'un vecteur directeur de $\mathcal{D}$ :
		
		$\bullet\ $ équation cartésienne $ax+by+c=0$ : $\overrightarrow{u}(-b\;;\ a).$
		
		$\bullet\ $ système d'équations paramétriques $$\left\lbrace\begin{array}{lcl} x& = & x_{A}+t\alpha\\ \\ y& =& y_{A}+t\beta \end{array}\qquad \vec{u}(\alpha\;,\ \beta)\right.$$
		
		$\bullet\ $ équation réduite $y=mx+p$ : $\ \vec{u}(1\;,\ m).$
		
		Conditions de parallélisme
		
		$\left.\begin{array}{lcl} \mathcal{D}\ :\ ax+by+c& = &0\\ \mathcal{D'}\ :\ a'x+b'y+c'& = &0 \end{array}\right\rbrace\qquad\mathcal{D}-\mathcal{D'}\Leftrightarrow ab'-ba'=0$
		
		$\left.\begin{array}{lcl} \mathcal{D}\ :\ y& =& mx+p\\ \mathcal{D'}\ :\ y& =& m'x+p' \end{array}\right\rbrace\qquad\mathcal{D}\times\mathcal{D'}\Leftrightarrow m=m'$
		
\end{document}