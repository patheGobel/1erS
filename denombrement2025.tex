\documentclass[a4paper,12pt]{article}
\usepackage{amsmath, amssymb, mathrsfs}
\usepackage{xcolor}
\usepackage{colortbl}
\usepackage[utf8]{inputenc}
\usepackage[T1]{fontenc}
\usepackage[french]{babel}
\usepackage{tikz}
\usetikzlibrary{calc}
\usepackage{yhmath}
\usepackage{tcolorbox}
\usepackage{enumitem}
\usepackage{graphicx}
\usepackage{tkz-tab}
\usepackage[top=1.8cm, bottom=2cm, left=2cm, right=2cm]{geometry}

\begin{document}
\small

% En-tête personnalisée
\begin{center}
    \Large\textbf{\underline{Dénombrement}}\\[-0.1cm]
    \normalsize\textbf{Prof : M. BA} \hfill \textbf{Classe : Première S2}\\[-0.1cm]
    \textbf{Année scolaire : 2024 -- 2025}
\end{center}

% Titre rouge
\section*{\underline{\textcolor{red}{I. Rappel sur la théorie des ensembles}}}

\subsubsection*{\underline{\textcolor{red}{1. Définition et vocabulaire}}}

Un ensemble est un regroupement d’objet défini par une caractéristique commune ou par une énumération complète.\\
Chaque objet de l’ensemble est appelé un \textcolor{red}{\underline{élément}}.\\
Un ensemble est dit \textcolor{red}{\underline{fini}} s’il est constitué d’un nombre fini d’éléments et ce nombre est appelé le \textcolor{red}{\underline{cardinal}} de l’ensemble.\\
Il existe un ensemble qui ne contient aucun élément, il est appelé \textcolor{red}{\underline{l’ensemble vide}} noté \textcolor{red}{\( \varnothing \)}.

\textbf{\underline{\textcolor{red}{Exemples :}}}\\
$E = \{ a, b, c, d, e \}$ \quad ; \quad $J = \{ x \in \mathbb{R} \mid x \geq 2 \}$\\
$K = \{ 1, 2, 3, 4, 5, 6 \}$ \quad ; \quad $P = \{ n \in \mathbb{N} \mid 10 \leq n \leq 100 \}$

\textbf{\underline{\textcolor{red}{Solution :}}}\\
\textbf{$E$, $K$ et $P$ sont des ensembles finis.}\\
$\text{card}(E) = 5$ \quad ; \quad $\text{card}(P) = \textcolor{red}{91}$ \quad ; \quad $\text{card}(K) = 6$ \quad ; \quad $\text{card}(\varnothing) = 0$

\subsubsection*{\underline{\textcolor{red}{2. Partie d’un ensemble :}}}

Soit $E$ un ensemble fini de cardinal $n$.\\
$A$ est dit un sous-ensemble ou une partie de $E$ ssi $x \in A \Rightarrow x \in E$ $\Leftrightarrow$ \textcolor{red}{\underline{$A \subset E$}} \quad (\textcolor{red}{\small $A$ inclue dans $E$})\\
On écrit que $A \in \mathcal{P}(E)$ où $\mathcal{P}(E)$ est l’ensemble des sous-ensembles de $E$ et 
\[
\textcolor{red}{\underline{\text{card} \; \mathcal{P}(E) = 2^{\text{card}(E)}} \quad \Rightarrow \quad \text{card} \; \mathcal{P}(E) = 2^n}
\]

\textcolor{red}{\underline{exemple :}} Soit $E = \{1\, ;\, 2\, ;\, 3\}$\\
$\text{card}(E) = 3$\\
$\text{card}(\mathcal{P}(E)) = 2^3 = 8$\\
\[
\mathcal{P}(E) = \left\{ 
E\, ;\,
\varnothing\, ;\,
\{1\}\, ;\,
\{2\}\, ;\,
\{3\}\, ;\,
\{1,2\}\, ;\,
\{1,3\}\, ;\,
\{2,3\}
\right\}
\]

\end{document}