\documentclass[12pt]{article}
\usepackage{stmaryrd}
\usepackage{graphicx}
\usepackage[utf8]{inputenc}
\usepackage[french]{babel}
\usepackage[T1]{fontenc}
\usepackage{hyperref}
\usepackage{verbatim}
\usepackage{color,soul}
\usepackage{amsmath}
\usepackage[table]{xcolor}
\usepackage{amsfonts}
\usepackage{amssymb}
\usepackage{systeme}
\usepackage{tkz-tab}
\author{Destiné à la 1èrS2\\Au Lycée de Dindéfelo}
\title{\textbf{Série Sur les Fonctions Numériques}}
\date{\today}
\usepackage{tikz}
\usetikzlibrary{arrows}
\usepackage[a4paper,left=20mm,right=20mm,top=15mm,bottom=15mm]{geometry}
\usepackage{mathtools}
\usepackage{systeme}

\usepackage{pgfplots}
\pgfplotsset{compat=1.15}
\usepackage{mathrsfs}

\usetikzlibrary{arrows}
\pagestyle{empty}

\DecimalMathComma

\usepackage{eso-pic} % Pour ajouter des éléments en arrière-plan

% Commande pour ajouter du texte en arrière-plan, centré au milieu de chaque page
\AddToShipoutPicture{
    \AtPageCenter{%
        \makebox(0,0)[c]{\rotatebox{60}{\textcolor[gray]{0.9}{\fontsize{2cm}{2cm}\selectfont Pathé Gobel BA}}}
    }
}

\begin{document}

\maketitle
\newpage
\section*{\underline{\textbf{\textcolor{red}{I. Equations}}}}
\subsection*{\underline{\textbf{\textcolor{red}{1. Compléments sur les trinômes du \( 2^{nd} \)degré}}}}
\subsection*{\underline{\textbf{\textcolor{red}{2. Equations irrationnelles}}}}
\subsubsection*{\underline{\textbf{\textcolor{red}{a. Équations du type \( \sqrt{f(x)} = \sqrt{g(x)} \)}}}}

\( \sqrt{x^{2}+3x-1} = \sqrt{x+2} \) est une équation irrationnelle du type \( \sqrt{f(x)} = \sqrt{g(x)} \) avec :

\[ f(x) = x^{2}+3x-1 \quad \text{et} \quad g(x) = x+2. \]

Pour résoudre une équation du type \( \sqrt{f(x)} = \sqrt{g(x)} \), on peut utiliser deux méthodes :

\begin{itemize}
    \item[•] \textbf{Méthode par équivalence}
\end{itemize}

\begin{itemize}
    \item \textbf{1ère étape :} Vérification des conditions de validité.
\end{itemize}

On a les équivalences suivantes :

\[
\sqrt{ f(x)} = \sqrt{ g(x)} \Leftrightarrow \begin{cases}
f(x) \geq 0 \\
g(x) \geq 0 \\
f(x) = g(x)
\end{cases}
\]

**Explication :** Pour que les deux racines soient définies et égales, il faut que \( f(x) \) et \( g(x) \) soient toutes deux positives, d'où les conditions \( f(x) \geq 0 \) et \( g(x) \geq 0 \). Ensuite, il faut que les deux expressions soient égales, soit \( f(x) = g(x) \).

En simplifiant ces conditions, on obtient :

\[
\Leftrightarrow \begin{cases}
g(x) \geq 0 \\
f(x) = g(x)
\end{cases}
\]

**Explication :** Ici, on peut réduire les conditions en ne gardant que \( g(x) \geq 0 \), car l'égalité \( f(x) = g(x) \) implique automatiquement que \( f(x) \geq 0 \) lorsque \( g(x) \geq 0 \).

Finalement, cela se résume à :

\[
\Leftrightarrow \begin{cases}
f(x) \geq 0 \\
f(x) = g(x)
\end{cases}
\]

**Explication :** Cette équivalence montre que, pour que l'équation soit valide, il faut vérifier que \( f(x) \geq 0 \), tout en résolvant l'équation \( f(x) = g(x) \).

 \textbf{Exemple} 
 
\( \sqrt{ 2x+1} = \sqrt{ x+3}  \)

\( \sqrt{ x^{2}+3x-1} = \sqrt{ x+2}  \)

 \textbf{solution} 

\begin{itemize}
    \item[•] \textbf{Étape 1 : Conditions de validité \( D_{v} \)}
\end{itemize}

L'équation existe ssi \( x^2 + 3x - 1 \geq 0 \) et \( x + 2 \geq 0 \). 

 La deuxième condition nous donne \( x \geq -2 \).

\( D_{v} =[-2; +\infty[ \)

\begin{itemize}
    \item[•] \textbf{Étape 2 : Résolution de l'équation}
\end{itemize}

Élevons les deux membres de l'équation au carré :

\[
x^2 + 3x - 1 = x + 2
\]

Cela revient à résoudre l'équation :

\[
x^2 + 3x - 1 = x + 2 \quad \Leftrightarrow \quad x^2 + 2x - 3 = 0
\]


\[
 x = 1 \quad \text{ou} \quad x = -3
\]

\textbf{Conclusion :} La seule solution est \( x = 1 \).

 
\subsubsection*{\underline{\textbf{\textcolor{red}{b. Equations du type \( \sqrt{f(x)} = ax+b \) }}}}
 \[\text{On a les équivalences suivantes :}\]
	\[ \sqrt{ f(x)} = g(x) \Leftrightarrow \begin{cases}f(x) \geq 0 \\ g(x) \geq 0 \\ f(x) = g(x)^{2} \end{cases} \Leftrightarrow \begin{cases} g(x) \geq 0 \\ f(x) = g(x)^{2} \end{cases} \]

 \textbf{Exemple } 
 
 Résoudre dans $\mathbb{R}$
 
\( \sqrt{ 2x+1} = \sqrt{ x+3}  \)

\( \sqrt{ x^{2}+3x-1} = \sqrt{ x+2}  \)	
	
\section*{\underline{\textbf{\textcolor{red}{II. Inéquations irrationnelles}}}}

\subsection*{\underline{\textbf{\textcolor{red}{1. Inéquations du type \( \sqrt{f(x)} \leq ax+b \) }}}}
\subsection*{\underline{\textbf{\textcolor{red}{2. Inéquations du type \( \sqrt{f(x)} \geq ax+b \) }}}}
\subsection*{\underline{\textbf{\textcolor{red}{3. Inéquations du type \( \sqrt{f(x)} \leq  \sqrt{g(x)} \) }}}}

\section*{\underline{\textbf{\textcolor{red}{III. Systèmes}}}}
\subsection*{\underline{\textbf{\textcolor{red}{1. Systèmes de 3 équations linéaires à trois inconnues}}}}
\subsubsection*{\underline{\textbf{\textcolor{red}{a. Exemple}}}}
\subsubsection*{\underline{\textbf{\textcolor{red}{b. Méthode du pivot de Gauss }}}}
\subsection*{\underline{\textbf{\textcolor{red}{2. Systèmes d’inéquations linéaires }}}}
\subsubsection*{\underline{\textbf{\textcolor{red}{a. Inéquations linéaires à deux inconnues}}}}
\subsubsection*{\underline{\textbf{\textcolor{red}{b. Système de deux inéquations linéaires à deux inconnues }}}}
\end{document}
