\documentclass[12pt]{article}
\usepackage{stmaryrd}
\usepackage{graphicx} % Pour l'insertion d'images
\usepackage{float}    % Pour contrôler précisément le placement
\usepackage[utf8]{inputenc}

\usepackage[french]{babel}
\usepackage[T1]{fontenc}
\usepackage{hyperref}
\usepackage{verbatim}

\usepackage{color, soul}

\usepackage{pgfplots}
\pgfplotsset{compat=1.15}
\usepackage{mathrsfs}

\usepackage{amsmath}
\usepackage{amsfonts}
\usepackage{amssymb}
\usepackage{tkz-tab}
\author{Destiné aux élèves de la 1er S2\\Lycée de Dindéfelo\\Présenté par M. BA}
\title{\textbf{Equations-Inéquations-Systèmes}}
\date{\today}
\usepackage{tikz}
\usetikzlibrary{arrows, shapes.geometric, fit}

% Commande pour la couleur d'accentuation
\newcommand{\myul}[2][black]{\setulcolor{#1}\ul{#2}\setulcolor{black}}
\newcommand\tab[1][1cm]{\hspace*{#1}}

\usepackage[margin=2cm]{geometry}
\usepackage{eso-pic}         % Pour ajouter des éléments en arrière-plan

\usepackage{enumitem}
%---------------------------------------
% Définir un compteur pour les exemples
\newcounter{exemple}

% Définir la commande \exemple pour afficher un exemple numéroté
\newcommand{\exemple}{%
  \refstepcounter{exemple}% Incrémenter le compteur
  \textbf{\textcolor{orange}{Exemple \theexemple : }} \ignorespaces
}
%---------------------------------------
\newcounter{solution}

% Définir la commande \solutione pour afficher un solution numéroté
\newcommand{\solution}{%
  \refstepcounter{solution}% Incrémenter le compteur
  \textbf{\textcolor{orange}{Solution \thesolution : }} \ignorespaces
}
%---------------------------------------
\definecolor{myorange}{rgb}{1.0, 0.8, 0.0}

% Définir un compteur pour les exercices d'application
\newcounter{exerciceapp}

% Définir la commande pour afficher un exercice d'application numéroté
\newcommand{\exerciceapp}{%
  \refstepcounter{exerciceapp}%
  \textbf{\textcolor{myorange}{Exercice d'application \theexerciceapp :}} \ignorespaces
}
%--------------------------------------
% Définir un compteur pour les exercices d'application
\newcounter{correction}

% Définir la commande pour afficher un correction exercice d'application numéroté
\newcommand{\correction}{%
  \refstepcounter{correction}%
  \textbf{\textcolor{myorange}{Correction \thecorrection :}} \ignorespaces
}
%--------------------------------------
% Définir un compteur pour les remarque d'application
\newcounter{remarque}

%----------------------------------------
\definecolor{myorange1}{rgb}{1.0, 1.5, 0}
% Définir la commande pour afficher un remarque numéroté
\newcommand{\remarque}{%
  \refstepcounter{remarque}%
  \textbf{\textcolor{myorange1}{Remarque \theremarque :}} \ignorespaces
}
% Commande pour ajouter du texte en arrière-plan
\AddToShipoutPicture{
    \AtTextCenter{%
        \makebox[0pt]{\rotatebox{45}{\textcolor[gray]{0.9}{\fontsize{5cm}{5cm}\selectfont Pathé Gobel BA}}}
    }
}

\begin{document}

\maketitle
\newpage
\section*{\underline{\textbf{\textcolor{red}{I. Equations}}}}
\subsection*{\underline{\textbf{\textcolor{red}{1. Compléments sur les trinômes du \( 2^{nd} \)degré}}}}
\subsection*{\underline{\textbf{\textcolor{red}{2. Equations irrationnelles}}}}
\subsubsection*{\underline{\textbf{\textcolor{red}{a. Équations du type \( \sqrt{f(x)} = \sqrt{g(x)} \)}}}}

\( \sqrt{x^{2}+3x-1} = \sqrt{x+2} \) est une équation irrationnelle du type \( \sqrt{f(x)} = \sqrt{g(x)} \) avec :

\[ f(x) = x^{2}+3x-1 \quad \text{et} \quad g(x) = x+2. \]

Pour résoudre une équation du type \( \sqrt{f(x)} = \sqrt{g(x)} \), on peut utiliser deux méthodes :

\underline{\textbf{Méthode de Résolution}}

\begin{itemize}
    \item \textbf{1ère étape :} Vérification des conditions de validité.
\end{itemize}

On a les équivalences suivantes :

\[
\sqrt{ f(x)} = \sqrt{ g(x)} \Leftrightarrow \begin{cases}
f(x) \geq 0 \\
g(x) \geq 0 \\
f(x) = g(x)
\end{cases}
\]

**Explication :** Pour que les deux racines soient définies et égales, il faut que \( f(x) \) et \( g(x) \) soient toutes deux positives, d'où les conditions \( f(x) \geq 0 \) et \( g(x) \geq 0 \). Ensuite, il faut que les deux expressions soient égales, soit \( f(x) = g(x) \).

En simplifiant ces conditions, on obtient :

\[
\Leftrightarrow \begin{cases}
g(x) \geq 0 \\
f(x) = g(x)
\end{cases}
\]

**Explication :** Ici, on peut réduire les conditions en ne gardant que \( g(x) \geq 0 \), car l'égalité \( f(x) = g(x) \) implique automatiquement que \( f(x) \geq 0 \) lorsque \( g(x) \geq 0 \).

Finalement, cela se résume à :

\[
\Leftrightarrow \begin{cases}
f(x) \geq 0 \\
f(x) = g(x)
\end{cases}
\]

**Explication :** Cette équivalence montre que, pour que l'équation soit valide, il faut vérifier que \( f(x) \geq 0 \), tout en résolvant l'équation \( f(x) = g(x) \).

 \textbf{\exemple}

Résoudre l'équation:

\[
\sqrt{2x + 1} = \sqrt{x + 3} \quad \text{et} \quad \sqrt{x^2 + 3x - 1} = \sqrt{x + 2}
\]

\textbf{\solution}

\[
\sqrt{2x + 1} = \sqrt{x + 3}
\]


\begin{itemize}
    \item \textbf{Étape 1 : Conditions de validité $D_v$}
    
    Pour que les racines carrées existent, il faut que les expressions à l'intérieur des racines soient positives ou nulles :
    \[
    2x + 1 \geq 0 \implies x \geq -\frac{1}{2}
    \]
    \[
    x + 3 \geq 0 \implies x \geq -3
    \]
    Donc, la condition de validité est :
    \[
    D_v = \left[-\frac{1}{2}; +\infty \right[
    \]
    
    \item \textbf{Étape 2 : Résolution de l'équation}
    
    On élève les deux membres de l'équation au carré :
    \[
    (\sqrt{2x + 1})^2 = (\sqrt{x + 3})^2
    \]
    Ce qui donne :
    \[
    2x + 1 = x + 3
    \]
    
    \item \textbf{Étape 3 : Simplification de l'équation}
    
    Isolons $x$ :
    \[
    2x + 1 = x + 3
    \]
    \[
    2x - x = 3 - 1
    \]
    \[
    x = 2
    \]
    
    \item \textbf{Étape 4 : Vérification de la solution}
    
    Vérifions que $x = 2$ appartient bien à l'ensemble de validité $D_v$ :
    
    \begin{itemize}
        \item $x = 2$ est bien dans $\left[-\frac{1}{2}; +\infty \right[$.
        \item Vérification dans l'équation initiale :
        \[
        \sqrt{2 \times 2 + 1} = \sqrt{2 + 3}
        \]
        \[
        \sqrt{5} = \sqrt{5}
        \]
    \end{itemize}
    
    La solution est donc correcte.
    
\end{itemize}

\textbf{Conclusion :} La solution de l'équation est $x = 2$.

\[
\sqrt{x^2 + 3x - 1} = \sqrt{x + 2}
\]

\begin{itemize}
    \item \textbf{Étape 1 : Conditions de validité $D_v$}
    
    L'équation existe ssi $x^2 + 3x - 1 \geq 0$ et $x + 2 \geq 0$.\\
    La deuxième condition nous donne $x \geq -2$.\\
    Donc $D_v = [-2; +\infty[$.
    
    \item \textbf{Étape 2 : Résolution de l'équation}
    
    Élevons les deux membres de l'équation au carré :
    \[
    x^2 + 3x - 1 = x + 2
    \]
    Cela revient à résoudre l'équation :
    \[
    x^2 + 3x - 1 = x + 2 \implies x^2 + 3x - x - 2 = 0
    \]
    \[
    x^2 + 2x - 3 = 0
    \]
    
    \textbf{Conclusion :} La seule solution est $x = 1$.
\end{itemize}

\begin{itemize}
    \item[•] \textbf{Étape 1 : Conditions de validité}
\end{itemize}

On doit vérifier que \( f(x) = x^2 + 3x - 1 \geq 0 \) et \( g(x) = x + 2 \geq 0 \). La deuxième condition nous donne \( x \geq -2 \).

\begin{itemize}
    \item[•] \textbf{Étape 2 : Résolution de l'équation}
\end{itemize}

Élevons les deux membres de l'équation au carré :

\[
x^2 + 3x - 1 = x + 2
\]

Cela revient à résoudre l'équation :

\[
x^2 + 3x - 1 = x + 2 \quad \Leftrightarrow \quad x^2 + 2x - 3 = 0
\]

\begin{itemize}
    \item[•] \textbf{Étape 3 : Résolution de l'équation quadratique}
\end{itemize}

On résout \( x^2 + 2x - 3 = 0 \) par factorisation :

\[
(x - 1)(x + 3) = 0 \quad \Rightarrow \quad x = 1 \quad \text{ou} \quad x = -3
\]

\begin{itemize}
    \item[•] \textbf{Étape 4 : Vérification des solutions}
\end{itemize}

- Pour \( x = 1 \) :

\[
f(1) = 1^2 + 3(1) - 1 = 3 \quad \text{et} \quad g(1) = 1 + 2 = 3
\]

Les deux fonctions sont positives et égales, donc \( x = 1 \) est une solution.

- Pour \( x = -3 \) :

\[
f(-3) = (-3)^2 + 3(-3) - 1 = 9 - 9 - 1 = -1 \quad \text{et} \quad g(-3) = -3 + 2 = -1
\]

Les deux fonctions sont négatives, donc \( x = -3 \) n'est pas une solution.

\textbf{Conclusion :} La seule solution est \( x = 1 \).

\subsubsection*{\underline{\textbf{\textcolor{red}{b. Équations du type \( \sqrt{f(x)} = ax + b \)}}}}

\underline{\textbf{Méthode de Résolution}}

\[\text{On a les équivalences suivantes :}\]

\[
\sqrt{f(x)} = g(x) \Leftrightarrow \begin{cases} 
f(x) \geq 0 \\
g(x) \geq 0 \\
f(x) = g(x)^{2} 
\end{cases}
\]

**Explication :**
  
- \( f(x) \geq 0 \) : Pour que la racine carrée \( \sqrt{f(x)} \) soit définie, il est nécessaire que l'expression sous la racine soit positive ou nulle, d'où \( f(x) \geq 0 \).

- \( g(x) \geq 0 \) : Comme la racine carrée d'un nombre est toujours positive, il faut également que \( g(x) \), qui représente l'autre membre de l'équation, soit positif ou nul.

- \( f(x) = g(x)^2 \) : Une fois ces conditions vérifiées, on élève les deux membres de l'équation au carré afin d'éliminer la racine carrée, ce qui donne \( f(x) = g(x)^2 \).

En simplifiant, on obtient :

\[
\Leftrightarrow \begin{cases}
g(x) \geq 0 \\
f(x) = g(x)^{2}
\end{cases}
\]

**Explication :**
  
Dans certains cas, la condition \( f(x) \geq 0 \) devient redondante, car l'égalité \( f(x) = g(x)^2 \) implique que \( f(x) \) est nécessairement positive si \( g(x) \geq 0 \).

 \textbf{\exemple}

Résoudre l'équation:

\( \sqrt{2x^{2}+4x-6} = 2x - 1 \)

\textbf{\solution}

\section*{\underline{\textbf{\textcolor{red}{II. Inéquations irrationnelles}}}}
\subsection*{\underline{\textbf{\textcolor{red}{1. Inéquations du type \( \sqrt{f(x)} \leq ax+b \) }}}}

\underline{\textbf{Méthode de Résolution}}

\[\text{On a les équivalences suivantes :}\]

\[
\sqrt{f(x)} \leq g(x) \Leftrightarrow \begin{cases} 
f(x) \geq 0 \\
g(x) \geq 0 \\
f(x) \leq g(x)^{2}
\end{cases}
\]

**Explication :**  

- **\( f(x) \geq 0 \)** : L'expression sous la racine carrée doit être positive ou nulle pour que la racine soit définie.

- **\( g(x) \geq 0 \)** : Comme la racine carrée est toujours positive ou nulle, il est nécessaire que l'expression \( g(x) \), ici \( ax + b \), soit aussi positive ou nulle.

- **\( f(x) \leq g(x)^2 \)** : On élève les deux membres de l'inéquation au carré pour éliminer la racine, ce qui donne l'inéquation \( f(x) \leq g(x)^2 \).


\textbf{\exemple}

Résolvons l'inéquation \( \sqrt{2x+1} \leq x+3 \).

\textbf{\solution}

\begin{itemize}
    \item \textbf{Étape 1 : Conditions de validité \( D_v \)}
\end{itemize}

L'inéquation est définie si \( 2x + 1 \geq 0 \) et \( x + 3 \geq 0 \).

- \( 2x + 1 \geq 0 \) donne \( x \geq -\frac{1}{2} \).

- \( x + 3 \geq 0 \) donne \( x \geq -3 \).

Le domaine de validité de l'inéquation est donc \( D_v = [-\frac{1}{2}; +\infty[ \).

\begin{itemize}
    \item \textbf{Étape 2 : Résolution de l'inéquation}
\end{itemize}

On élève les deux membres de l'inéquation au carré :

\[
2x + 1 \leq (x + 3)^2
\]

Cela revient à résoudre l'inéquation :

\[
2x + 1 \leq (x + 3)^2 \quad \Leftrightarrow \quad 2x + 1 \leq x^2 + 6x + 9.
\]

En simplifiant, on obtient :

\[
0 \leq x^2 + 4x + 8.
\]

Cette inéquation est toujours vraie pour tout \( x \in \mathbb{R} \), donc elle est vérifiée pour tout \( x \geq -\frac{1}{2} \).

\textbf{Conclusion :} L'ensemble des solutions est \( S = [-\frac{1}{2}; +\infty[ \).

\textbf{\exemple}

Résolvons l'inéquation \( \sqrt{x^{2}+3x-1} \leq x+2 \).

\textbf{\solution}

\begin{itemize}
    \item \textbf{Étape 1 : Conditions de validité \( D_v \)}
\end{itemize}

L'inéquation est définie si \( x^{2} + 3x - 1 \geq 0 \) et \( x + 2 \geq 0 \).

- \( x + 2 \geq 0 \) donne \( x \geq -2 \).

- Pour \( x^{2} + 3x - 1 \geq 0 \), on trouve les racines du polynôme : \( x = -3.41 \) et \( x = 0.41 \).

Le domaine de validité est donc \( D_v = [-2; +\infty[ \).

\begin{itemize}
    \item \textbf{Étape 2 : Résolution de l'inéquation}
\end{itemize}

On élève les deux membres de l'inéquation au carré :

\[
x^{2} + 3x - 1 \leq (x + 2)^{2}
\]

Cela revient à résoudre l'inéquation :

\[
x^{2} + 3x - 1 \leq (x + 2)^{2} \quad \Leftrightarrow \quad x^{2} + 3x - 1 \leq x^{2} + 4x + 4.
\]

En simplifiant, on obtient :

\[
0 \leq x + 5.
\]

Cette inéquation est vérifiée pour tout \( x \geq -5 \).

\textbf{Conclusion :} L'ensemble des solutions est \( S = [-2; +\infty[ \), car il faut tenir compte des conditions initiales.

\subsection*{\underline{\textbf{\textcolor{red}{2. Inéquations du type \( \sqrt{f(x)} \geq ax+b \) }}}}

\underline{\textbf{Méthode de Résolution}}

\[\text{On a les équivalences suivantes :}\]

\[
\sqrt{f(x)} \geq g(x) \Leftrightarrow \begin{cases} 
f(x) \geq 0 \\
g(x) \geq 0 \\
f(x) \geq g(x)^{2}
\end{cases}
\textbf{ ou }
\begin{cases}
f(x) \geq 0 \\
g(x) \leq 0
\end{cases}
\]

**Explication :**  

- **\( f(x) \geq 0 \)** : Comme la racine carrée \( \sqrt{f(x)} \) doit être définie, il est nécessaire que

 l'expression sous la racine soit positive ou nulle.

- **\( g(x) \geq 0 \)** : La racine carrée étant toujours positive ou nulle, il faut également que 

\( g(x) = ax + b \) soit positive ou nulle.

- **\( f(x) \geq g(x)^2 \)** : En élevant les deux membres de l'inéquation au carré, on obtient 

\( f(x) \geq g(x)^2 \).

En simplifiant, on obtient :

\textbf{\exemple}

Résolvons l'inéquation \( \sqrt{2x+1} \geq x+3 \).

\textbf{\solution}


\[
\textbf{Posons }
S_{1}:
\begin{cases} 
f(x) \geq 0 \\
g(x) \geq 0 \\
f(x) \geq g(x)^{2}
\end{cases}
\textbf{Posons }
S_{2}:
\begin{cases}
f(x) \geq 0 \\
g(x) \leq 0
\end{cases}
\]

Ce qui correspond à :

\[
\textbf{Posons }
S_{1}:
\begin{cases} 
2x + 1 \geq 0 \\
x + 3 \geq 0 \\
2x + 1 \geq x + 3
\end{cases}
\textbf{Posons }
S_{2}:
\begin{cases}
2x + 1 \geq 0 \\ \\
x + 3 \leq 0 
\end{cases}
\]

\underline{Résolvons \( S_1 \)} :

1. \( 2x + 1 \geq 0 \) :  
   \( x \geq -\frac{1}{2} \).

2. \( x + 3 \geq 0 \) :  
   \( x \geq -3 \).

3. \( 2x + 1 \geq (x + 3)^2 \) :  
   Développons \( (x + 3)^2 \) :  
   \[
   (x + 3)^2 = x^2 + 6x + 9.
   \]
   Nous obtenons l'inéquation suivante :  
   \[
   2x + 1 \geq x^2 + 6x + 9.
   \]
   Cela revient à résoudre :
   \[
   0 \geq x^2 + 4x + 8.
   \]
   Le discriminant de \( x^2 + 4x + 8 \) est :
   \[
   \Delta = 4^2 - 4 \times 1 \times 8 = 16 - 32 = -16.
   \]
   Comme le discriminant est négatif, cette équation n'a pas de solution réelle. Ainsi, il n'existe aucune valeur de \( x \) pour laquelle \( 2x + 1 \geq (x + 3)^2 \).

\[
S_1 = \varnothing.
\]

\underline{Résolvons \( S_2 \)} :

1. \( 2x + 1 \geq 0 \) :  
   \( x \geq -\frac{1}{2} \).

2. \( x + 3 \leq 0 \) :  
   \( x \leq -3 \).

Les deux conditions \( x \geq -\frac{1}{2} \) et \( x \leq -3 \) sont contradictoires, donc :

\[
S_2 = \varnothing.
\]

\subsection*{\underline{\textbf{\textcolor{red}{3. Inéquations du type \( \sqrt{f(x)} \leq  \sqrt{g(x)} \) }}}}
On a : $\sqrt{f(x)}\leq\sqrt{g(x)}\;\Leftrightarrow\;\left\lbrace\begin{array}{lll} f(x) \geq 0 \\ g(x)\geq 0\\ f(x)\leq g(x)\end{array}\right.$

\textbf{\exemple}

Résoudre dans $\mathbb{R}\;\sqrt{x^{2}-4x}\leq\sqrt{x+7}$.

\textbf{\solution}

$\begin{array}{rcl} \sqrt{x^{2}-4x}\leq\sqrt{x+7} & \Leftrightarrow & \left\lbrace\begin{array}{rcl} x^{2}-4x&\geq&0 \\ x+7&\geq&0 \\ (\sqrt{x^{2}-4x})^{2}&\leq&(\sqrt{x+7})^{2} \end{array}\right. \\ \\ & \Leftrightarrow & \left\lbrace\begin{array}{rcl} x(x-4)&\geq&0 \\ x&\geq&-7 \\ x^{2}-4x&\leq&x+7 \end{array}\right. \\ \\ & \Leftrightarrow & \left\lbrace\begin{array}{lll} x\in\;(]-\infty\;;\ 0]\cup[4\;;\ +\infty[)\cap[-7\;;\ +\infty[ \\ x^{2}-5x-7\leq 0 \end{array}\right. \\ \\  & \Leftrightarrow & \left\lbrace\begin{array}{lll} x\in\;D_{E}=]-7\;;\ 0]\cup[4\;;\ +\infty[ \\ x^{2}-5x-7\leq 0 \end{array}\right. \end{array}$

Soit l'équation $x^{2}-5x-7=0$

On a : $\Delta=25+28=53\;\Rightarrow\;\sqrt{\Delta}=\sqrt{53}.$

Soient $x_{1}$ et $x_{2}$ les solutions distinctes de $x^{2}-5x-7=0$.

Alors, $x_{1}=\dfrac{5-\sqrt{53}}{2}\ $ et $\ x_{2}=\dfrac{5+\sqrt{53}}{2}$

Ainsi, sans aucune contrainte, l'inéquation $x^{2}-5x-7\leq 0$ admettra comme solution $S_{1}=[x_{1}\;;\ x_{2}]$

$$\begin{array}{|c|lclclclclcr|} \hline x & -7 & & x_{1} &  & 0 &  & 4 & & x_{2} & & +\infty \\ \hline x^{2}-4x & & + & | & + & 0 & - & 0 & + &|& + & \\ \hline x^{2}-5x-7 &  & + & 0 & - & | & - & | & - & 0 & + & \\ \hline S &  & + & 0 & (-) & | & + & |& (-) & 0 & + & \\ \hline \end{array}$$

Donc en tenant compte des conditions initiales, l'ensemble des solutions du problème sera donné par $$S=S_{1}\cap D_{E}=[x_{1}\;;\ 0]\cup[4\;;\ x_{2}]$$

$\centerdot\ \ \sqrt{f(x)}\geq g(x)$

On a : $\sqrt{f(x)}\geq g(x)\;\Leftrightarrow\;\left\lbrace\begin{array}{lll} g(x)\geq 0\\ f(x)\geq (g(x))^{2} \end{array}\right.\quad\text{ou}\quad\left\lbrace\begin{array}{lll} f(x)\geq 0 \\ g(x)&lt; 0\end{array}\right.$

Donc, on obtient deux solutions $S_{1}$ et $S_{2}.$

Ainsi, la solution finale est donnée par $S=S_{1}\cup S_{2}.$


\textbf{\exerciceapp}

Résoudre dans \( \mathbb{R} \)

\begin{itemize}
\item \( \sqrt{x+3} \leq 5 \)
\item \( \sqrt{-2x+1} \geq -3 \)
\item \( \sqrt{x+3} \leq \sqrt{-2x+4} \)
\item \( \sqrt{x^{2}-4} \leq x+1 \)
\item \( \sqrt{x^{2}+2x -3} \leq x-2 \)
\item \( \sqrt{x^{2}-6x}=\sqrt{x-6} \)
\end{itemize}
\textbf{\correction}
\section*{\underline{\textbf{\textcolor{red}{III. Systèmes}}}}
\subsection*{\underline{\textbf{\textcolor{red}{1. Systèmes de 3 équations linéaires à trois inconnues}}}}

Le système 
\[
\begin{cases}
a_1x + b_1y + c_1z = d_1 \\
a_2x + b_2y + c_2z = d_2 \\
a_3x + b_3y + c_3z = d_3
\end{cases}
\]
est formé de trois équations. Chacune d’elle contient trois inconnues \(x\), \(y\) et \(z\) avec des exposants tous égaux à 1. On dit qu’on a un système de trois équations linéaires à trois inconnues \(x\), \(y\) et \(z\).

Résoudre un tel système c’est trouver tous les triplets \((x, y, z)\) de nombres réels qui vérifient les trois équations du système.

\subsubsection*{\underline{\textbf{\textcolor{red}{ Exemple }}}}

Exemple de système :

\[
\begin{cases}
2x + y - z = 3 \\
x - 2y + 3z = -4 \\
3x + y + 2z = 7
\end{cases}
\]

\subsubsection*{\underline{\textbf{\textcolor{red}{Exemple de système triangulaire}}}}

Un système triangulaire supérieur est un système d'équations où chaque équation contient une inconnue de moins que la précédente, formant un triangle.

\[
\begin{cases}
2x + 3y - z = 5 \\
0 \cdot x + y + 4z = -2 \\
0 \cdot x + 0 \cdot y + z = 3
\end{cases}
\]

Dans ce système, la dernière équation contient seulement \( z \), ce qui permet de le résoudre directement, puis de remonter en substituant dans les équations précédentes pour obtenir \( y \) et \( x \).

\subsubsection*{\underline{\textbf{\textcolor{red}{b. Méthode du pivot de Gauss }}}}

\subsection*{\underline{\textbf{\textcolor{red}{2. Systèmes d’inéquations linéaires }}}}
\subsubsection*{\underline{\textbf{\textcolor{red}{a. Inéquations linéaires à deux inconnues}}}}
\subsubsection*{\underline{\textbf{\textcolor{red}{b. Système de deux inéquations linéaires à deux inconnues }}}}
\end{document}
