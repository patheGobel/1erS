\documentclass[12pt]{article}
\usepackage{stmaryrd}
\usepackage{graphicx} % Pour l'insertion d'images
\usepackage{float}    % Pour contrôler précisément le placement
\usepackage[utf8]{inputenc}

\usepackage[french]{babel}
\usepackage[T1]{fontenc}
\usepackage{hyperref}
\usepackage{verbatim}

\usepackage{color, soul}

\usepackage{pgfplots}
\pgfplotsset{compat=1.15}
\usepackage{mathrsfs}

\usepackage{amsmath}
\usepackage{amsfonts}
\usepackage{amssymb}
\usepackage{tkz-tab}
\author{Destiné aux élèves de la 1er S2\\Lycée de Dindéfelo\\Présenté par M. BA}
\title{\textbf{Equations-Inéquations-Systèmes}}
\date{\today}
\usepackage{tikz}
\usetikzlibrary{arrows, shapes.geometric, fit}

% Commande pour la couleur d'accentuation
\newcommand{\myul}[2][black]{\setulcolor{#1}\ul{#2}\setulcolor{black}}
\newcommand\tab[1][1cm]{\hspace*{#1}}

\usepackage[margin=2cm]{geometry}
\usepackage{eso-pic}         % Pour ajouter des éléments en arrière-plan

\usepackage{enumitem}
%---------------------------------------
% Définir un compteur pour les exemples
\newcounter{exemple}

% Définir la commande \exemple pour afficher un exemple numéroté
\newcommand{\exemple}{%
  \refstepcounter{exemple}% Incrémenter le compteur
  \textbf{\textcolor{orange}{Exemple \theexemple : }} \ignorespaces
}
%---------------------------------------
\newcounter{solution}

% Définir la commande \solutione pour afficher un solution numéroté
\newcommand{\solution}{%
  \refstepcounter{solution}% Incrémenter le compteur
  \textbf{\textcolor{orange}{Solution \thesolution : }} \ignorespaces
}
%---------------------------------------
\definecolor{myorange}{rgb}{1.0, 0.8, 0.0}

% Définir un compteur pour les exercices d'application
\newcounter{exerciceapp}

% Définir la commande pour afficher un exercice d'application numéroté
\newcommand{\exerciceapp}{%
  \refstepcounter{exerciceapp}%
  \textbf{\textcolor{myorange}{Exercice d'application \theexerciceapp :}} \ignorespaces
}
%--------------------------------------
% Définir un compteur pour les exercices d'application
\newcounter{correction}

% Définir la commande pour afficher un correction exercice d'application numéroté
\newcommand{\correction}{%
  \refstepcounter{correction}%
  \textbf{\textcolor{myorange}{Correction \thecorrection :}} \ignorespaces
}
%--------------------------------------
% Définir un compteur pour les remarque d'application
\newcounter{remarque}

%----------------------------------------
\definecolor{myorange1}{rgb}{1.0, 1.5, 0}
% Définir la commande pour afficher un remarque numéroté
\newcommand{\remarque}{%
  \refstepcounter{remarque}%
  \textbf{\textcolor{myorange1}{Remarque \theremarque :}} \ignorespaces
}
% Commande pour ajouter du texte en arrière-plan
\AddToShipoutPicture{
    \AtTextCenter{%
        \makebox[0pt]{\rotatebox{45}{\textcolor[gray]{0.9}{\fontsize{5cm}{5cm}\selectfont Pathé Gobel BA}}}
    }
}

\begin{document}

\maketitle
\newpage
\section*{\underline{\textbf{\textcolor{red}{I. Equations}}}}
\subsection*{\underline{\textbf{\textcolor{red}{1. Compléments sur les trinômes du \( 2^{nd} \)degré}}}}
\subsection*{\underline{\textbf{\textcolor{red}{2. Equations irrationnelles}}}}
\subsubsection*{\underline{\textbf{\textcolor{red}{a. Équations du type \( \sqrt{f(x)} = \sqrt{g(x)} \)}}}}

\( \sqrt{x^{2}+3x-1} = \sqrt{x+2} \) est une équation irrationnelle du type \( \sqrt{f(x)} = \sqrt{g(x)} \) avec :

\[ f(x) = x^{2}+3x-1 \quad \text{et} \quad g(x) = x+2. \]

Pour résoudre une équation du type \( \sqrt{f(x)} = \sqrt{g(x)} \), on peut utiliser deux méthodes :

\underline{\textbf{Méthode de Résolution}}

\begin{itemize}
    \item \textbf{1ère étape :} Vérification des conditions de validité.
\end{itemize}

On a les équivalences suivantes :

\[
\sqrt{ f(x)} = \sqrt{ g(x)} \Leftrightarrow \begin{cases}
f(x) \geq 0 \\
g(x) \geq 0 \\
f(x) = g(x)
\end{cases}
\]

**Explication :** Pour que les deux racines soient définies et égales, il faut que \( f(x) \) et \( g(x) \) soient toutes deux positives, d'où les conditions \( f(x) \geq 0 \) et \( g(x) \geq 0 \). Ensuite, il faut que les deux expressions soient égales, soit \( f(x) = g(x) \).

En simplifiant ces conditions, on obtient :

\[
\Leftrightarrow \begin{cases}
g(x) \geq 0 \\
f(x) = g(x)
\end{cases}
\]

**Explication :** Ici, on peut réduire les conditions en ne gardant que \( g(x) \geq 0 \), car l'égalité \( f(x) = g(x) \) implique automatiquement que \( f(x) \geq 0 \) lorsque \( g(x) \geq 0 \).

Finalement, cela se résume à :

\[
\Leftrightarrow \begin{cases}
f(x) \geq 0 \\
f(x) = g(x)
\end{cases}
\]

**Explication :** Cette équivalence montre que, pour que l'équation soit valide, il faut vérifier que \( f(x) \geq 0 \), tout en résolvant l'équation \( f(x) = g(x) \).

 \textbf{\exemple}
 
\( \sqrt{ 2x+1} = \sqrt{ x+3}  \)

\( \sqrt{ x^{2}+3x-1} = \sqrt{ x+2}  \)

 \textbf{\solution} 

\begin{itemize}
    \item[•] \textbf{Étape 1 : Conditions de validité \( D_{v} \)}
\end{itemize}

L'équation existe ssi \( x^2 + 3x - 1 \geq 0 \) et \( x + 2 \geq 0 \). 

 La deuxième condition nous donne \( x \geq -2 \).

\( D_{v} =[-2; +\infty[ \)

\begin{itemize}
    \item[•] \textbf{Étape 2 : Résolution de l'équation}
\end{itemize}

Élevons les deux membres de l'équation au carré :

\[
x^2 + 3x - 1 = x + 2
\]

Cela revient à résoudre l'équation :

\[
x^2 + 3x - 1 = x + 2 \quad \Leftrightarrow \quad x^2 + 2x - 3 = 0
\]


\[
 x = 1 \quad \text{ou} \quad x = -3
\]

\textbf{Conclusion :} La seule solution est \( x = 1 \).

\textbf{\exerciceapp}
Résoudre l'équation:

\( \sqrt{2x^{2}+4x-6} = 2x - 1 \)

\subsubsection*{\underline{\textbf{\textcolor{red}{b. Équations du type \( \sqrt{f(x)} = ax + b \)}}}}

\[\text{On a les équivalences suivantes :}\]

\[
\sqrt{f(x)} = g(x) \Leftrightarrow \begin{cases} 
f(x) \geq 0 \\
g(x) \geq 0 \\
f(x) = g(x)^{2} 
\end{cases}
\]

**Explication :**
  
- \( f(x) \geq 0 \) : Pour que la racine carrée \( \sqrt{f(x)} \) soit définie, il est nécessaire que l'expression sous la racine soit positive ou nulle, d'où \( f(x) \geq 0 \).

- \( g(x) \geq 0 \) : Comme la racine carrée d'un nombre est toujours positive, il faut également que \( g(x) \), qui représente l'autre membre de l'équation, soit positif ou nul.

- \( f(x) = g(x)^2 \) : Une fois ces conditions vérifiées, on élève les deux membres de l'équation au carré afin d'éliminer la racine carrée, ce qui donne \( f(x) = g(x)^2 \).

En simplifiant, on obtient :

\[
\Leftrightarrow \begin{cases}
g(x) \geq 0 \\
f(x) = g(x)^{2}
\end{cases}
\]

**Explication :**
  
Dans certains cas, la condition \( f(x) \geq 0 \) devient redondante, car l'égalité \( f(x) = g(x)^2 \) implique que \( f(x) \) est nécessairement positive si \( g(x) \geq 0 \).

\textbf{\exemple}

Résolvons l'équation \( \sqrt{2x+1} = \sqrt{x+3} \).

\textbf{\solution}

\begin{itemize}
    \item \textbf{Étape 1 : Conditions de validité \( D_v \)}
\end{itemize}

L'équation est définie si \( 2x + 1 \geq 0 \) et \( x + 3 \geq 0 \).

- \( 2x + 1 \geq 0 \) donne \( x \geq -\frac{1}{2} \).

- \( x + 3 \geq 0 \) donne \( x \geq -3 \).

Le domaine de validité de l'équation est donc \( D_v = [-\frac{1}{2}; +\infty[ \).

\begin{itemize}
    \item \textbf{Étape 2 : Résolution de l'équation}
\end{itemize}

Élevons les deux membres de l'équation au carré :

\[
2x + 1 = x + 3
\]

Cela revient à résoudre l'équation :

\[
2x + 1 = x + 3 \quad \Leftrightarrow \quad x = 2.
\]

\begin{itemize}
    \item \textbf{Étape 3 : Vérification des solutions}
\end{itemize}

La solution \( x = 2 \) est dans le domaine de validité \( D_v = [-\frac{1}{2}; +\infty[ \).

\textbf{Conclusion :} La solution est \( x = 2 \).

\textbf{\exemple}

Résolvons l'équation \( \sqrt{x^{2}+3x-1} = \sqrt{x+2} \).

\textbf{\solution}

\begin{itemize}
    \item \textbf{Étape 1 : Conditions de validité \( D_v \)}
\end{itemize}

L'équation est définie si \( x^{2} + 3x - 1 \geq 0 \) et \( x + 2 \geq 0 \).

- \( x + 2 \geq 0 \) donne \( x \geq -2 \).

- Pour \( x^{2} + 3x - 1 \geq 0 \), on résout l'inéquation en trouvant les racines du polynôme : \( x = -3.41 \) et \( x = 0.41 \).

Le domaine de validité est donc \( D_v = [-2; +\infty[ \).

\begin{itemize}
    \item \textbf{Étape 2 : Résolution de l'équation}
\end{itemize}

Élevons les deux membres de l'équation au carré :

\[
x^{2} + 3x - 1 = x + 2
\]

Cela revient à résoudre l'équation :

\[
x^{2} + 3x - 1 = x + 2 \quad \Leftrightarrow \quad x^{2} + 2x - 3 = 0.
\]

Les solutions de cette équation sont \( x = 1 \) et \( x = -3 \).

\begin{itemize}
    \item \textbf{Étape 3 : Vérification des solutions}
\end{itemize}

- \( x = 1 \) est dans le domaine \( D_v = [-2; +\infty[ \), donc c'est une solution valide.

- \( x = -3 \) n'appartient pas au domaine de validité, donc cette solution est exclue.

\textbf{Conclusion :} La seule solution est \( x = 1 \).

\textbf{\exerciceapp}

Résoudre l'équation:

\section*{\underline{\textbf{\textcolor{red}{II. Inéquations irrationnelles}}}}
\subsection*{\underline{\textbf{\textcolor{red}{1. Inéquations du type \( \sqrt{f(x)} \leq ax+b \) }}}}

\[\text{On a les équivalences suivantes :}\]

\[
\sqrt{f(x)} \leq g(x) \Leftrightarrow \begin{cases} 
f(x) \geq 0 \\
g(x) \geq 0 \\
f(x) \leq g(x)^{2}
\end{cases}
\]

**Explication :**  

- **\( f(x) \geq 0 \)** : L'expression sous la racine carrée doit être positive ou nulle pour que la racine soit définie.

- **\( g(x) \geq 0 \)** : Comme la racine carrée est toujours positive ou nulle, il est nécessaire que l'expression \( g(x) \), ici \( ax + b \), soit aussi positive ou nulle.

- **\( f(x) \leq g(x)^2 \)** : On élève les deux membres de l'inéquation au carré pour éliminer la racine, ce qui donne l'inéquation \( f(x) \leq g(x)^2 \).

En simplifiant, on obtient :

\[
\Leftrightarrow \begin{cases}
g(x) \geq 0 \\
f(x) \leq g(x)^{2}
\end{cases}
\]

**Explication :**  

Il est souvent possible de ne vérifier que \( g(x) \geq 0 \), car si \( f(x) \leq g(x)^2 \), cela garantit que \( f(x) \geq 0 \) quand \( g(x) \geq 0 \).

\textcolor{green}{\textbf{Exemple 1}}

Résolvons l'inéquation \( \sqrt{2x+1} \leq x+3 \).

\begin{itemize}
    \item \textbf{Étape 1 : Conditions de validité \( D_v \)}
\end{itemize}

L'inéquation est définie si \( 2x + 1 \geq 0 \) et \( x + 3 \geq 0 \).

- \( 2x + 1 \geq 0 \) donne \( x \geq -\frac{1}{2} \).

- \( x + 3 \geq 0 \) donne \( x \geq -3 \).

Le domaine de validité de l'inéquation est donc \( D_v = [-\frac{1}{2}; +\infty[ \).

\begin{itemize}
    \item \textbf{Étape 2 : Résolution de l'inéquation}
\end{itemize}

On élève les deux membres de l'inéquation au carré :

\[
2x + 1 \leq (x + 3)^2
\]

Cela revient à résoudre l'inéquation :

\[
2x + 1 \leq (x + 3)^2 \quad \Leftrightarrow \quad 2x + 1 \leq x^2 + 6x + 9.
\]

En simplifiant, on obtient :

\[
0 \leq x^2 + 4x + 8.
\]

Cette inéquation est toujours vraie pour tout \( x \in \mathbb{R} \), donc elle est vérifiée pour tout \( x \geq -\frac{1}{2} \).

\textbf{Conclusion :} L'ensemble des solutions est \( S = [-\frac{1}{2}; +\infty[ \).

\textbf{\exemple}

Résolvons l'inéquation \( \sqrt{x^{2}+3x-1} \leq x+2 \).

\textbf{\solution}

\begin{itemize}
    \item \textbf{Étape 1 : Conditions de validité \( D_v \)}
\end{itemize}

L'inéquation est définie si \( x^{2} + 3x - 1 \geq 0 \) et \( x + 2 \geq 0 \).

- \( x + 2 \geq 0 \) donne \( x \geq -2 \).

- Pour \( x^{2} + 3x - 1 \geq 0 \), on trouve les racines du polynôme : \( x = -3.41 \) et \( x = 0.41 \).

Le domaine de validité est donc \( D_v = [-2; +\infty[ \).

\begin{itemize}
    \item \textbf{Étape 2 : Résolution de l'inéquation}
\end{itemize}

On élève les deux membres de l'inéquation au carré :

\[
x^{2} + 3x - 1 \leq (x + 2)^{2}
\]

Cela revient à résoudre l'inéquation :

\[
x^{2} + 3x - 1 \leq (x + 2)^{2} \quad \Leftrightarrow \quad x^{2} + 3x - 1 \leq x^{2} + 4x + 4.
\]

En simplifiant, on obtient :

\[
0 \leq x + 5.
\]

Cette inéquation est vérifiée pour tout \( x \geq -5 \).

\textbf{Conclusion :} L'ensemble des solutions est \( S = [-2; +\infty[ \), car il faut tenir compte des conditions initiales.

\subsection*{\underline{\textbf{\textcolor{red}{2. Inéquations du type \( \sqrt{f(x)} \geq ax+b \) }}}}

\[\text{On a les équivalences suivantes :}\]

\[
\sqrt{f(x)} \geq g(x) \Leftrightarrow \begin{cases} 
f(x) \geq 0 \\
g(x) \geq 0 \\
f(x) \geq g(x)^{2}
\end{cases}
\]

**Explication :**  

- **\( f(x) \geq 0 \)** : Comme la racine carrée \( \sqrt{f(x)} \) doit être définie, il est nécessaire que l'expression sous la racine soit positive ou nulle.

- **\( g(x) \geq 0 \)** : La racine carrée étant toujours positive ou nulle, il faut également que \( g(x) = ax + b \) soit positive ou nulle.

- **\( f(x) \geq g(x)^2 \)** : En élevant les deux membres de l'inéquation au carré, on obtient \( f(x) \geq g(x)^2 \).

En simplifiant, on obtient :

\[
\Leftrightarrow \begin{cases}
g(x) \geq 0 \\
f(x) \geq g(x)^{2}
\end{cases}
\]

**Explication :**  

La condition \( f(x) \geq 0 \) devient souvent implicite, car si \( f(x) \geq g(x)^2 \) et que \( g(x) \geq 0 \), alors \( f(x) \) est automatiquement positif ou nul.

\textbf{\exemple}

Résolvons l'inéquation \( \sqrt{2x+1} \geq x+3 \).

\textbf{\solution}

\begin{itemize}
    \item \textbf{Étape 1 : Conditions de validité \( D_v \)}
\end{itemize}

L'inéquation est définie si \( 2x + 1 \geq 0 \) et \( x + 3 \geq 0 \).

- \( 2x + 1 \geq 0 \) donne \( x \geq -\frac{1}{2} \).

- \( x + 3 \geq 0 \) donne \( x \geq -3 \).

Le domaine de validité de l'inéquation est donc \( D_v = [-\frac{1}{2}; +\infty[ \).

\begin{itemize}
    \item \textbf{Étape 2 : Résolution de l'inéquation}
\end{itemize}

On élève les deux membres de l'inéquation au carré :

\[
2x + 1 \geq (x + 3)^2
\]

Cela revient à résoudre l'inéquation :

\[
2x + 1 \geq (x + 3)^2 \quad \Leftrightarrow \quad 2x + 1 \geq x^2 + 6x + 9.
\]

En simplifiant, on obtient :

\[
0 \geq x^2 + 4x + 8,
\]

qui n'admet pas de solution réelle, car un polynôme de la forme \( x^2 + 4x + 8 \) est toujours strictement positif.

\textbf{Conclusion :} Il n'y a pas de solution pour cette inéquation.

\textbf{\exemple}

Résolvons l'inéquation \( \sqrt{x^{2}+3x-1} \geq x+2 \).

\textbf{\solution}

\begin{itemize}
    \item \textbf{Étape 1 : Conditions de validité \( D_v \)}
\end{itemize}

L'inéquation est définie si \( x^{2} + 3x - 1 \geq 0 \) et \( x + 2 \geq 0 \).

- \( x + 2 \geq 0 \) donne \( x \geq -2 \).

- Pour \( x^{2} + 3x - 1 \geq 0 \), on trouve les racines du polynôme : \( x = -3.41 \) et \( x = 0.41 \).

Le domaine de validité est donc \( D_v = [-2; +\infty[ \).

\begin{itemize}
    \item \textbf{Étape 2 : Résolution de l'inéquation}
\end{itemize}

On élève les deux membres de l'inéquation au carré :

\[
x^{2} + 3x - 1 \geq (x + 2)^{2}
\]

Cela revient à résoudre l'inéquation :

\[
x^{2} + 3x - 1 \geq (x + 2)^{2} \quad \Leftrightarrow \quad x^{2} + 3x - 1 \geq x^{2} + 4x + 4.
\]

En simplifiant, on obtient :

\[
0 \geq x + 5.
\]

Cette inéquation est vérifiée pour \( x \leq -5 \), mais comme le domaine de validité est \( D_v = [-2; +\infty[ \), il n'y a pas de solution réelle à cette inéquation.

\textbf{Conclusion :} L'inéquation n'admet pas de solution réelle sur le domaine donné.

\subsection*{\underline{\textbf{\textcolor{red}{3. Inéquations du type \( \sqrt{f(x)} \leq  \sqrt{g(x)} \) }}}}

\section*{\underline{\textbf{\textcolor{red}{III. Systèmes}}}}
\subsection*{\underline{\textbf{\textcolor{red}{1. Systèmes de 3 équations linéaires à trois inconnues}}}}
\subsubsection*{\underline{\textbf{\textcolor{red}{a. Exemple}}}}
\subsubsection*{\underline{\textbf{\textcolor{red}{b. Méthode du pivot de Gauss }}}}
\subsection*{\underline{\textbf{\textcolor{red}{2. Systèmes d’inéquations linéaires }}}}
\subsubsection*{\underline{\textbf{\textcolor{red}{a. Inéquations linéaires à deux inconnues}}}}
\subsubsection*{\underline{\textbf{\textcolor{red}{b. Système de deux inéquations linéaires à deux inconnues }}}}
\end{document}
