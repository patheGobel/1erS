\documentclass[12pt]{article}
\usepackage{stmaryrd}
\usepackage{graphicx}
\usepackage[utf8]{inputenc}
\usepackage[french]{babel}
\usepackage[T1]{fontenc}
\usepackage{hyperref}
\usepackage{verbatim}
\usepackage{color,soul}
\usepackage{amsmath}
\usepackage[table]{xcolor}
\usepackage{amsfonts}
\usepackage{amssymb}
\usepackage{systeme}
\usepackage{tkz-tab}
\author{Destiné à la 1èrS2\\Au Lycée de Dindéfelo}
\title{\textbf{Dénombrement}}
\date{\today}
\usepackage{tikz}
\usetikzlibrary{arrows}
\usepackage[a4paper,left=20mm,right=20mm,top=15mm,bottom=15mm]{geometry}
\usepackage{mathtools}
\usepackage{systeme}

\usepackage{pgfplots}
\pgfplotsset{compat=1.15}
\usepackage{mathrsfs}

\usetikzlibrary{arrows}
\pagestyle{empty}

\DecimalMathComma

\begin{document}

\maketitle
\newpage
\section*{\underline{\textbf{\textcolor{red}{I. Equations}}}}
\subsection*{\underline{\textbf{\textcolor{red}{1. Compléments sur les trinômes du \( 2^{nd} \)degré}}}}
\subsection*{\underline{\textbf{\textcolor{red}{2. Equations irrationnelles}}}}
\subsubsection*{\underline{\textbf{\textcolor{red}{a. Equations du type \( \sqrt{f(x)} = \sqrt{g(x)} \) }}}}
\( \sqrt{x^{2}+3x-1} = \sqrt{x+2} \) est une équation irrationnelle du type \( \sqrt{f(x)} = \sqrt{g(x)} \) avec 

\( f(x) = x^{2}+3x-1 \) et \( g(x) = x+2 \).

Pour résoudre une équation du type \( \sqrt{f(x)} = \sqrt{g(x)} \) on peut utiliser deux méthodes :

\begin{itemize}
\item[•]\textbf{Méthode par implication}

\begin{itemize}
 \item \textbf{1ère étape :}
 \[ 
 \text{On écrit } \sqrt{f(x)} = \sqrt{g(x)} \implies f(x) = g(x) \text{ puis on résout l’équation }f(x) = g(x)
 \] 
 \[ \text{ Les solutions éventuelles de } f(x) = g(x) \text{ sont les potentiels candidats } \]
 \[ \text{pour être solution de l’équation} \sqrt{f(x)} = \sqrt{g(x)}\]
 \item \textbf{2ème étape :}
 \[ 
 \text{Parmi les éventuelles solutions de l’équation } f(x) = g(x) \implies f(x) = g(x),\] 
 \[ \text{ on cherche celles qui vérifient }\sqrt{f(x)} = \sqrt{g(x)} \] 
 \[ \text{ Ce sont ces réels qui sont les solutions de l’équation } \sqrt{f(x)} = \sqrt{g(x)} \]

\end{itemize}
\textbf{Exemple} 
\textbf{Exercice d’application}
\item[•]\textbf{Méthode par équivalence}

\begin{itemize}
 \item \textbf{1ère étape :}
	\[ \text{On écrit :} \sqrt{ f(x)} = \sqrt{ g(x)} \Leftrightarrow \begin{cases} f(x) \geq 0 \\ f(x) = g(x) \end{cases} \text{ou bien } \sqrt{ f(x)} = \sqrt{ g(x)} \Leftrightarrow \begin{cases} g(x) \geq 0 \\ f(x) = g(x) \end{cases}\]
 \item \textbf{2ème étape :}
  \[ \text{On résout le système }\begin{cases}f(x) \geq 0 \\ f(x) = g(x) \end{cases} \text{ou bien le système} \begin{cases}f(x) \geq 0 \\ f(x) = g(x) \end{cases} \] 
 \[ \text{L’ensemble des solutions de l’un de ces systèmes est l’ensemble des solutions de l’équation } \]
 \[ \sqrt{ f(x)} = \sqrt{ g(x)} \]
 \textbf{Exemple} 
 
	Résoudre dans \( \mathbb{R} \) l’équation suivante: \( \sqrt{f(x)} = \sqrt{g(x)} \)
	
	\textbf{Exercice d’application}
	
Résoudre dans \( \mathbb{R} \) l’équation suivante: \( \sqrt{f(x)} = \sqrt{g(x)} \)
\end{itemize}

\end{itemize}

\subsubsection*{\underline{\textbf{\textcolor{red}{b. Equations du type \( \sqrt{f(x)} = ax+b \) }}}}

\section*{\underline{\textbf{\textcolor{red}{II. Inéquations irrationnelles}}}}
\subsection*{\underline{\textbf{\textcolor{red}{1. Inéquations du type \( \sqrt{f(x)} \leq ax+b \) }}}}
\subsection*{\underline{\textbf{\textcolor{red}{2. Inéquations du type \( \sqrt{f(x)} \geq ax+b \) }}}}
\subsection*{\underline{\textbf{\textcolor{red}{3. Inéquations du type \( \sqrt{f(x)} \leq  \sqrt{g(x)} \) }}}}

\section*{\underline{\textbf{\textcolor{red}{III. Systèmes}}}}
\subsection*{\underline{\textbf{\textcolor{red}{1. Systèmes de 3 équations linéaires à trois inconnues}}}}
\subsubsection*{\underline{\textbf{\textcolor{red}{a. Exemple}}}}
\subsubsection*{\underline{\textbf{\textcolor{red}{b. Méthode du pivot de Gauss }}}}
\subsection*{\underline{\textbf{\textcolor{red}{2. Systèmes d’inéquations linéaires }}}}
\subsubsection*{\underline{\textbf{\textcolor{red}{a. Inéquations linéaires à deux inconnues}}}}
\subsubsection*{\underline{\textbf{\textcolor{red}{b. Système de deux inéquations linéaires à deux inconnues }}}}
\end{document}