\documentclass[12pt]{article}
\usepackage{stmaryrd}
\usepackage{graphicx}
\usepackage[utf8]{inputenc}

\usepackage[french]{babel}
\usepackage[T1]{fontenc}
\usepackage{hyperref}
\usepackage{verbatim}

\usepackage{color, soul}

\usepackage{pgfplots}
\pgfplotsset{compat=1.15}
\usepackage{mathrsfs}

\usepackage{amsmath}
\usepackage{amsfonts}
\usepackage{amssymb}
\usepackage{tkz-tab}
\author{Destinés à la 1\textsuperscript{ère}S2\\Au Lycée de Dindéferlo}
\title{\textbf{Limites-Et-continuité}}
\date{\today}
\usepackage{tikz}
\usetikzlibrary{arrows, shapes.geometric, fit}

% Commande pour la couleur d'accentuation
\newcommand{\myul}[2][black]{\setulcolor{#1}\ul{#2}\setulcolor{black}}
\newcommand\tab[1][1cm]{\hspace*{#1}}

\begin{document}
\maketitle
\newpage

\section*{\underline{\textbf{\textcolor{red}{I.Limites}}}}
\subsection*{\underline{\textbf{\textcolor{red}{1.Limites infinies à l'infinie}}}}
\definecolor{qqwuqq}{rgb}{0,0.39215686274509803,0}
\begin{tikzpicture}[line cap=round,line join=round,>=triangle 45,x=1cm,y=1cm]
\begin{axis}[
  x=1cm, y=1cm,
  axis lines=middle,
  ymajorgrids=true,
  xmajorgrids=true,
  xmin=-7, xmax=7,
  ymin=-8, ymax=8,
  xtick={-2,-15,...,30},
  ytick={-18,-16,...,12},
]

\draw[line width=2pt,color=qqwuqq,smooth,samples=100,domain=-8:3] plot(\x,{(\x)^(2)});

\begin{scriptsize}
  \draw[color=qqwuqq] (-6,2) node {$f$};
\end{scriptsize}

\end{axis}
\end{tikzpicture}
\subsection*{\underline{\textbf{\textcolor{red}{2. Limites finie à l'infinie}}}}
\subsection*{\underline{\textbf{\textcolor{red}{3. Limites finie en $x_{0}$}}}}
\subsection*{\underline{\textbf{\textcolor{red}{4. Limites infinie en $x_{0}$}}}}
\subsection*{\underline{\textbf{\textcolor{red}{5. Limites à gauche et à droite}}}}
\subsection*{\underline{\textbf{\textcolor{red}{Remarque}}}}
\subsection*{\underline{\textbf{\textcolor{red}{II.Calcule de limite}}}}

\subsection*{\underline{\textbf{\textcolor{red}{1.limite en $x_{0}$ d'une fonction écrite avec une seul expression}}}}
\subsection*{\underline{\textbf{\textcolor{red}{2.Limite à l'infinie de $x^{n}$ n $\in\mathbb{N}$}}}}
\subsection*{\underline{\textbf{\textcolor{red}{3.Opération sur les limites}}}}
\subsection*{\underline{\textbf{\textcolor{red}{a.Limite d'une somme}}}}
\subsection*{\underline{\textbf{\textcolor{red}{b.Limite d'un produit}}}}
\subsection*{\underline{\textbf{\textcolor{red}{c.Limite d'un quotient}}}}
\subsection*{\underline{\textbf{\textcolor{red}{d.Technique de calcul de limite}}}}
\subsection*{\underline{\textbf{\textcolor{red}{4.Limites à l'infinie d'une fonction polynôme}}}}
La limite à l'infini d'une fonction polynomiale est égale à la limite à l'infini de son terme(monôme) de plus haut degré.
\subsection*{\underline{\textbf{\textcolor{red}{5.Limites à l'infinie d'un quotient}}}}
La limite à l'infini d'une fonction rationnelle est égale à la limite du quotient des termes de plus haut degré.
\subsection*{\underline{\textbf{\textcolor{red}{6.théorèmes}}}}
\subsection*{\underline{\textbf{\textcolor{red}{a.Théorème de comparaison}}}}
Soit f tel que $f:I\longrightarrow \mathbb{R}$ ; $x_{0} \in I$ où $x_{0}=-+\infty$\\
Si au voisinage de $x_{0}$, $f(x)\leq g(x)$ On a:\\
$\bullet$ Si  $\lim_{x \to x_{0}}f(x)=+\infty$ alors $\lim_{x \to x_{0}}g(x)=+\infty$\\
$\bullet$ Si  $\lim_{x \to x_{0}}f(x)=-\infty$ alors $\lim_{x \to x_{0}}g(x)=-\infty$
\subsection*{\underline{\textbf{\textcolor{red}{b.Théorème des gendarmes}}}}
Si aux voisinage de $x_{0}$ on a $g(x) \leq f(x) \leq h(x)$ et que 
$\lim_{x \to x_{0}}f(x)=\lim_{x \to x_{0}}h(x)=l$(finie ou non) alors $\lim_{x \to x_{0}}f(x)=l$
\subsection*{\underline{\textbf{\textcolor{blue}{Exemple}}}}
a)\[\lim_{x \to 0} x^{2} sin(\frac{1}{x})\]\\
b)\[\lim_{x \to +\infty} \frac{sinx + x^{4}}{x^{2}+1}\]
\subsection*{\underline{\textbf{\textcolor{red}{c.Corollaire}}}}
Si au voisinage de $x_{0}$ $|f(x)-l| \leq g(x)$ et $\lim_{x \to x_{0}}g(x)=0$ alors \\
$\lim_{x \to x_{0}}f(x)=l$
\subsection*{\underline{\textbf{\textcolor{red}{7.Limite en zéro d'une fonction trigonométrique}}}}
\[\lim_{x \to 0} \frac{sinx}{x} = 1\]
\[\lim_{x \to 0} \frac{sinax}{ax} = 1\] avec $a\in\mathbb{R}$
\[\lim_{x \to 0} \frac{tanx}{x} = 1\]
\[\lim_{x \to 0} \frac{cosx-1}{x} = 0\]
\[\lim_{x \to 0} \frac{cosx-1}{x^{2}} = -\frac{1}{2}\]
\[\lim_{x \to 0} \frac{tanax}{ax} = 1\]
\subsection*{\underline{\textbf{\textcolor{red}{Application}}}}
Calculer les limites suivantes\\
a)\[\lim_{x \to x_{0}} \frac{sin5x}{sin3x}\]
b)\[\lim_{x \to x_{0}} \frac{\sqrt{1-cosx}}{tan2x}\]
c)\[\lim_{x \to x_{0}} \frac{2sin^{2}x-2(1-cosx)}{5x^{2}}\]
d)\[\lim_{x \to x_{0}} \frac{sinx}{\sqrt{x}} = 0\]
\subsection*{\underline{\textbf{\textcolor{red}{Solution}}}}
\subsection*{\underline{\textbf{\textcolor{red}{III.Continuité}}}}
\subsection*{\underline{\textbf{\textcolor{red}{1.Définition}}}}
Soit f une fonction definie en $x_{0}$ on dit que f est continue en $x_{0}$ si
$\lim_{x \to x_{0}} f(x)=f(x_{0})$.
\subsection*{\underline{\textbf{\textcolor{red}{Exemple}}}}
Etudions la continuité de f en $x_{0}$ dans chacun des cas.\\
a)$f(x)=2x-1$ ; en $x_{0}=3$\\
b)\[ f(x) = \begin{cases} 
  \frac{x^2-1}{x-1}, & \text{si } x \neq 1 \\
  f(1)=2, & \text{si } x = 1 
\end{cases} \]
en $x_{0}=1$\\
c)\[ f(x) = \begin{cases} 
  \frac{sinx}{x}, & \text{si } x \neq 0 \\
  -1, & \text{si } x=0 
\end{cases} \]
en $x_{0}=0$
\subsection*{\underline{\textbf{\textcolor{red}{Solution}}}}
a)A-t-on $\lim_{x \to 3}f(x)=f(3)$?\\
$\lim_{x \to 3}f(x)=\lim_{x \to 3}2x-1=5$ or $f(3)=5$\\
Ainsi, on a $\lim_{x \to 3}f(x)=f(3)=5$\\
Donc f est continue en 3.\\
b)A-t-on $\lim_{x \to 1}f(x)=f(1)$?\\
$D_{f}=\mathbb{R}$\\
$\lim_{x \to 1}f(x)=\lim_{x \to 1}\frac{x^2-1}{x-1}=\lim_{x \to 1}x+1=2$ or f(1)=2\\
Ainsi, on a $\lim_{x \to 1}f(x)=f(1)=2$. Donc f est une fonction continue en 1.
c)A-t-on $\lim_{x \to 0}f(x)=f(0)$?\\
$\lim_{x \to 0}f(x)=\lim_{x \to 0}\frac{sinx}{x}=1$ or f(0)=-1\\
Ainsi, on a $\lim_{x \to 0}f(x) \neq f(0)$. Donc f n'est pas conitnue en 0
\subsection*{\underline{\textbf{\textcolor{red}{2.Continuité à gauche, continuité à droite}}}}
Soit f une fonction definie en $x_{0}$\\ 
f est continue à \textcolor{blue}{gauche} de $x_{0} \Leftrightarrow \lim_{x \to x_{0}^{-}}f(x)=f(x_{0})$\\
Soit f une fonction definie en $x_{0}$\\ 
f est continue à \textcolor{blue}{droite} de $x_{0} \Leftrightarrow \lim_{x \to x_{0}^{+}}f(x)=f(x_{0})$\\
f est continue en $x_{0} \Leftrightarrow \lim_{x \to x_{0}^{-}}f(x)=\lim_{x \to x_{0}^{+}}f(x)=f(x_{0})$
\subsection*{\underline{\textbf{\textcolor{red}{Exemple}}}}
Etudions la continuite de f en $x_{0}$\\
\[ f(x) = \begin{cases} 
  \frac{x^2-4}{x+2}, & \text{si } x > 2 \\
  3x-6, & \text{si } x\leq 2
\end{cases} \]
en $x_{0}=2$
\subsection*{\underline{\textbf{\textcolor{red}{Solution}}}}
\subsection*{\underline{\textbf{\textcolor{red}{3.Continuité sur un intervalle}}}}
Une fonction f est continue sur un intervalle I si elle est continue en tout point de I.\\
$\bullet$ Tout fonction polynôme est continue sur $\mathbb{R}$.\\
$\bullet$ Les fonctions rationnelle et irrationnelle sont continues sur leur ensemble de définition.\\
$\bullet$ La somme et le produit de deux fonctions continues sur un intervalle I sont continue sur cet intervalle I.\\
$\bullet$  Si f et g sont deux fonctions continues sur I alors $f \times g$ est continue sur I de meme $\alpha \times f$ est continue où $\alpha \in \mathbb{R}$.\\
$\bullet$ Si f et g sont deux fonctions continue sur I alors f+g est cotinue et si $\forall x \in I$ 
g(x)$\neq 0$  alors $\frac{f(x)}{g(x)}$ est continue sur I\\

\subsection*{\underline{\textbf{\textcolor{red}{Exemple}}}}
Etudions la continuite de f sur son $D_{f}$\\
\[ f(x) = \begin{cases} 
  \frac{x^2-1}{x-1}, & \text{si } x < 1 \\
   2x,\quad\quad\quad\quad\quad \text{si } 1 \leq x \leq 2\\
  \sqrt{x+2}, & \text{si } > 2
\end{cases} \]
\subsection*{\underline{\textbf{\textcolor{red}{4.Prolongement par continuité en u point $x_{0}$}}}}
Soit f une fonction non definie en $x_{0}$ ($x_{0} \in \mathbb{R}$) tel que $\lim_{x \to x_{0}}f(x)=l \in \mathbb{R}$\\
La fonction Definie par 
\[ g(x) = \begin{cases} 
  f(x), & \text{si } x \neq x_{0} \\
  l, & \text{si } x= x_{0}
\end{cases} \]
g est appelée prolongement par conitinuite de f en $ x_{0} $
\subsection*{\underline{\textbf{\textcolor{red}{Exemple}}}}
Soit $f(x)=\frac{x^{2}-1}{x-1}$.\\
f est-elle prolongeable par continuité en $ x_{0}=1$
\subsection*{\underline{\textbf{\textcolor{red}{Solution}}}}
$D_{f}=\mathbb{R}\setminus\{1\}$\\
On a $1 \notin D_{f}$ et $\lim_{x \to 1}\frac{x^{2}-1}{x-1}=\lim_{x \to 1} x+1=2$\\
Donc f est prologeable par continuité en $x_{0}=1$ et son prolongement par continuité en $x_{0}=1$ est 
\[ g(x) = \begin{cases} 
  f(x), & \text{si } x \neq 1 \\
  2, & \text{si } x= 1
\end{cases} \]
\subsection*{\underline{\textbf{\textcolor{red}{}}}}
\end{document}
