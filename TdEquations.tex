\documentclass[12pt]{article}
\usepackage{lmodern} % Pour une police plus nette
\usepackage{stmaryrd}
\usepackage{graphicx} % Pour l'insertion d'images
\usepackage{float}    % Pour contrôler précisément le placement
\usepackage[utf8]{inputenc}
\usepackage[french]{babel}
\usepackage[T1]{fontenc}
\usepackage{hyperref}
\usepackage{verbatim}
\usepackage{color, soul}
\usepackage{pgfplots}
\pgfplotsset{compat=1.18} % Version plus récente de pgfplots
\usepackage{mathrsfs}
\usepackage{amsmath}
\usepackage{amsfonts}
\usepackage{amssymb}
\usepackage{tkz-tab}
%\author{Destiné aux élèves de Terminale S\\Lycée de Dindéfelo\\Présenté par M. BA}
%\title{\textbf{Rappels et compléments sur les fonctions numériques}}
%\date{\today}
\usepackage{tikz}
\usetikzlibrary{arrows, shapes.geometric, fit}
% Commande pour la couleur d'accentuation
\newcommand{\myul}[2][black]{\setulcolor{#1}\ul{#2}\setulcolor{black}}
\newcommand\tab[1][1cm]{\hspace*{#1}}
\usepackage[margin=2.5cm]{geometry} % Ajustement des marges
\usepackage{eso-pic} % Pour ajouter des éléments en arrière-plan

% Commande pour ajouter du texte en arrière-plan, centré au milieu de chaque page
\AddToShipoutPicture{
    \AtPageCenter{%
        \makebox(0,0)[c]{\rotatebox{60}{\textcolor[gray]{0.9}{\fontsize{2cm}{2cm}\selectfont Pathé Gobel BA}}}
    }
}

\begin{document}

\noindent
\begin{minipage}[t]{0.48\textwidth}
\raggedright
\textbf{Ministère de l'Éducation Nationale}\\
Inspection Académique de Kédougou\\
Lycée Dindéfelo\\
Cellule de Mathématiques
\end{minipage}
\hfill
\begin{minipage}[t]{0.48\textwidth}
\raggedleft
\textbf{Année scolaire 2024-2025}\\
Date : 03/10/2024\\
Classe : 1er S2\\
Professeur : M. BA
\end{minipage}

\vspace{1cm}
\section*{Exercice 1}
\begin{enumerate}
\item Résoudre dans \( \mathbb{R} \) les équations et inéquations suivantes :

\[ \text{a)} \sqrt{2x-1}-\sqrt{x-5} = 0    \quad\quad;\quad\quad   \text{b)} \sqrt{4x^{2}+x+5}=\sqrt{2x+1} \]

\[ \text{c)} \sqrt{-x^{2}+x+1} \leq x-2 = 0    \quad\quad;\quad\quad   \text{d)} \sqrt{x^{2}+6x+6} \geq 2x+1 \]

\end{enumerate}
\section*{Exercice 2}
\begin{enumerate}
\item Soit \( m \) un nombre réel. On considère l’équation d’inconnue \( x \) :

\( E \) \( (x+3)x^{2} +2mx + m + 5 = 0\)
\item[a)] Discuter suivant les valeurs de \( m \), l’existence, le nombre et le signe des racines de \( (E) \) 

Dans le cas où \( (E) \) admet deux racines distinctes \( x_{2} \) et \( x_{2} \) , peut-on déterminer \( m \) pour que:

\( ( 2x_{1} - 1 ) (2x_{2}-1) = 6 \) puis \( x_{1}^{2} + x_{2}^{2} = 2 \)

\item[b)]Trouver une relation indépendante de \( m \) entre\( x_{2} \) et \( x_{2} \)
\item[c)]Utiliser cette relation pour trouver les racines doubles de \( (E) \), lorsqu’elles existent
\end{enumerate}
\section*{Exercice 3}
\begin{enumerate}
\item Résoudre dans \( \mathbb{R}^{2} \) par le pivot de Gauss le système \( \begin{cases} 27x + 9y + 3z = -18\\ -x + y -z = 2\\ x + y + z = -8 \end{cases}\)
\end{enumerate}
\section*{Exercice 4}
\begin{enumerate}
\item Soit le polynôme de degré 3 tel que \( P(0) = 8 \) ; \( P(1) = 0 \) ; \( P(3) = -10 \) et \( P(-1) = 10 \).
\begin{enumerate}
\item Montrer que \( P(x) = x^{3} - 3x^{2} - 6x + 8 \).
\item Résoudre dans \( \mathbb{R} \) \( P(x) = 0 \).
\item Résoudre dans \( \mathbb{R} \) \( P(x) = \sqrt{x+3}-x = 0\).
\end{enumerate}
\end{enumerate}
\end{document}
