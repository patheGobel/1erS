\documentclass[12pt]{article}
\usepackage{lmodern} % Pour une police plus nette
\usepackage{stmaryrd}
\usepackage{graphicx} % Pour l'insertion d'images
\usepackage{float}    % Pour contrôler précisément le placement
\usepackage[utf8]{inputenc}
\usepackage[french]{babel}
\usepackage[T1]{fontenc}
\usepackage{hyperref}
\usepackage{verbatim}
\usepackage{color, soul}
\usepackage{pgfplots}
\pgfplotsset{compat=1.18} % Version plus récente de pgfplots
\usepackage{mathrsfs}
\usepackage{amsmath}
\usepackage{amsfonts}
\usepackage{amssymb}
\usepackage{tkz-tab}
\usepackage{enumitem}
%\author{Destiné aux élèves de Terminale S\\Lycée de Dindéfelo\\Présenté par M. BA}
%\title{\textbf{Rappels et compléments sur les fonctions numériques}}
%\date{\today}
\usepackage{tikz}
\usetikzlibrary{arrows, shapes.geometric, fit}
% Commande pour la couleur d'accentuation
\newcommand{\myul}[2][black]{\setulcolor{#1}\ul{#2}\setulcolor{black}}
\newcommand\tab[1][1cm]{\hspace*{#1}}
\usepackage[margin=2.5cm]{geometry} % Ajustement des marges
\usepackage{eso-pic} % Pour ajouter des éléments en arrière-plan

% Commande pour ajouter du texte en arrière-plan, centré au milieu de chaque page
\AddToShipoutPicture{
    \AtPageCenter{%
        \makebox(0,0)[c]{\rotatebox{60}{\textcolor[gray]{0.9}{\fontsize{2cm}{2cm}\selectfont PGB}}}
    }
}

\newcounter{exercice}

% Définir la commande \exemple pour afficher un exemple numéroté
\newcommand{\exercice}{%
  \refstepcounter{exercice}% Incrémenter le compteur
  \textbf{\textcolor{black}{Exercice \theexercice  }} \ignorespaces
}

\begin{document}

\noindent
\begin{minipage}[t]{0.48\textwidth}
\raggedright
\textbf{Ministère de l'Éducation Nationale}\\
Inspection Académique de Kédougou\\
Lycée Dindéfelo\\
Cellule de Mathématiques
\end{minipage}
\hfill
\begin{minipage}[t]{0.48\textwidth}
\raggedleft
\textbf{Année scolaire 2024-2025}\\
Date : 04/11/2024\\
Classe : 1er S2\\
Professeur : M. BA
\end{minipage}

\vspace{1cm}

\textbf{\underline{\exercice}:}

\begin{enumerate}
\item Résoudre dans \( \mathbb{R} \) les équations et inéquations suivantes :

\[ \text{a)} \sqrt{2x-1}-\sqrt{x-5} = 0    \quad\quad;\quad\quad   \text{b)} \sqrt{4x^{2}+x+5}=2x+1 \]

\[ \text{c)} \sqrt{-x^{2}+x+1} \leq x-2    \quad\quad;\quad\quad   \text{d)} \sqrt{x^{2}+6x+6} \geq 2x+1 \]

\[ \text{e)} \sqrt{2x^{2}-3x-2} = 2x-4    \quad\quad;\quad\quad   \text{f)} \sqrt{4-x} = \sqrt{x^{2}+2x-6} \]

\[ \text{g)} \sqrt{x^{2}+4x} < \sqrt{x^{2}-5x+4}    \quad\quad;\quad\quad   \text{h)} \sqrt{x^{2}-x-2} > x-1 \]

\end{enumerate}

\textbf{\underline{\exercice}:}

\begin{enumerate}
\item Soit \( m \) un nombre réel. On considère l’équation d’inconnue \( x \) :

\( (E) \) \( (m+3)x^{2} +2mx + m + 5 = 0\)
\item[a)] Discuter suivant les valeurs de \( m \), l’existence, le nombre et le signe des racines de \( (E) \) 

Dans le cas où \( (E) \) admet deux racines distinctes \( x_{2} \) et \( x_{2} \) , peut-on déterminer \( m \) pour que:

\( ( 2x_{1} - 1 ) (2x_{2}-1) = 6 \) puis \( x_{1}^{2} + x_{2}^{2} = 2 \)

\item[b)]Trouver une relation indépendante de \( m \) entre\( x_{2} \) et \( x_{2} \)
\item[c)]Utiliser cette relation pour trouver les racines doubles de \( (E) \), lorsqu’elles existent
\end{enumerate}

\textbf{\underline{\exercice}:}

\begin{enumerate}[label=\Roman*.]
    \item Soit l'équation $(E) : x^2 - (2m + 1)x + m^2 - 2 = 0$
    \begin{enumerate}[label=\arabic*.]
        \item Discuter valeurs de $m$, les solutions de l'équation $(E)$
        \item Dans le cas où $(E)$ admet deux solutions $x'$ et $x''$, déterminer $m$ tel que :
        \begin{enumerate}[label=\alph*.]
            \item $x' + x'' = 35$
            \item $|x' - x''| = 5$
        \end{enumerate}
    \end{enumerate}
    
    \item On considère l'équation $(E_m) : 3(-m + 2)x^2 + (3m + 9)x + 9m + 3 = 0$.
    \begin{enumerate}[label=\arabic*.]
        \item Déterminer $m$ pour que $-1$ soit racine de $(E_m)$.
        \item Résoudre $(E_m)$ pour $m = 2$.
        \item On suppose que $m \neq 2$.
        \begin{enumerate}[label=\alph*.]
            \item Montrer que le discriminant $\Delta = 9(13m^2 - 14m + 1)$
            \item Pour quelles valeurs de $m$ l'équation n'admet pas de solutions.
            \item Pour quelles valeurs de $m$ l'équation admet des solutions distinguées.
            \item Dans le cas où les racines existent.
            \begin{enumerate}[label=\roman*.]
                \item Trouver une relation indépendante de $m$ entre les racines.
                \item Trouver alors les racines doubles de $(E_m)$.
            \end{enumerate}
        \end{enumerate}
    \end{enumerate}
\end{enumerate}

\textbf{\underline{\exercice}:}

\begin{enumerate}
\item Résoudre dans \( \mathbb{R}^{3} \) par le pivot de Gauss le système \( \begin{cases} 27x + 9y + 3z = -18\\ -x + y -z = 2\\ x + y + z = -8 \end{cases}\)

\item Résoudre dans \( \mathbb{R}^{3} \) par le pivot de Gauss le système \( \begin{cases} 4x - 2y + z = 6\\ 8x + 6y -3z = 22\\ 2x + 4y + 7z = 44 \end{cases}\)
\item En déduire la résolution dans \( \mathbb{R}^{3} \) du système \( \begin{cases} \frac{4}{x+3} - \frac{2}{y-2} + \frac{1}{z} = 6\\ \frac{8}{x+3} + \frac{6}{y-2} -\frac{3}{z} = 22\\ \frac{2}{x+3} + \frac{4}{y-2} + \frac{7}{z} = 44 \end{cases}\)
\end{enumerate}

\textbf{\underline{\exercice}:}

\begin{enumerate}
\item Soit le polynôme de degré 3 tel que \( P(0) = 8 \) ; \( P(1) = 0 \) ; \( P(3) = -10 \) et \( P(-1) = 10 \).
\begin{enumerate}
\item Montrer que \( P(x) = x^{3} - 3x^{2} - 6x + 8 \).
\item Résoudre dans \( \mathbb{R} \) \( P(x) = 0 \).
\item Résoudre dans \( \mathbb{R} \) \( P(\sqrt{x+3}-x) = 0\).
\end{enumerate}
\end{enumerate}

\textbf{\underline{\exercice}:}

\begin{enumerate}
    \item Résoudre dans $ \mathbb{R} $, les équations suivantes :
    \begin{enumerate}
        \item $\sqrt{x^2 - 5x} = x$
        \item $\sqrt{x^2 - 5x + 6} \geq x - 2$
        \item $\sqrt{2x^2 - x} < 2x - 3$
    \end{enumerate}
    
    \item 
    \begin{enumerate}
        \item[a)] Résoudre dans $\mathbb{R}$, l'équation suivante : $2x^2 - 4x - 30 = 0$
        \item[b)] En déduire l’ensemble des solutions de l’équation : $2(x^2 - 1)^2 - 4|x^2 - 1| - 30 = 0$
    \end{enumerate}
    
    \item Résoudre le système suivant par la méthode du pivot de Gauss :
    \[
    \begin{cases}
        5x - 4y - z = 3 \\
        -x + 2y - 3z = 7 \\
        x - 3y + 2z = 5
    \end{cases}
    \]
\end{enumerate}

\textbf{\underline{\exercice}:}

Soit l'équation $(E_m) : (m - 2)x^2 + (2m + 2)x + 10m - 14 = 0 ; m \in \mathbb{R}$

\begin{enumerate}
    \item Discuter suivant les valeurs de $m$ le nombre de racines de $(E_m)$
    \item Pour quelle(s) valeur(s) de $m$, les solutions $x_1$ et $x_2$ de $(E_m)$ :
    \begin{enumerate}
        \item sont de signes contraires ?
        \item sont strictement positives ?
    \end{enumerate}
\end{enumerate}

\textbf{\underline{\exercice}:}

\begin{enumerate}
    \item Résoudre dans $\mathbb{R}$ les équations suivantes.
    \begin{enumerate}
        \item $\sqrt{-x^2 + 5x + 9} = \sqrt{x - 3}$
        \item $\sqrt{x^2 - x - 2} = x + 1$
    \end{enumerate}
    
    \item Résoudre dans $\mathbb{R}$ les inéquations suivantes.
    \begin{enumerate}
        \item $\sqrt{(x + 3)(x - 1)} \leq x - 3$
        \item $\sqrt{-5x + 1} \geq \sqrt{x^2 - 3x + 2}$
        \item $x^4 + x^2 - 12 \leq 0$
    \end{enumerate}
    
    \item Résoudre dans $\mathbb{R}^2$ le système suivant
    \[
    \begin{cases}
        x^3 + y^3 = 9 \\
        \frac{x}{y} + \frac{y}{x} = \frac{5}{2}
    \end{cases}
    \]
    
    \item Résoudre dans $\mathbb{R}^3$ par la méthode du Pivot de Gauss le système
    \[
    \begin{cases}
        x - 2y + z = 6 \\
        -2x + y - z = -6 \\
        3x - y - 2z = -2
    \end{cases}
    \]
\end{enumerate}

\textbf{\underline{\exercice}:}

On donne \( P(x) = 2x^3 + ax^2 + bx - 6 \) où \( a \) et \( b \) sont des réels.
\begin{enumerate}
    \item Trouver les réels \( a \) et \( b \) pour que le polynôme \( P(x) \) soit divisible par \( x^2 - x - 2 \).
    \item En déduire une factorisation de \( P(x) \).
    \item Résoudre dans \( \mathbb{R} \) l'équation \( P(x) = 0 \).
    \item En déduire les solutions de l'équation \( P(x^2 - 2) = 0 \).
    \item Résoudre dans \( \mathbb{R} \) l'inéquation \( P(x) \leq 0 \).
    \item En déduire les solutions de l'inéquation \( P(3 - 2x) \leq 0 \).
    \item Résoudre dans \( \mathbb{R} \) l'inéquation \( \frac{2x^3 + x^2 - 7x - 6}{-x + 1} > 0 \).
\end{enumerate}

\end{document}
