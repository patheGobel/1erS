\documentclass[12pt]{article}
\usepackage{stmaryrd}
\usepackage{graphicx}
\usepackage[utf8]{inputenc}

\usepackage[french]{babel}
\usepackage[T1]{fontenc}
\usepackage{hyperref}
\usepackage{verbatim}

\usepackage{color, soul}

\usepackage{pgfplots}
\pgfplotsset{compat=1.15}
\usepackage{mathrsfs}

\usepackage{amsmath}
\usepackage{amsfonts}
\usepackage{amssymb}
\usepackage{mdframed}
\usepackage{tkz-tab}
\author{Destinés à la 1erS\\Au Lycée de Dindéferlo}
\title{\textbf{Suites Numériques }}
\date{\today}
\usepackage{tikz}
\usetikzlibrary{arrows, shapes.geometric, fit}

% Commande pour la couleur d'accentuation
\newcommand{\myul}[2][black]{\setulcolor{#1}\ul{#2}\setulcolor{black}}
\newcommand\tab[1][1cm]{\hspace*{#1}}

\begin{document}
\maketitle
\newpage
\section*{\underline{\textbf{\textcolor{red}{I. Généralités}}}}
\subsection*{\underline{\textbf{\textcolor{red}{1. Définition}}}}
On appelle \textit{suite numérique} toute fonction définie de $\mathbb{N}$ ou d'une partie $E$ de $\mathbb{N}$ vers $\mathbb{R}$.

On note : $U : \mathbb{N} \rightarrow \mathbb{R}$ \\
\hspace*{3.3cm}$n \mapsto U_{n}$

Le réel $U_n$ est appelé  \textcolor{red}{\textit{terme général}} ou  \textcolor{red}{\textit{terme d'indice $n$}}.
L'ensemble des termes de la suite est noté $(U_n)$ et $n \in \mathbb{N}$ ou $(U_{n})n\in\mathbb{N}$.
\subsection*{\underline{\textbf{\textcolor{red}{2. Modes de définition d'une suite }}}}
\subsection*{\underline{\textbf{\textcolor{red}{2.1. Suite explicite}}}}
\subsection*{\underline{\textbf{\textcolor{red}{2.2.Définition}}}}
Lorsqu'une suite $(U_{n})$ est exprimée en fonction de $n$, alors on dit que la suite $(U_{n})$ est définie par une \textcolor{red}{\textit{formule explicite}} et on note $U_{n} = f(n)$.
\subsection*{\underline{\textbf{\textcolor{red}{Exemple :}}}}
Soit la suite $\left(u_{n}\right)$ définie par : $u_{n}=n^{2}-5n+3.$

Calculer 
$u_{0}$ ; $u_{1}$ ;	$u_{2}$ ; $u_{10}$ ; $u_{50}$
\subsection*{\underline{\textbf{\textcolor{red}{Solution :}}}}
$u_{0}=0^{2}-(5\times 0)+3=3$ ;
	
$u_{1}=1^{2}-(5\times 1)+3=-1$ ;
	
$u_{2}=2^{2}-(5\times 2)+3=-3$ ;
	
$u_{10}=10^{2}-(5\times 10)+3=53$ ;
	
$u_{50}=50^{2}-(5\times 50)+3=2253$
\subsection*{\underline{\textbf{\textcolor{red}{3.Suite définie par récurrence}}}}
\subsection*{\underline{\textbf{\textcolor{red}{3.1.Définition}}}}	
Lorsque la suite $(U_n)$ est définie par une relation entre $U_n$ et $U_{n+1}$, alors on dit que $(U_n)$ est définie par une \textcolor{red}{\textit{relation de récurrence}}.
\subsection*{\underline{\textbf{\textcolor{red}{Exemple 1 :}}}}
\begin{equation*} 
    \begin{cases}
        u_{0}=2 \\
        U_{n + 1}=\frac{2U_n + 3}{U_n + 1}
    \end{cases}
\end{equation*}
Calculer les cinq premier termes de $u_{n}$.
\subsection*{\underline{\textbf{\textcolor{red}{Solution :}}}}
\subsection*{\underline{\textbf{\textcolor{red}{Exemple 2 :}}}}
Soit la suite $\left(u_{n}\right)$ définie par :
	
$u_{n+1}=2u_{n}+3\text{ et }u_{0}=-1.$

Calculer les trois premier termes de $u_{n}$.
\subsection*{\underline{\textbf{\textcolor{red}{Solution :}}}}	
$u_{1}=u_{0+1}=2u_{0}+3=1$ ;
	
$u_{2}=u_{1+1}=2u_{1}+3=5$ ;
	
$u_{3}=u_{2+1}=2u_{2}+3=13\ ;\ \text{ etc}\ldots$
	
Par exemple, pour calculer $u_{50}$, il faudrait faire $50$ calculs successifs.
\subsection*{\underline{\textbf{\textcolor{red}{Exercie d'application :}}}}
Dans chacun des cas suivants, calculer les $6$ premiers termes de la suite $U_{n}$.
\begin{itemize}
\item[1.]$U_{n} = 7n^{2} - 5n + 2$, $n \in \mathbb{N}$.

\item[2.]$U_{n}=\frac{3n-5}{n+2}$, $n \in \mathbb{N}$
\end{itemize}
\subsection*{\underline{\textbf{\textcolor{red}{4. Sens de variation d'une suite}}}}
\subsection*{\underline{\textbf{\textcolor{red}{Définition :}}}}
	
Une suite $\left(u_{n}\right)$ est dite :
	
$-\ $Croissante si : $\forall\;n\;,\ :\ u_{n+1}\geq u_{n} .$ c'est-à-dire, $\ u_{n+1}- u_{n}>0$
	
$-\ $Décroissante si : $\forall\;n;,\ :\ u_{n+1}\leq u_{n}$. c'est-à-dire, $\ u_{n+1}- u_{n}<0$

$-\ $Monotone si elle est croissante ou décroissante.

$-\ $Constante si : $\forall\;n\;,\ :\ u_{n+1}=u_{n}.$

Étudier le sens de variation d'une suite $\left(u_{n}\right)$

C'est dire si elle est croissante ou décroissante ou constante.
\subsection*{\underline{\textbf{\textcolor{red}{Règle :}}}}
Pour étudier le sens de variation d'une suite $\left(u_{n}\right)$,on compare deux termes consécutifs, pour cela, on peut étudier le signe de leur différence, ou, s'il s'agit de nombres strictement positifs, comparer leur quotient à $1.$
\subsection*{\underline{\textbf{\textcolor{red}{Exemple :}}}}
Soit la suite $\left(u_{n}\right)$ définie par : $u_{n}=\dfrac{n+2}{2n+1}$

Alors :

$u_{n+1}=\dfrac{(n+1)+2}{2(n+1)+1}=\dfrac{n+3}{2n+3}$
	
$u_{n+1}-u_{n}=\dfrac{n+3}{2n+3}-\dfrac{n+2}{2n+1}=\dfrac{-3}{(2n+1)(2n+3})$
	
Pour tout entier naturel $n$, on a donc : $u_{n+1}-u_{n} < 0.$
	
La suite étudiée est par conséquent décroissante.
\section*{\underline{\textbf{\textcolor{red}{II. Suite arithmétiques}}}}
Une suite $\left(u_{n}\right)$ est \textcolor{red}{arithmétique} si chaque terme s'obtient en ajoutant au précédent un même nombre \textcolor{red}{$r$} appelé raison : 
\textcolor{red}{$u_{n+1}=u_{n}+r.$} 
\subsection*{\underline{\textbf{\textcolor{red}{1.Expression du terme général}}}}		
$\bullet\ \text{Si }\left(u_{n}\right)$ est une suite arithmétique de premier terme $u_{0}$ et de raison $r$, alors :
	
$$u_{n}=u_{0}+nr$$
	
$\bullet\ $Si le premier terme est $u_{1}$, alors :
$$u_{n}=u_{1}+(n-1)r$$

$\bullet\ $ Si $(U_n)$ une suite arithmétique de raison $r$ et de premier terme $U_{p}$ alors on a : 
\begin{mdframed}[linecolor=red] % Définit la couleur du cadre
    \[
    \color{red} u_{n} = u_{p} + (n-p)r
    \]
\end{mdframed}
Avec $u_{p}$ le premier terme, p l'indice du premier terme et r la raison
\subsection*{\underline{\textbf{\textcolor{red}{Exercie d'application 1 :}}}}
Dans chacun des cas suivants, donner le terme général de la suite.
\begin{itemize}
\item[1.]$(u_n)$ est une suite arithmétique de raison $r = 3$ et de premier terme $u_{0}= 7$.
\item[2.]$(u_n)$ est une suite arithmétique de raison $r=-\frac{3}{4}$ et de premier terme 
$u_{1}=\frac{-7}{4}$.
\item[3.] $(U_n)$ est une suite arithmétique de raison $r$ et de premier terme $U_{0}$ tels que $U_{50} = 406$ et $U_{100} = 806$.
\end{itemize}
\subsection*{\underline{\textbf{\textcolor{green}{Correction :}}}}
\subsection*{\underline{\textbf{\textcolor{red}{Exercie d'application 2 :}}}}
Le prix du transport augmente de $100^{F}$ chaque année dans une ville. En $2019$, le prix du transport était de $600FCFA$.
\begin{itemize}
\item[1.]Calculer le prix du transport en $2020$, $2021$ et en $2022.$
\item[2.]On note par $U_{0}$ le prix en $2019$ et $U_{n}$ le prix en $2019 + n.$
\begin{itemize}
	\item[a.]Que représente $U_{1}$, $U_{2}$ et $U_{3}$ ?
	\item[b.]Exprimer $U_{n+1}$ en fonction de $U_{n}.$
	\item[c.]Quelle est la nature de la suite $U_{n}.$
	\item[d.]Exprimer $U_{n}$ en fonction de $n$.
\end{itemize}
\end{itemize}
\subsection*{\underline{\textbf{\textcolor{red}{Correction :}}}}
\subsection*{\underline{\textbf{\textcolor{red}{A retenir 1 }}}}
Pour montrer qu'une suite $(U_n)$ est arithmétique de raison $r$, il suffit de montrer que :
\[ U_{n+1} - U_n = r \]
\subsection*{\underline{\textbf{\textcolor{red}{Exemple :}}}}
On considère la suite $(U_n)$ définie par :\[ U_n = 5n + 3 \]
Montrer que $(U_n)$ est une suite arithmétique dont on précisera la raison et le premier terme.
\subsection*{\underline{\textbf{\textcolor{red}{A retenir 2}}}}
Pour montrer qu’une suite $(U_n)$ n’est pas arithmétique , il suffit de montrer que : 
$U_{1}$-$U_{0}\neq U_{2}$-$U_{1}$.
\subsection*{\underline{\textbf{\textcolor{red}{Exemple :}}}}
Soit la suite $(U_n)$ définie par : $U_{n}=\frac{n}{n+1}$

Montrer que $(U_n)$ n’est pas une suite arithmétique.
\subsection*{\underline{\textbf{\textcolor{red}{2.Somme des premiers termes}}}}	
Soit (Un) une suite arithmétique.

Pour tous entiers naturels n et p tels que $p \leq n$, on a :

$\bullet\ $Si la suite a pour premier terme $u_{0}$, alors la somme $S_{n}=u_{0}+u_{1}+\ldots+u_{n}$ vaut :
	
$$S_{n}=\dfrac{(n+1)\left(u_{0}+u_{n}\right)}{2}$$

$\bullet\ $Si la suite a pour premier terme $u_{1}$, alors la somme $S_{n}=u_{1}+u_{1}+\ldots+u_{n}$ vaut :

$$S_{n}=\dfrac{n(u_{1}+u_{n})}{2}$$

$\bullet\ $De façon général si la suite a pour premier terme $u_{p}$, alors la somme $S_{n}=u_{p}+u_{p+1}+\ldots+u_{n}$ vaut :

\begin{mdframed}[linecolor=red] % Définit la couleur du cadre
    \[
    \color{red} S_{n}=\dfrac{(n-p+1)(u_{p}+u_{n})}{2}
    \]
\end{mdframed}
\subsection*{\underline{\textbf{\textcolor{red}{Exercie d'application :}}}}
Soit $\left(u_{n}\right)$ la suite définie par : $u_{n}=2n+7$
\begin{itemize}
\item[1.]Calculer $U_{1}$, $U_{2}$ et $U_{2024}.$
\item[2.] Montrer que (Un) est une suite arithmétique.
\item[3.] Calculer la somme : $S_{n} = U_{0} + U_{1} + \cdots+ U_{n}.$
\item[4.] En déduire le somme $S = U_{1} + U_{2} +\cdots+ U_{2024}.$
\end{itemize}
\section*{\underline{\textbf{\textcolor{red}{III. Suites géométriques}}}}
Une suite $\left(u_{n}\right)$ est dite géométrie si chaque terme s'obtient en multipliant le précédent par un même nombre $q$ appelé raison : $u_{n+1}=u_{n}\times q.$
\subsection*{\underline{\textbf{\textcolor{red}{1. Expression du terme général}}}}
$\bullet\ $Si la suite géométrique $\left(u_{n}\right)$ a pour premier terme $u_{0}$ et pour raison $q$, alors : $$u_{n}=u_{0}\times q^{n}$$

$\bullet\ $Si le premier terme est $u_{1}$, alors :$$u_{n}=u_{1}\times q^{n-1}$$

$\bullet\ $ Si le premier terme est $u_{p}$, alors:
\begin{mdframed}[linecolor=red] % Définit la couleur du cadre
    \[
    \color{red} u_{n}=u_{p}\times q^{n-p}
    \]
\end{mdframed}
\subsection*{\underline{\textbf{\textcolor{red}{Exemple }}}}
Dans chacun des cas suivants, montrer que $(v_{n})$ est une suite géométrique dont on précisera
la raison et le premier terme puis exprimer $(v_{n})$ en fonction de $n$.
\begin{itemize}
\item[1.]$v_{n} = 3 \times 5^{n}$
\item[2.] \begin{equation*} 
    \begin{cases}
        U_{0}=5 \\
        U_{n + 1}=2U_{n}+3
    \end{cases}
    et\quad V_{n} = U_{n} + 3
\end{equation*}
\item[3.] \begin{equation*} 
    \begin{cases}
        U_{1}=0 \\
        U_{n + 1}=-\frac{2}{3}U_{n}+1
    \end{cases}
    et\quad V_{n} = U_{n} - \frac{3}{5}
\end{equation*}
\end{itemize}
\subsection*{\underline{\textbf{\textcolor{red}{A retenir }}}}
Pour montrer qu’une suite $u_{n}$ est géométrique de raison q, il suffit de montrer que :
\begin{mdframed}[linecolor=red] % Définit la couleur du cadre
    \[
    \color{red} \dfrac{U_{n+1}}{U_{n}}=q
    \]
\end{mdframed}
\subsection*{\underline{\textbf{\textcolor{red}{2. Somme des premiers termes }}}}

Pour toute suite géométrique, de raison $q\neq 1$, on a:

$$S_{n}=u_{0}+u_{1}+\ldots+u_{n}=u_{0}\times\dfrac{1-q^{n+1}}{1-q}$$

$$S_{n}u_{1}+u_{2}+\ldots+u_{n}=u_{1}\times\dfrac{1-q^{n}}{1-q}$$

$$S_{n}=u_{p}+u_{p+1}+\ldots+u_{n}=u_{p}\times\dfrac{1-q^{n-p+1}}{1-q}$$

\begin{mdframed}[linecolor=red] % Définit la couleur du cadre
    \[
    \color{red} S_{n}=u_{p}\times\dfrac{1-q^{n-p+1}}{1-q}
    \]
\end{mdframed}
\section*{\underline{\textbf{\textcolor{red}{3. Sens de variation}}}}
    \begin{itemize}
        \item Si $0 < q < 1$, la suite $\left(u_{n}\right)$ est décroissante.
        \item Si $q > 1$, la suite $\left(u_{n}\right)$ est croissante.
        \item Si $q=1$, la suite $\left(u_{n}\right)$ est constante.
    \end{itemize}
\subsection*{\underline{\textbf{\textcolor{red}{Exemple :}}}}
$1+\dfrac{1}{2}+\dfrac{1}{2^{2}}+\ldots\ldots+\dfrac{1}{2^{n}}=S_{n}$ est la somme des premiers termes d'une suite géométrique de raison $\dfrac{1}{2}.$ 

Donc : $1+\dfrac{1}{2}+\dfrac{1}{2^{2}}+\ldots\ldots+\dfrac{1}{2^{n}}=\dfrac{1-\left(\dfrac{1}{2}\right)^{n+1}}{1-\dfrac{1}{2}}=2-\dfrac{1}{2^{n}}$
\section*{\underline{\textbf{\textcolor{red}{IV. limites d'une suite}}}}
La notion de limite en $+\infty$, déjà rencontrée à propos des fonction, s'étend au cas des suites.
	
On a les résultats suivantes :
\subsection*{\underline{\textbf{\textcolor{red}{Théorème 1 :}}}}
a. $\lim\limits_{n\;\longrightarrow\;+\infty}\sqrt{n}=+\infty\ ;\ \lim\limits_{n\;\longrightarrow\;+\infty}n^{2}=+\infty\ ;\ \lim\limits_{n\;\longrightarrow\;+\infty}n^{3}=+\infty.$

b. $\lim\limits_{n\;\longrightarrow\;+\infty}\dfrac{1}{\sqrt{n}}=0\ ;\ \lim\limits_{n\;\longrightarrow\;+\infty}\dfrac{1}{n}=0\ ;\ \lim\limits_{n\;\longrightarrow\;+\infty}\dfrac{1}{n^{2}}=0\ ;\ \lim\limits_{n\;\longrightarrow\;+\infty}\dfrac{1}{n^{3}}=0.$
\subsection*{\underline{\textbf{\textcolor{red}{Théorème 2 :}}}}
Soit $q$ un nombre réel.
Soit \( q \) un nombre réel.

\begin{itemize}
    \item Si \( q > 1 \), alors
    \[
    \lim_{n \rightarrow +\infty} q^n = +\infty.
    \]
    
    \item Si \( -1 < q < 1 \), alors
    \[
    \lim_{n \rightarrow +\infty} q^n = 0.
    \]
\end{itemize}
\subsection*{\underline{\textbf{\textcolor{red}{Théorème 3 :}}}}
Les résultats concernant les opérations sur les limites de fonctions s'étendent aux limites de suites.
\subsection*{\underline{\textbf{\textcolor{red}{Exemple :}}}}

1) Soit la suite \( (u_{n}) \) définie par : \( u_{n} = \dfrac{3n^{3} - 5n^{2} + 1}{2n^{3} + 1} \)

Alors \( \lim_{n \rightarrow +\infty} u_{n} = \lim_{n \rightarrow +\infty} \dfrac{3n^{3}}{2n^{3}} = \dfrac{3}{2} \).

2) Soit la suite \( (v_{n}) \) définie par : \( v_{n} = 1 + \dfrac{1}{3} + \left(\dfrac{1}{3}\right)^{2} + \ldots + \left(\dfrac{1}{3}\right)^{n+1} \).

On a d'après le paragraphe III :

\[
v_{n} = \dfrac{1 - \left(\dfrac{1}{3}\right)^{''}}{1 - \dfrac{1}{3}}
\]

car \( v_{n} \) est la somme des termes consécutifs d'une suite géométrique de raison \( \dfrac{1}{3} \), or comme \( -1 < \dfrac{1}{3} < 1 \), \( \lim_{n \rightarrow +\infty} \left(\dfrac{1}{3}\right)^{n} = 0 \).

D'où : \( \lim_{n \rightarrow +\infty} \dfrac{1}{1 - \dfrac{1}{3}} = \dfrac{3}{2} \).
\end{document}