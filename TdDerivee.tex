\documentclass[12pt]{article}
\usepackage{stmaryrd}
\usepackage{graphicx}
\usepackage[utf8]{inputenc}

\usepackage[french]{babel}
\usepackage[T1]{fontenc}
\usepackage{hyperref}
\usepackage{verbatim}

\usepackage{color, soul}

\usepackage{pgfplots}
\pgfplotsset{compat=1.15}
\usepackage{mathrsfs}

\usepackage{amsmath}
\usepackage{amsfonts}
\usepackage{amssymb}
\usepackage{tkz-tab}
\author{\\Lycée de Dindéfelo\\Mr BA}
\title{\textbf{Dérivabilité}}
\date{\today}
\usepackage{tikz}
\usetikzlibrary{arrows, shapes.geometric, fit}

% Commande pour la couleur d'accentuation
\newcommand{\myul}[2][black]{\setulcolor{#1}\ul{#2}\setulcolor{black}}
\newcommand\tab[1][1cm]{\hspace*{#1}}

\begin{document}
\maketitle
\newpage
\section*{\underline{\textbf{\textcolor{red}{Exercice 1}}}}
Dans chacun des cas suivants, calculer la dérivé et donner son signe.

(1) \[f(x)=3x^{2}-5x+2\]

(2) \[f(x)=-x^{2}+x-3\]

(3) \[f(x)=-\frac{1}{3}x^{3}-x^{2}+1\]

(4) \[f(x)=x^{4}-2x^{2}+1\]

(5) \[f(x)=x^{3}-\frac{1}{3}x^{2}+x-2\]

(6) \[f(x)=-\frac{3}{4}x^{4}+\frac{1}{3}x^{3}+2\]

(7) \[f(x)=\frac{1}{2}x^{4}+\frac{5}{3}x^{3}-x^{2}-5x\]

(8) \[f(x)=(x^{2}+1)(5x-7)\]
\section*{\underline{\textbf{\textcolor{red}{Exercice 2}}}}
Dans chacun des cas suivants, calculer la dérivé et donner son signe.
\[(1).f(x)=\frac{-2x+1}{3x+5}\]
\[(2).f(x)=x+3+\frac{4}{x-1}\]
\[(3).f(x)=\sqrt{2x+1}\]
\[(4).f(x)=\frac{1}{\sqrt{3x-1}}\]
\section*{\underline{\textbf{\textcolor{red}{Exercice 3}}}}
$f$ étant la fonction de $\mathbb{R}$ vers $\mathbb{R}$, calculer les nombres dérivés dans chacun des cas suivants.

(1)$f'(-3)$ et $f'(2)$ : $f(x)=|1+x|+2x^{2}$

(2)$f'(-2)$ et $f'(0)$ : $f(x)=|1-x^{2}|-x$
\section*{\underline{\textbf{\textcolor{red}{Exercice 4}}}}
$f$ étant une fonction de $\mathbb{R}$ vers $\mathbb{R}$, calculer les nombres dérivés en $a$, dans chacun des cas suivants:

(1) \[f(x)=3x^{4}+5x^{2}+1\quad ;\quad \alpha =-2\]

(2) \[f(x)=\frac{3x-1}{x^{2}+2}\quad ;\quad \alpha =-1\]

(3) \[f(x)=2x^{3}+\frac{4}{x^{2}+1}\quad ;\quad \alpha =-1\]

(4) \[f(x)=\sqrt{3x+5}\quad ;\quad \alpha =0\]

(5) \[f(x)=\frac{1}{\sqrt{1-2x}}\quad ;\quad \alpha =0\]

(6) \[f(x)=\cos x +\sin x\quad ;\quad \alpha =\frac{\pi}{6}\]

(7) \[f(x)=x\sin x +1\quad ;\quad \alpha =\frac{\pi}{3}\]

(8) \[f(x)=(x^{2}+1)\tan x\quad ;\quad \alpha =\frac{\pi}{4}\]

(7) \[f(x)=\sin^{2}x-3\sin x\quad ;\quad \alpha =\frac{5\pi}{6}\]

(8) \[f(x)=\cos(x-2)\quad ;\quad \alpha = \frac{\pi}{3}+2 \]
\section*{\underline{\textbf{\textcolor{red}{Exercice 5 : Dérivabilité en a}}}}
Dans chacun des cas suivants, $f$ est uen fonction de $\mathbb{R}$ vers $\mathbb{R}$ définie par son expression explicite.

A l'aide du taux de variation de $f$ en $1$, étudier la dérivabilité de $f$ en $1$

\[(1) f(x)=-2x+7\quad \quad (4)f(x)=\sqrt{x+1}\]

\[(2) f(x)=3x^{2}+5\quad ;\quad (5)f(x)=\frac{1}{x^{2}+1}\]

\[(3) f(x)=2x^{3}-9\quad ;\quad (6)f(x)=\frac{x^{2}}{x+1}\]
\section*{\underline{\textbf{\textcolor{red}{Exercice 6 : Dérivabilité en a}}}}
Etudier la dérivabilité en $0$ de chacune des fonctions $f$, $g$, $h$ définié par :

\[f(x)=|x^{3}|\quad\quad ;\quad\quad g(x)=x|x|\quad\quad ;\quad\quad h(x)=\sqrt{|x|}\]

\section*{\underline{\textbf{\textcolor{red}{Exercice 7 : Fonction dérivée}}}}
Démontrer que la fonction $f$ est dérivable de $\mathbb{R}$ vers $\mathbb{R}$ définie par:\\
$f(x)=|x|x^{2}$ est dérivable sur $\mathbb{R}$.

déterminer $f'$

Dans le plan, muni d'un repère orthogonal, contruire les représentation graphiques de $f$ et de $f'$.
\section*{\underline{\textbf{\textcolor{red}{Exercice 8 : Fonction dérivée}}}}
Démontrer que la fonction $f$ de $\mathbb{R}$ vers $\mathbb{R}$ définie par: $f(x)=x^{2}\sqrt{x}$ est dérivable en $0$ et sur $]0; +\infty[$. 

Déterminer $f'.$
\section*{\underline{\textbf{\textcolor{red}{Exercice 9 : Fonction dérivée}}}}
Le plan est muni d'un repère (O, I, J). La courbe $C_{f}$ est la représentation graphiquede la fonction f définie par $f(x)=x^{4}-2x^{2}+2x$.

Démontrer que la droite $(\Delta)$ d'équation $y=2x-1$ est tangente en deux points de la courbe $C_{f}$
\section*{\underline{\textbf{\textcolor{red}{Exercice 10 : Opération sur les fonctions dérivables}}}}
$f$ étant une fonction polyôme, déterminer sa dérivée dans chacun des cas suivants.

\[(1).\quad f(x)=-3x+1\]

\[(2).\quad f(x)=-3x^{2}+\frac{1}{2}x-2\]

\[(3).\quad f(x)=(3x-2)^{3}\]

\[(4).\quad f(x)=2x^{2}+6x-1\]

\[(5).\quad f(x)=(x^{2-1})^{4}\]

\[(6).\quad f(x)=3x(x-4)^{2}(3x+1)^{3}\]

\[(7).\quad f(x)=-\frac{1}{4}x^{4}+\frac{2}{3}x^{2}-\frac{3}{4}x-5\]

\[(8).\quad f(x)=(2x+1)^{3}(1-5x)^{4}\]

\[(9).\quad f(x)=\frac{1}{3}x^{3}+3x^{2}-\frac{3}{4}x-5\]

\[(10).\quad f(x)=x^{4}-x^{3}+3x^{2}-7\]

\[(11).\quad f(x)=-\frac{3}{5}x^{3}-5x^{2}+\frac{2}{3}x-2\]

\[(8).\quad f(x)=x^{4}(x^{3}+4x-3)^{3}\]
\end{document}