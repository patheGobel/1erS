\documentclass[a4paper,12pt]{article}
\usepackage{amsmath, amssymb}
\usepackage{xcolor}
\usepackage[utf8]{inputenc} % pour pdflatex
\usepackage[T1]{fontenc}    % améliore le rendu des accents
\usepackage[french]{babel}
\usepackage{tikz}
\usetikzlibrary{calc}
\usepackage{yhmath}
\usepackage{tcolorbox}
\usepackage{enumitem}
\usepackage{graphicx} % pour inclure les images
\usepackage{mathrsfs}
\begin{document}

\section*{\underline{\textbf{\textcolor{red}{L'outil métrique trigonométrique}}}}

\subsection*{\textcolor{red}{A. Angles orientés}}

\subsection*{\textcolor{red}{I. Cercle trigonométrique, radian}}

\subsubsection*{\textcolor{red}{1. Cercle trigonométrique}}

\underline{\textbf{\textcolor{red}{Définition 1}}}

\textcolor{red}{Dans un repère orthonormé} \( (O, \vec{i}, \vec{j}) \) on appelle \textcolor{red}{cercle trigonométrique} le cercle orienté \( C(0,1) \), de centre \( O \) et de rayon 1. Sur ce cercle on définit une origine \( I \) et deux sens.


    \begin{itemize}
        \item[\textcolor{red}{-}] Le sens direct, ou sens trigonométrique (positif), est le sens contraire du déplacement des aiguilles d'une montre.
        \item[\textcolor{red}{-}] Le sens indirect, ou sens antitrigonométrique (négatif), est le sens du déplacement des aiguilles d'une montre.
    \end{itemize}

\textcolor{red}{Remarque 1:} 

Puisque le rayon d'un cercle trigonométrique est 1, son périmètre est donné par \( P = 2\pi \) et l'aire de son disque est \( A = \pi \).

\begin{center}
\begin{tikzpicture}[scale=3]
    % Cercle
    \draw[gray] (0,0) circle(1);

    % Axes
    \draw[->] (-1.2,0) -- (1.5,0) node[right]{axe des abscisses};
    \draw[->] (0,-0.2) -- (0,1.5) node[above]{axe des ordonnées};

    % Points I, J, O
    \fill (1,0) circle(0.02) node[below right] {I};
    \fill (0,1) circle(0.02) node[above left] {J};
    \fill (0,0) circle(0.02) node[below left] {O};

    % Rayon OM (pi/4)
    \draw[thick] (0,0) -- ({cos(45)}, {sin(45)}) node[above right] {M};
    \fill ({cos(45)}, {sin(45)}) circle(0.02);

    % Angle alpha
    \draw[red, thick] (0.3,0) arc (0:45:0.3);
    \node at (0.35,0.13) {\textcolor{red}{$\alpha$}};

    % Arc de cercle IM
    \draw[red, thick] (1,0) arc (0:45:1);

    % Marquage de l'angle π
     \node at (1.2,0.5) {\textcolor{red}{$\wideparen{IM}$}};
\end{tikzpicture}

\vspace{0.3em}
$OI = 1 = OJ = R$
\end{center}
\subsubsection*{\textcolor{red}{2) Le radian}}

\underline{\textbf{\textcolor{red}{Définition 2}}}

\vspace{1em}

La mesure d’un angle \( \widehat{IOM} \) est de \textcolor{red}{1 radian} lorsque la mesure de l’arc du cercle trigonométrique qu’il intercepte est de \textcolor{red}{1 rayon}.

\vspace{0.5em}

Il existe une correspondance entre une mesure d’angle \( a \) \textcolor{red}{en degré} et \( b \) \textcolor{red}{en radian}, donnée par :
\[
\boxed{b = \dfrac{a\pi}{180}} \quad \text{ou encore} \quad \boxed{a = \dfrac{180b}{\pi}}
\]

\subsection*{\underline{\textcolor{red}{II. Angle orienté d’un couple de vecteurs}}}

\subsubsection*{\underline{\textcolor{red}{1) Angle géométrique - Angles orientés}}}

\textbf{\underline{\textcolor{red}{Définition 1 :}}}

Dans un repère \( (O, I, J) \), on considère deux vecteurs \( \vec{u} \) et \( \vec{v} \) non nuls.\\
Soient \( A \) et \( B \) deux points du plan tels que \( \vec{u} = \overrightarrow{OA} \) et \( \vec{v} = \overrightarrow{OB} \). Les deux angles \( \widehat{AOB} \) et \( \widehat{BOA} \) sont des angles géométriques de même mesure \textcolor{red}{toujours positive} : 
\(\textcolor{red}{\text{mes } \widehat{AOB} = \text{mes } \widehat{BOA}}\)

L’angle \( (\vec{u}, \vec{v}) \) formé par les deux vecteurs \( \overrightarrow{OA} \) et \( \overrightarrow{OB} \) est un angle orienté (on tourne de \( \vec{u} \) vers \( \vec{v} \)). Alors que l’angle \( (\vec{v}, \vec{u}) \) est un angle orienté mais de sens contraire. Donc : \( \textcolor{red}{ (\vec{v}, \vec{u}) = -(\vec{u}, \vec{v})}\)
\begin{tcolorbox}[colback=red!5!white, colframe=red!75!black, title=Propriété 1]
\( (\vec{v}, \vec{u}) = -(\vec{u}, \vec{v}) \)
\end{tcolorbox}
\vspace{1em}

\begin{center}
\begin{tikzpicture}[scale=1.5]
    \coordinate (O) at (0,0);
    \coordinate (A) at (3,0);
    \coordinate (B) at (1.5,1.5);

    % Vecteurs
    \draw[->, thick] (O) -- (A) node[below right] {A};
    \draw[->, thick] (O) -- (B) node[above right] {B};

    % Points
    \fill (O) circle(0.03) node[below left] {O};

    % Noms des vecteurs
    \node at (1.5,-0.2) {\( \vec{u} \)};
    \node at (0.6,1.2) {\( \vec{v} \)};

    % Arc orienté de u vers v
    \draw[red, ->, thick] (1.2,0) arc[start angle=0, end angle=45, radius=1.2];
    \node[red] at (2,0.4) {\small sens de $\vec{u} \rightarrow \vec{v}$};

\end{tikzpicture}
\end{center}


\vspace{1em}

\noindent
\textcolor{red}{\underline{Définition 2 :}} \

\textcolor{red}{\underline{Théorème :}}

\vspace{0.5em}

Soit \( M \) un point quelconque du cercle trigonométrique tel que 
\[
(\overrightarrow{OI}, \overrightarrow{OM}) = \alpha \text{ rad}.
\]
On peut lui associer une famille de nombres réels de la forme 
\[
\alpha + 2k\pi, \quad k \in \mathbb{Z}
\]
qui correspondent au même point \( M \) du cercle trigonométrique.

\vspace{1em}
\noindent\textbf{\underline{\textcolor{red}{Définition 2:}}}
 
Dans le plan muni de \( \mathcal{R}ON(O, \vec{i}, \vec{j}) \), on considère les vecteurs \( \vec{u} \) et \( \vec{v} \) non nuls tel que la mesure de l’angle \( (\vec{u}, \vec{v}) = \alpha \, \text{rad} \).\\
Alors chacun des nombres associés à \( \alpha \) de la forme \( \alpha + 2k\pi \), \( k \in \mathbb{Z} \), est appelé une mesure de l’angle orienté de vecteurs \( (\vec{u}, \vec{v}) \).
\subsection*{\underline{\textcolor{red}{2) Mesure principale d’un angle orienté}}}

\textbf{\underline{\textcolor{red}{Définition 3:}}}

Dans un repère orthonormé \( (O, \vec{i}, \vec{j}) \), on considère deux vecteurs \( \vec{u} \) et \( \vec{v} \) non nuls tel que \( (\vec{u}, \vec{v}) = \alpha \) rad.\\
Parmi toutes les mesures d’angles associées à \( \alpha \), de la forme \( \alpha + 2k\pi \), \( k \in \mathbb{Z} \), il existe une et une seule mesure qui appartient à l’intervalle \( ]-\pi, \pi] \).\\
Cette mesure particulière s’appelle la \textbf{mesure principale} de l’angle \( (\vec{u}, \vec{v}) \).

\vspace{1em}

\textcolor{red}{\textbf{Exemple :}} 

Dans chacun des cas suivants, détermine la mesure \textbf{principale} de l’angle \( (\vec{u}, \vec{v}) \) :

\[
\alpha = \dfrac{89\pi}{7} \qquad \beta = \dfrac{229\pi}{12}
\]
\subsubsection*{\underline{\textcolor{red}{Résolution :}}}

\subsubsection*{\underline{\textcolor{red}{a) Méthode algébrique}}}

\(
\alpha = \dfrac{89\pi}{7}
\), la mesure principale de \( \alpha \) est de la forme \( \alpha + 2k\pi \),

\( k \in \mathbb{Z} \) et \( -\pi < \alpha + 2k\pi \leq \pi \)

\( 
\begin{aligned}
-\pi < \dfrac{89\pi}{7} + 2k\pi &\implies \leq  -1 <  \dfrac{89}{7} + 2k \leq 1 \\
        &\implies -1 - \dfrac{89}{7} < 2k \leq 1 - \dfrac{89}{7}\\
        &\implies -\dfrac{96}{7} < 2k \leq \dfrac{-82}{7}\\
        &\implies \dfrac{-48}{7} < 2k \leq \dfrac{-41}{7}\\
        &\implies -6{,}10 < 2k \leq -5{,}86
\end{aligned}
\)

donc $\boxed{k =-6}$

\subsubsection*{\underline{\textcolor{red}{b) Méthode pratique (facile rigoureuse et pratique)}}}

\[
\begin{array}{c|c}
89 & 7 \\
\hline
18 &  \\
5 & 
\end{array}
\]

\[
\dfrac{89\pi}{7} = 12\pi + \dfrac{5\pi}{7}
\]

Comme \( 12\pi \) est le nombre pair de \( \pi \), donc \( \dfrac{5\pi}{7} \) est la mesure principale de \( \dfrac{89\pi}{7} \)

\subsection*{\underline{\textcolor{red}{III. Propriétés des angles orientés}}}

\subsubsection*{\textcolor{red}{1) Angle orienté de deux vecteurs colinéaires}}

\underline{\textcolor{red}{Théorème}}

Soient \( \vec{u} \) et \( \vec{v} \) deux vecteurs non nuls du plan dans un RON direct \( (O, I, J) \) :

\begin{itemize}
    \item[\( \mathbf{P_1} :\)] L’angle \( (\vec{u}, \vec{u}) = 0 \) et \( (\vec{u}, -\vec{u}) = 180 \)
    \item[\( \mathbf{P_2} :\)] Les vecteurs \( \vec{u} \) et \( \vec{v} \) sont deux vecteurs colinéaires et de même sens ssi \( (\vec{u}, \vec{v}) = 0 \)
    \item[\( \mathbf{P_3} :\)] Les vecteurs \( \vec{u} \) et \( \vec{v} \) sont colinéaires et de sens contraires ssi \( (\vec{u}, \vec{v}) = \pi \)
\end{itemize}

Ces propriétés permettent de démontrer le parallélisme de deux droites, l’alignement de 3 points ou la colinéarité de deux vecteurs.

\subsubsection*{\underline{\textcolor{red}{2) Relation de Chasles}}}

Soient \( \vec{u}, \vec{v} \) et \( \vec{w} \) trois vecteurs non nuls du plan dans un RON direct \( (O, I, J) \)

\( \textbf{P}_4: \quad (\vec{u}, \vec{v}) + (\vec{v}, \vec{w}) = (\vec{u}, \vec{w}) \)

\subsubsection*{\underline{\textcolor{red}{3) Angle orienté et vecteurs opposés}}}

Soient \( \vec{u} \) et \( \vec{v} \) deux vecteurs non nuls du plan dans un repère orthonormé \( (O, I, J) \) :

\(
\begin{aligned}
\textbf{P}_5: \quad & (\vec{v}, \vec{u}) = -(\vec{u}, \vec{v})\\ 
\textbf{P}_6: \quad & (\vec{u}, -\vec{v}) = (\vec{u}, \vec{v}) + \pi \\
\textbf{P}_7: \quad & (-\vec{u}, \vec{v}) = (\vec{u}, \vec{v}) + \pi \\
\textbf{P}_8: \quad & (-\vec{u}, -\vec{v}) = (\vec{u}, \vec{v})
\end{aligned}
\)

\subsection*{\underline{\textcolor{red}{Application}}}

\textbf{Démo :} Soit \( ABC \) un triangle, montrer que la somme des mesures orientées de ses trois angles est égale à \( \pi \).

\begin{center}
\begin{tikzpicture}[scale=3]
    % Coordonnées des sommets
    \coordinate (A) at (0,1.8);
    \coordinate (B) at (-1,0);
    \coordinate (C) at (1,0);

    % Triangle
    \draw[thick] (A) -- (B) -- (C) -- cycle;

    % Points nommés
    \fill (A) circle(0.02) node[above] {A};
    \fill (B) circle(0.02) node[below left] {B};
    \fill (C) circle(0.02) node[below right] {C};

    % Angle orienté en A (de AB vers AC)
    %\node at (-0.2,1.5) {I};
    %\node at (0.2,1.5) {J};
    \draw[->, thick, red] (-0.2,1.5) arc[start angle=180, end angle=360, radius=0.17];

    % Angle orienté en B (de BC vers BA → sens antihoraire donc 330 à 240)
    \draw[->, red, thick] (B)++(0.3,0.01) arc[start angle=330,end angle=420,radius=0.25];

    % Angle orienté en C (de CB vers CA → sens antihoraire donc 150 à 90)
    %\node at (0.5,0.01) {K};
    %\node at (0.,0.5) {L};
    \draw[->, thick, red] (0.9,0.13) arc[start angle=0, end angle=270, radius=0.1];
    %\draw[->, red, thick] (C)++(-0.3,0.01) arc[start angle=150,end angle=90,radius=0.25];
\end{tikzpicture}
\end{center}

\[
\widehat{A} = (\overrightarrow{AB}, \overrightarrow{AC}) \qquad
\widehat{B} = (\overrightarrow{BC}, \overrightarrow{BA}) \qquad
\widehat{C} = (\overrightarrow{CA}, \overrightarrow{CB})
\]

\(
\begin{aligned}
    \widehat{A} + \widehat{B} + \widehat{C} &= (\overrightarrow{AB}, \overrightarrow{AC}) + (\overrightarrow{BC}, \overrightarrow{BA}) + (\overrightarrow{CA}, \overrightarrow{CB})\\
    &= (\overrightarrow{AB}, \overrightarrow{AC}) + (\overrightarrow{BC}, -\overrightarrow{AB}) + (-\overrightarrow{AC}, -\overrightarrow{BC})\\
    &= (\overrightarrow{AB}, \overrightarrow{AC}) + \pi + (\overrightarrow{BC}, \overrightarrow{AB}) + (\overrightarrow{AC}, \overrightarrow{BC})\\
    &= (\overrightarrow{AB}, \overrightarrow{AC}) + (\overrightarrow{AC}, \overrightarrow{BC}) + (\overrightarrow{BC}, \overrightarrow{AB}) + \pi\\
    &= (\overrightarrow{AB}, \overrightarrow{AB}) + \pi \\
    &= 0 + \pi \\
    &= \pi
\end{aligned}
\)

\( \widehat{A} + \widehat{B} + \widehat{C} = \pi \)

\subsubsection*{\underline{\textcolor{red}{4) Généralisation}}}

Soient \( \overrightarrow{u} \) et \( \overrightarrow{v} \) deux vecteurs non nuls du plan dans un R.O.N direct \( (O, \overrightarrow{i}, \overrightarrow{j}) \).\\
Soient \( k \) et \( k' \) deux réels non nuls.

\begin{itemize}
    \item[\textcolor{red}{a)}] Si \( k \) et \( k' \) sont de même signe, alors :
    \[
    (k \overrightarrow{u}, k' \overrightarrow{v}) = (\overrightarrow{u}, \overrightarrow{v})
    \]
    
    \item[\textcolor{red}{b)}] Si \( k \) et \( k' \) sont de signes contraires, alors :
    \[
    (k \overrightarrow{u}, k' \overrightarrow{v}) = \pi + (\overrightarrow{u}, \overrightarrow{v})
    \]
\end{itemize}

\begin{tikzpicture}[scale=2, >=stealth]
    % Coordonnées des points
    \coordinate (A) at (-2,1);
    \coordinate (B) at (0,0);
    \coordinate (C) at (0,-1.5);
    \coordinate (D) at (1,0.5);
    \coordinate (E) at (2.5,0);

    % Segments
    \draw[thick] (A) -- (B) -- (C) -- (D) -- (E);

    % Points
    \fill (A) circle(0.02) node[above left] {A};
    \fill (B) circle(0.02) node[above left] {B};
    \fill (C) circle(0.02) node[below] {C};
    \fill (D) circle(0.02) node[above right] {D};
    \fill (E) circle(0.02) node[above right] {E};

    % Arcs d'angles orientés
    \draw[->, red, thick] (B)++(-0.2,0.1) arc[start angle=90, end angle=310, radius=0.25];
    \node at ($(B)+(-0.6,-0.05)$) {\small $\dfrac{2\pi}{3}$};

    \draw[->, red, thick] (C)++(0.3,0.6) arc[start angle=0, end angle=180, radius=0.15];
    \node at ($(C)+(0.25,0.3)$) {\small $\dfrac{\pi}{4}$};

    \draw[->, red, thick] (D)++(-0.1,-0.3) arc[start angle=-90, end angle=0, radius=0.25];
    \node at ($(D)+(0.2,-0.2)$) {\small $?$};

\end{tikzpicture}

\subsection*{\underline{\textcolor{red}{Application}}}

Déterminer \( (\overrightarrow{DC}, \overrightarrow{DE}) \) sachant que \( (AB) \parallel (DE) \)

\(
\begin{aligned}
(\overrightarrow{AB}, \overrightarrow{DE}) = 0 &\Rightarrow (\overrightarrow{AB}, \overrightarrow{BC}) + (\overrightarrow{BC}, \overrightarrow{CD}) + (\overrightarrow{CD}, \overrightarrow{DE}) = 0\\
&\Rightarrow (-\overrightarrow{BA}, \overrightarrow{BC}) + (-\overrightarrow{CB}, \overrightarrow{CD}) + (\overrightarrow{-DC}, \overrightarrow{DE}) = 0\\
&\Rightarrow \pi + (\overrightarrow{BA}, \overrightarrow{BC}) + \pi + (\overrightarrow{CB}, \overrightarrow{CD}) + \pi + (\overrightarrow{DC}, \overrightarrow{DE}) = 0\\
&\Rightarrow 3\pi + \dfrac{2\pi}{3} + \left( -\dfrac{\pi}{4} \right) + (\overrightarrow{DC}, \overrightarrow{DE}) = 0\\
&\Rightarrow (\overrightarrow{DC}, \overrightarrow{DE}) = - 3\pi - \dfrac{2\pi}{3} + \dfrac{\pi}{4}\\
&=-\dfrac{41\pi}{12}
\end{aligned}
\)

\[
\text{Mesure principale } (\overrightarrow{DC}, \overrightarrow{DE}) = -\dfrac{41\pi}{12} + 4\pi =   \dfrac{7\pi}{12}
\]

\[
\boxed{(\overrightarrow{DC}, \overrightarrow{DE}) = \dfrac{7\pi}{12}}
\]

\subsection*{\textcolor{red}{B. Trigonométrie}}

\subsection*{\textcolor{red}{I. Cosinus, sinus et tangente d’un nombre réel}}

\subsubsection*{\textcolor{red}{1.Définition et propriétés }}

Le plan est muni d’un repère orthonormé direct \( (O, I, J) \), $\mathcal{(C)}$ et le cercle trigonométrique.

L’unité de mesure des angles orientés est le radian.

\begin{flushleft}
\textit{(À considérer dans toute la suite de la leçon sauf mention contraire.)}
\end{flushleft}

\vspace{1em}
\textcolor{red}{\underline{a) Définition}}

\vspace{0.5em}

\( x \) est un nombre réel et \( \alpha \) la mesure principale de l’angle orienté dont \( x \) est une mesure.

\begin{itemize}[label=\textbullet]
    \item On appelle \textbf{cosinus de \( x \)}, le nombre réel \( \cos \alpha \)
    \item On appelle \textbf{sinus de \( x \)}, le nombre réel \( \sin \alpha \)
    \item On appelle \textbf{tangente de \( x \)}, le nombre réel \( \tan \alpha \)
\end{itemize}
\begin{center}
\begin{tikzpicture}[scale=3]
    % Axes
    \draw[->, thick] (-1.2,0) -- (1.5,0) node[right]{};
    \draw[->, thick] (0,-1.2) -- (0,1.5) node[above]{};

    % Cercle trigonométrique
    \draw[red, thick] (0,0) circle(1);

    % Angle alpha (magenta)
    %\draw[magenta, thick, ->] (0,0) -- ({cos(30)}, {sin(30)});
    \draw[magenta, thick] (-1.8,-1.5) -- (1.8,1.5); % prolongement droite MT'

    % Points
    \fill (0,0) circle(0.02) node[below left] {O};
    \fill (1,0) circle(0.02) node[below right] {A};
    \fill (0,1) circle(0.02) node[above left] {B};
    \fill ({cos(39)},{sin(39)}) circle(0.02) node[above right] {M};
    \fill ({cos(39)},0) circle(0.02) node[below] {H};
    \fill (0,{sin(39)}) circle(0.02) node[left] {K};
    \fill (1,{tan(39)}) circle(0.02) node[right] {T};
    \fill (1.2,{1.3*tan(37.4)}) circle(0.02) node[above right] {$T'$};

    % Droites perpendiculaires en bleu
    \draw[blue, thick] (-1.5, 1) -- (1.5, 1);
    \draw[blue, thick] (1, -1) -- (1, 1.5);

    % Pointillés verts
    \draw[green!70!black, dashed] ({cos(39)}, {sin(39)}) -- ({cos(39)}, 0);
    \draw[green!70!black, dashed] ({cos(39)}, {sin(39)}) -- (0, {sin(39)});

    % Angle alpha
    \draw[magenta] (0.3,0) arc (0:35:0.3);
    \node[magenta] at (0.35,0.07) {$\alpha$};

    % Formules à droite
    \node[right] at (1.8,1) {$\cos \alpha = \overline{OH}$};
    \node[right] at (1.8,0.5) {$\sin \alpha = \overline{OK}$};
    \node[right] at (1.8,0) {$\tan \alpha = \dfrac{\overline{OK}}{\overline{OH}} = \overline{AT}$};
    \node[right] at (1.8,-0.5) {$\cot \alpha = \dfrac{\overline{OH}}{\overline{OK}} = \overline{BT'}$};
    \node[right] at (1.8,-1) {$\sin^2\alpha + \cos^2\alpha = 1$};

    % Annotation
    \node at (0,-1.5) {\small \textit{Cercle trigonométrique : $OM = 1$}};

\end{tikzpicture}
\end{center}

\subsubsection*{Condition d’existence de $\tan x$}

$\tan x$ est un nombre réel \quad $\Longleftrightarrow \quad x \neq \dfrac{\pi}{2} + k\pi \quad \left[ k \in \mathbb{Z} \right]$

\subsection*{\textcolor{red}{b) Propriétés}}

\begin{itemize}
    \item Pour tout nombre réel \( x \) et pour tout entier relatif \( k \),
    \[
    -1 \leq \sin x \leq 1, \quad \cos^2 x + \sin^2 x = 1, \quad -1 \leq \cos x \leq 1,
    \]
    \[
    \cos(x + 2k\pi) = \cos x, \quad \sin(x + 2k\pi) = \sin x
    \]

    \item Pour tout nombre réel \( x \) qui n’annule pas \( \cos x \) et pour tout entier relatif \( k \),
    \[
    \tan x = \dfrac{\sin x}{\cos x}, \qquad \dfrac{1}{\cos^2 x} = 1 + \tan^2 x, \qquad \tan(x + k\pi) = \tan x
    \]
\end{itemize}

\subsection*{\underline{\textcolor{red}{2) Cosinus, sinus et tangente d’angles orientés associés}}}

\subsubsection*{\underline{\textcolor{red}{a) Propriétés}}}

\begin{center}
    \includegraphics[width=0.8\textwidth]{Ctrigo.png} % modifie ce chemin
\end{center}

\subsection*{\textcolor{red}{b) Application}}

Simplifier les expressions suivantes :

\begin{enumerate}
    \item 
    \(
    \sin(x + \pi) + \cos(x - \pi) - \sin(x - 2\pi) + \cos(x + 5\pi)
    \)
    
    \item 
    \(
    \cos\left( \dfrac{\pi}{2} - x \right) - \sin\left( x + \dfrac{\pi}{2} \right) + \cos\left( \dfrac{7\pi}{2} - x \right) - \sin\left( x + \dfrac{5\pi}{2} \right)
    \)
\end{enumerate}

\textbf{\textcolor{red}{Résolution :}}

\[
A = \sin(x + \pi) + \cos(x - \pi) - \sin(x - 2\pi) + \cos(x + 5\pi)
\]

\[
\begin{aligned}
A &= -\sin x - \cos x - \sin x + \cos(x + \pi) \\
  &= -\sin x - \cos x - \sin x - \cos x \\
  &= -2\sin x -2\cos x
\end{aligned}
\]

\medskip

\[
B = \cos\left(\frac{\pi}{2} - x\right) - \sin\left(x + \frac{\pi}{2}\right) + \cos\left(\frac{7\pi}{2} - x\right) - \sin\left(x + \frac{5\pi}{2}\right)
\]

\[
\begin{aligned}
B &= \sin x - \cos x + \cos\left(4\pi - \left(\frac{\pi}{2} + x\right)\right) - \cos\left(x + \frac{\pi}{2}\right) \\
  &= \sin x - \cos x + \cos\left(-\left(\frac{\pi}{2} + x\right)\right) - \cos\left(x + \frac{\pi}{2}\right) \\
  &= \sin x - \cos x + \cos\left(\frac{\pi}{2} + x\right) - \cos\left(x + \frac{\pi}{2}\right) \\
  &= \sin x - \cos x - \sin x - \cos x \\
  &= -2\cos x
\end{aligned}
\]

\subsection*{\textcolor{red}{II. Formules trigonométriques}}
\subsubsection*{\textcolor{red}{1.Formule d’addition }}
\subsubsection*{\textcolor{red}{a) Propriétés }}

\noindent
\textbf{Pour tous nombres réels $a$ et $b$ on a :}

\vspace{1em}

\begin{enumerate}
    \item[\textcolor{red}{1)}] $\cos(a - b) = \cos a \cos b + \sin a \sin b$
    \item[\textcolor{red}{2)}] $\cos(a + b) = \cos a \cos b - \sin a \sin b$
    \item[\textcolor{red}{3)}] $\sin(a - b) = \sin a \cos b - \sin b \cos a$
    \item[\textcolor{red}{4)}] $\sin(a + b) = \sin a \cos b + \sin b \cos a$
    \item[\textcolor{red}{5)}] \textcolor{blue}{$\forall \, a,b \in \mathbb{R} \setminus \left\{\dfrac{\pi}{2} + k\pi\right\}, \, k \in \mathbb{Z}$} : 
\(\tan(a + b) = \dfrac{\tan a + \tan b}{1 - \tan a \tan b}\)
    \item[\textcolor{red}{6)}]
$\tan(a - b) = \dfrac{\tan a - \tan b}{1 + \tan a \tan b}$
\end{enumerate}
\textbf{\underline{Preuve :}}

\begin{center}
\begin{tikzpicture}[scale=2.5]
    % Cercle trigonométrique
    \draw[thick] (0,0) circle(1);

    % Flèche d'orientation (en dehors du cercle)
    \draw[->, thick, red] (-1,1) arc[start angle=60, end angle=180, radius=0.3];
    \node at (-1.2,0.7) {\small $+$};

    % Axes
    \draw[->] (-1.2,0) -- (1.2,0);
    \draw[->] (0,-1.2) -- (0,1.2);

    % Centre O
    \fill (0,0) circle(0.02);
    \node[below left] at (0,0) {O};

    % Vecteurs
    \draw[->, thick] (0,0) -- ({cos(60)}, {sin(60)}) node[above right] {$\vec{u}$};
    \draw[->, thick] (0,0) -- ({cos(30)}, {sin(30)}) node[right] {$\vec{v}$};

    % Arcs d'angle
    \draw[->, thick] (0.3,0) arc(0:30:0.3); % b
    \node at ({0.36*cos(15)}, {0.36*sin(15)}) {\small $b$};

    \draw[->, thick] (0.6,0) arc(0:60:0.6); % a
    \node at ({0.7*cos(45)}, {0.7*sin(45)}) {\small $a$};

    % Point Q
    \node[below right] at (1,0) {$Q(0,1)$};

\end{tikzpicture}
\end{center}
On va calculer \( \vec{u} \cdot \vec{v} \) de deux manières différentes :

\vspace{1em}
\underline{\textbf{\textcolor{red}{- Trigonométrie}}} \\
\[
\vec{u} \cdot \vec{v} = \|\vec{u}\| \times \|\vec{v}\| \times \cos(\vec{u}, \vec{v}) \quad \text{or} \quad \|\vec{u}\| = \|\vec{v}\| = 1
\]
\[
\Rightarrow \vec{u} \cdot \vec{v} = \cos(a - b) \quad \text{(1)}
\]

\vspace{1em}
\underline{\textbf{\textcolor{red}{- Analytique}}} \\
\[
\vec{u} = (\cos a) .\vec{i} + (\sin a) .\vec{j} \textbf{ et } \vec{v} = (\cos b) .\vec{i} + (\sin b) .\vec{j}
\]
\[
\vec{u} \cdot \vec{v} = \left[(\cos a) .\vec{i} + (\sin a) .\vec{j}\right] \times \left[\cos b) .\vec{i} + (\sin b) .\vec{j}\right]
\]
\[
\vec{u} \cdot \vec{v} = \cos a \cos b + \sin b \sin a
\]
\[
\vec{u} \cdot \vec{v} = \cos a \cos b + \sin a \sin b \quad \text{(2)}
\]

(1) et (2) donnent : \(\cos(a - b) = \cos a \cos b + \sin a \sin b\)

\vspace{1em}
2) \( \cos(a + b) = \cos[a - (-b)] \), d’après (1) on a :

\(
\begin{aligned}
    \cos[a - (-b)] &= \cos a \cos(-b) + \sin a \sin(-b)\\
    &= \cos a \cos b + \sin a (-\sin b)\\
    &= \cos a \cos b + \sin a (-\sin b)\\
    \cos(a + b) &= \cos a \cos b - \sin a \sin b
\end{aligned}
\)

\vspace{1em}
%\newline
3)

\(
\begin{aligned}
    \sin(a - b) &= \cos\left[ \dfrac{\pi}{2} - (a - b) \right]\\
                &= \cos\left[ \left( \dfrac{\pi}{2} - a \right) + b \right]\\
                &= \cos\left( \dfrac{\pi}{2} - a \right)\cos b - \sin\left( \dfrac{\pi}{2} - a \right)\sin b\\
                &= \sin a \cos b - \cos a \sin b\\
    \sin(a - b) &= \sin a \cos b - \cos a \sin b
\end{aligned}
\)

\textbf{4) }

\(
\begin{aligned}
    \sin(a + b) &= \sin[a - (-b)]\\
                &= \sin a \cos (-b) - \sin (-b) \cos a\\
                &= \sin a \cos b + \sin b \cos a\\
\end{aligned}
\)

\vspace{1em}
\textbf{5)} 

\(
\begin{aligned}
    \tan(a + b) &= \dfrac{\sin(a + b)}{\cos(a + b)}\\
    &= \dfrac{\sin a \cos b + \sin b \cos a}{\cos a \cos b - \sin a \sin b}\\
    &= \dfrac{\dfrac{\sin a}{\cos a} + \dfrac{\sin b}{\cos b}}{1 - \dfrac{\sin a}{\cos a} \cdot \dfrac{\sin b}{\cos b}}\\
    &= \dfrac{\tan a + \tan b}{1 - \tan a \tan b}
\end{aligned}
\)

\vspace{1em}
\textbf{6)} 

\(
\begin{aligned}
    \tan(a - b) &= \tan[a + (-b)] \quad \text{d'après (5)}\\
                &= \dfrac{\tan a + \tan(-b)}{1 - \tan a \tan(-b)}\\
                &= \dfrac{\tan a - \tan b}{1 + \tan a \tan b}
\end{aligned}
\)

\subsubsection*{\textcolor{red}{b) Application }}

Donne la valeur exacte de \( \cos\left(\dfrac{\pi}{12}\right) \) et \( \sin\left(\dfrac{7\pi}{12}\right) \).

\subsection*{\textcolor{red}{Résolution :}}

\(
\begin{aligned}
\frac{\pi}{12} &= \frac{\pi}{3} - \frac{\pi}{4} \\
\cos\left(\frac{\pi}{12}\right) &= \cos\left(\frac{\pi}{3} - \frac{\pi}{4}\right) \\
&= \cos\left(\frac{\pi}{3}\right)\cos\left(\frac{\pi}{4}\right) + \sin\left(\frac{\pi}{3}\right)\sin\left(\frac{\pi}{4}\right) \\
&= \frac{1}{2} \cdot \frac{\sqrt{2}}{2} + \frac{\sqrt{3}}{2} \cdot \frac{\sqrt{2}}{2} \\
&= \frac{\sqrt{2}}{4} + \frac{\sqrt{6}}{4} \\
&= \boxed{\frac{\sqrt{2} + \sqrt{6}}{4}}
\end{aligned}
\)

\[
\begin{aligned}
\sin\left(\frac{7\pi}{12}\right) 
&= \sin\left(\frac{\pi}{3} + \frac{\pi}{4} \right) \\
&= \sin\left(\frac{\pi}{3}\right) \cos\left(\frac{\pi}{4}\right) + \sin\left(\frac{\pi}{4}\right) \cos\left(\frac{\pi}{3}\right) \\
&= \frac{\sqrt{3}}{2} \cdot \frac{\sqrt{2}}{2} + \frac{\sqrt{2}}{2} \cdot \frac{1}{2} \\
&= \frac{\sqrt{6}}{4} + \frac{\sqrt{2}}{4} \\
&= \boxed{\frac{\sqrt{6} + \sqrt{2}}{4}}
\end{aligned}
\]

\subsubsection*{\textcolor{red}{2.Formules de duplication }}
\subsubsection*{\textcolor{red}{a. Propriétés }}

\(
\begin{aligned}
&\textcolor{red}{\text{7)}}\cos(2a) = \cos^2 a - \sin^2 a = 2\cos^2 a - 1 = 1 - 2\sin^2 a \\
&\textcolor{red}{\text{8)}} \sin(2a) = 2 \sin a \cos a \\
&\textcolor{red}{\text{9)}}\cos^2 a = \frac{1 + \cos(2a)}{2} \\
&\textcolor{red}{\text{10)}}\sin^2 a = \frac{1 - \cos(2a)}{2}
\end{aligned}
\)

\vspace{0.5em}
\textit{Cas particulier : en posant } \( x = 2a \Leftrightarrow a = \frac{x}{2} \)

\[
\begin{aligned}
\cos x &= \cos^2\left(\frac{x}{2}\right) - \sin^2\left(\frac{x}{2}\right)
= 2 \cos^2\left(\frac{x}{2}\right) - 1 = 1 - 2 \sin^2\left(\frac{x}{2}\right) \\
\sin x &= 2 \sin\left(\frac{x}{2}\right) \cos\left(\frac{x}{2}\right) \\
\cos^2\left(\frac{x}{2}\right) &= \frac{1 + \cos(x)}{2}, \qquad 
\sin^2\left(\frac{x}{2}\right) = \frac{1 - \cos(x)}{2}
\end{aligned}
\]

\section*{\underline{\textcolor{red}{Application}}}
\noindent Calculer \( \left(\dfrac{\sqrt{6} + \sqrt{2}}{4}\right)^2 \) puis calculer \( \cos^2\left(\dfrac{\pi}{12}\right) \), enfin en déduire \( \cos\left(\dfrac{\pi}{12}\right) \).\\
\section*{\underline{\textcolor{red}{Correction}}}

\[
\begin{aligned}
\left(\frac{\sqrt{6} + \sqrt{2}}{4}\right)^2 &= \frac{6 + 2 + 2\sqrt{12}}{16} \\
&= \frac{2 + \sqrt{3}}{4}
\end{aligned}
\]
\vspace{1cm}
\[
\begin{aligned}
\cos^2\left(\frac{\pi}{12}\right) 
&= \frac{1 + \cos\left(2 \times \frac{\pi}{12}\right)}{2} \\
&= \frac{1 + \cos\left(\frac{\pi}{6}\right)}{2} \\
&= \frac{1 + \frac{\sqrt{3}}{2}}{2} \\
&= \frac{2 + \sqrt{3}}{4}
\end{aligned}
\]
\vspace{1cm}
\[
\cos^2\left(\frac{\pi}{12}\right)
= \left( \frac{\sqrt{6} + \sqrt{2}}{4} \right)^2
\Rightarrow \cos\left(\frac{\pi}{12}\right) = \frac{\sqrt{6} + \sqrt{2}}{4}
\]

\[
\boxed{ \cos\left( \frac{\pi}{12} \right) = \frac{\sqrt{6} + \sqrt{2}}{4} }
\]

\section*{\textcolor{red}{\underline{3) Transformation d’un produit en une somme et d'une somme en produit}}}

\(
\begin{aligned}
\text{1)}&\cos(a - b) = \cos a \cos b + \sin a \sin b \\
\text{2)}&\cos(a + b) = \cos a \cos b - \sin a \sin b \\
\text{(1)} + \text{(2)}&\Rightarrow \cos(a - b) + \cos(a + b) = 2\cos a \cos b \textcolor{red}{\iff } \cos a \cos b =\frac{1}{2}\left[\cos(a - b) + \cos(a + b)\right] \\
\text{(1)} - \text{(2)}&\Rightarrow \cos(a - b) - \cos(a + b) = 2\sin a \sin b \textcolor{red}{\iff}\sin a \sin b=\frac{1}{2}\left[\cos(a - b) - \cos(a + b)\right] \\
\text{3)}&\sin(a + b) = \sin a \cos b + \cos a \sin b \\
\text{4)}&\sin(a - b) = \sin a \cos b - \cos a \sin b \\
\text{(3)} + \text{(4)}&\Rightarrow \sin(a + b) + \sin(a - b) = 2\sin a \cos b \textcolor{red}{\iff}\sin a \cos b=\frac{1}{2}\left[\sin(a + b) + \sin(a - b)\right]
\end{aligned}
\)








\medskip

\noindent\textbf{\textcolor{red}{Posons}} 
\[
\left\{
\begin{array}{l}
p = a + b \\
q = a - b
\end{array}
\right.
\Rightarrow
\left\{
\begin{array}{l}
a = \dfrac{p + q}{2} \\
b = \dfrac{p - q}{2}
\end{array}
\right.
\]

\end{document}
