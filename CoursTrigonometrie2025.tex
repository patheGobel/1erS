\documentclass[a4paper,12pt]{article}
\usepackage{amsmath, amssymb}
\usepackage{xcolor}
\usepackage[utf8]{inputenc} % pour pdflatex
\usepackage[T1]{fontenc}    % améliore le rendu des accents
\usepackage[french]{babel}
\usepackage{tikz}
\usepackage{yhmath}
\usepackage{tcolorbox}
\begin{document}

\section*{\underline{\textbf{\textcolor{red}{L'outil métrique trigonométrique}}}}

\subsection*{\textcolor{red}{A) Angles orientés}}

\subsection*{\textcolor{red}{I. Cercle trigonométrique, radian}}

\subsubsection*{\textcolor{red}{1) Cercle trigonométrique}}

\underline{\textbf{\textcolor{red}{Définition 1}}}

\textcolor{red}{Dans un repère orthonormé} \( (O, \vec{i}, \vec{j}) \) on appelle \textcolor{red}{cercle trigonométrique} le cercle orienté \( C(0,1) \), de centre \( O \) et de rayon 1. Sur ce cercle on définit une origine \( I \) et deux sens.


    \begin{itemize}
        \item[\textcolor{red}{-}] Le sens direct, ou sens trigonométrique (positif), est le sens contraire du déplacement des aiguilles d'une montre.
        \item[\textcolor{red}{-}] Le sens indirect, ou sens antitrigonométrique (négatif), est le sens du déplacement des aiguilles d'une montre.
    \end{itemize}

\textcolor{red}{Remarque 1:} 

Puisque le rayon d'un cercle trigonométrique est 1, son périmètre est donné par \( P = 2\pi \) et l'aire de son disque est \( A = \pi \).

\begin{center}
\begin{tikzpicture}[scale=3]
    % Cercle
    \draw[gray] (0,0) circle(1);

    % Axes
    \draw[->] (-1.2,0) -- (1.5,0) node[right]{axe des abscisses};
    \draw[->] (0,-0.2) -- (0,1.5) node[above]{axe des ordonnées};

    % Points I, J, O
    \fill (1,0) circle(0.02) node[below right] {I};
    \fill (0,1) circle(0.02) node[above left] {J};
    \fill (0,0) circle(0.02) node[below left] {O};

    % Rayon OM (pi/4)
    \draw[thick] (0,0) -- ({cos(45)}, {sin(45)}) node[above right] {M};
    \fill ({cos(45)}, {sin(45)}) circle(0.02);

    % Angle alpha
    \draw[red, thick] (0.3,0) arc (0:45:0.3);
    \node at (0.35,0.13) {\textcolor{red}{$\alpha$}};

    % Arc de cercle IM
    \draw[red, thick] (1,0) arc (0:45:1);

    % Marquage de l'angle π
     \node at (1.2,0.5) {\textcolor{red}{$\wideparen{IM}$}};
\end{tikzpicture}

\vspace{0.3em}
$OI = 1 = OJ = R$
\end{center}
\subsubsection*{\textcolor{red}{2) Le radian}}

\underline{\textbf{\textcolor{red}{Définition 2}}}

\vspace{1em}

La mesure d’un angle \( \widehat{IOM} \) est de \textcolor{red}{1 radian} lorsque la mesure de l’arc du cercle trigonométrique qu’il intercepte est de \textcolor{red}{1 rayon}.

\vspace{0.5em}

Il existe une correspondance entre une mesure d’angle \( a \) \textcolor{red}{en degré} et \( b \) \textcolor{red}{en radian}, donnée par :
\[
\boxed{b = \dfrac{a\pi}{180}} \quad \text{ou encore} \quad \boxed{a = \dfrac{180b}{\pi}}
\]

\subsection*{\underline{\textcolor{red}{II. Angle orienté d’un couple de vecteurs}}}

\subsubsection*{\underline{\textcolor{red}{1) Angle géométrique - Angles orientés}}}

\textbf{\underline{\textcolor{red}{Définition 1 :}}}

Dans un repère \( (O, I, J) \), on considère deux vecteurs \( \vec{u} \) et \( \vec{v} \) non nuls.\\
Soient \( A \) et \( B \) deux points du plan tels que \( \vec{u} = \overrightarrow{OA} \) et \( \vec{v} = \overrightarrow{OB} \). Les deux angles \( \widehat{AOB} \) et \( \widehat{BOA} \) sont des angles géométriques de même mesure \textcolor{red}{toujours positive} : 
\(\textcolor{red}{\text{mes } \widehat{AOB} = \text{mes } \widehat{BOA}}\)

L’angle \( (\vec{u}, \vec{v}) \) formé par les deux vecteurs \( \overrightarrow{OA} \) et \( \overrightarrow{OB} \) est un angle orienté (on tourne de \( \vec{u} \) vers \( \vec{v} \)). Alors que l’angle \( (\vec{v}, \vec{u}) \) est un angle orienté mais de sens contraire. Donc : \( \textcolor{red}{ (\vec{v}, \vec{u}) = -(\vec{u}, \vec{v})}\)
\begin{tcolorbox}[colback=red!5!white, colframe=red!75!black, title=Propriété 1]
\( (\vec{v}, \vec{u}) = -(\vec{u}, \vec{v}) \)
\end{tcolorbox}
\vspace{1em}

\begin{center}
\begin{tikzpicture}[scale=1.5]
    \coordinate (O) at (0,0);
    \coordinate (A) at (3,0);
    \coordinate (B) at (1.5,1.5);

    % Vecteurs
    \draw[->, thick] (O) -- (A) node[below right] {A};
    \draw[->, thick] (O) -- (B) node[above right] {B};

    % Points
    \fill (O) circle(0.03) node[below left] {O};

    % Noms des vecteurs
    \node at (1.5,-0.2) {\( \vec{u} \)};
    \node at (0.6,1.2) {\( \vec{v} \)};

    % Arc orienté de u vers v
    \draw[red, ->, thick] (1.2,0) arc[start angle=0, end angle=45, radius=1.2];
    \node[red] at (2,0.4) {\small sens de $\vec{u} \rightarrow \vec{v}$};

\end{tikzpicture}
\end{center}


\vspace{1em}

\noindent
\textcolor{red}{\underline{Définition 2 :}} \

\textcolor{red}{\underline{Théorème :}}

\vspace{0.5em}

Soit \( M \) un point quelconque du cercle trigonométrique tel que 
\[
(\overrightarrow{OI}, \overrightarrow{OM}) = \alpha \text{ rad}.
\]
On peut lui associer une famille de nombres réels de la forme 
\[
\alpha + 2k\pi, \quad k \in \mathbb{Z}
\]
qui correspondent au même point \( M \) du cercle trigonométrique.

\vspace{1em}
\noindent\textbf{\underline{\textcolor{red}{Définition 2:}}}
 
Dans le plan muni de \( \mathcal{R}ON(O, \vec{i}, \vec{j}) \), on considère les vecteurs \( \vec{u} \) et \( \vec{v} \) non nuls tel que la mesure de l’angle \( (\vec{u}, \vec{v}) = \alpha \, \text{rad} \).\\
Alors chacun des nombres associés à \( \alpha \) de la forme \( \alpha + 2k\pi \), \( k \in \mathbb{Z} \), est appelé une mesure de l’angle orienté de vecteurs \( (\vec{u}, \vec{v}) \).
\subsection*{\underline{\textcolor{red}{2) Mesure principale d’un angle orienté}}}

\textbf{\underline{\textcolor{red}{Définition 3:}}}

Dans un repère orthonormé \( (O, \vec{i}, \vec{j}) \), on considère deux vecteurs \( \vec{u} \) et \( \vec{v} \) non nuls tel que \( (\vec{u}, \vec{v}) = \alpha \) rad.\\
Parmi toutes les mesures d’angles associées à \( \alpha \), de la forme \( \alpha + 2k\pi \), \( k \in \mathbb{Z} \), il existe une et une seule mesure qui appartient à l’intervalle \( ]-\pi, \pi] \).\\
Cette mesure particulière s’appelle la \textbf{mesure principale} de l’angle \( (\vec{u}, \vec{v}) \).

\vspace{1em}

\textcolor{red}{\textbf{Exemple :}} 

Dans chacun des cas suivants, détermine la mesure \textbf{principale} de l’angle \( (\vec{u}, \vec{v}) \) :

\[
\alpha = \dfrac{89\pi}{7} \qquad \beta = \dfrac{229\pi}{12}
\]

\end{document}
