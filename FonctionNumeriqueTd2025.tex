\documentclass{article}
\usepackage{amsmath, amssymb}
\usepackage{geometry}
\usepackage{multicol}
\usepackage{fancybox}
\usepackage{relsize}
\usepackage{setspace}
\usepackage{graphicx}

\geometry{a4paper, margin=1in}

\begin{document}

\fbox{\textbf{Exercice 1}} \quad \textbf{Déterminer le domaine de définition des fonctions suivantes :}

\bigskip

\noindent
\begin{multicols}{3}
\smaller
\setstretch{2.5} % Augmente l'espacement entre les lignes
\begin{enumerate}
    \item $f(x) = \dfrac{2x^2 - 5x + 15}{8x^2 - 5x - 3}$
    \item $f(x) = \dfrac{4x^2 - 5x + 15}{x^3 + 6x}$
    \item $f(x) = \sqrt{x^2 - 3x - 18}$
    \item $f(x) = \sqrt{4x - x^3}$
    \item $f(x) = \sqrt{\dfrac{x^2 + x - 6}{-x^2 + x - 1}}$
    \item $f(x) = \dfrac{\sqrt{-4x+1}}{x+2}$
    \item $f(x) = \dfrac{x^2 + 5x - 7}{\sqrt{2x^2 + 3x - 2}}$
    \item $f(x) = \dfrac{\sqrt{x^2 - x}}{(x-2)(x+3)}$
    \item $f(x) = \sqrt{\dfrac{(1-x)(2+x)}{x^2 + x}}$
    \item $f(x) = \sqrt{|x-1|} - 2$
    \item $f(x) = \dfrac{\sqrt{x-1}}{|2x-5| -2}$
\end{enumerate}

\columnbreak

\begin{enumerate}
    \setcounter{enumi}{11}
    \item $f(x) = \sqrt{|1 - 3x| - x + 2}$
    \item $f(x) = \dfrac{4x^2 -1}{6x^2 - |13x - 5|}$
    \item $f(x) = \dfrac{\sqrt{2-3x+1}}{|x| - 1}$
    \item $f(x) = \dfrac{x^2 + 2x}{\sqrt{x+3} - \sqrt{2x-1}}$
    \item $f(x) = \dfrac{1}{|x| + x}$
    \item $f(x) = \sqrt{3 - |2x + 5|}$
    \item $f(x) = \sqrt{2x + 1} - \sqrt{7 - 6x}$
    \item $f(x) = \dfrac{x^2 + 1}{x - E(x)}$
    \item $\left\{
        \begin{array}{ll}
            f(x) = \dfrac{x}{x^2 - 1}, & \text{si } x \leq 0 \\
            f(x) = \sqrt{x - 1}, & \text{si } x > 0
        \end{array}
    \right.$
    
    \item $\left\{
        \begin{array}{ll}
            f(x) = \sqrt{x^2 - 2x} + x, & \text{si } x > 0 \\
            f(x) = \dfrac{x^2}{x+1}, & \text{si } x < 0
        \end{array}
    \right.$
\end{enumerate}

\columnbreak

\begin{enumerate}
    \setcounter{enumi}{21}
    \item $\left\{
        \begin{array}{ll}
            f(x) = |x + 1| - \dfrac{2}{x+2}, & \text{si } x < 0 \\
            f(x) = x^2 \sqrt{4 - x}, & \text{si } x \geq 0
        \end{array}
    \right.$
    
    \item $\left\{
        \begin{array}{ll}
            f(x) = x \sqrt{\left| \dfrac{x+1}{x} \right|}, & \text{si } x < 0 \\
            f(x) = \dfrac{x^3 - x^2}{x^2 + 1}, & \text{si } x \geq 0
        \end{array}
    \right.$
    
    \item $\left\{
        \begin{array}{ll}
            f(x) = x\sqrt{-x} + 1, & \text{si } x \leq 0 \\
            f(x) = \dfrac{1 - x}{1 + x^3}, & \text{si } x > 0
        \end{array}
    \right.$
    
    \item $\left\{
        \begin{array}{ll}
            f(x) = \dfrac{x(x-2)}{x-1}, & \text{si } x < 0 \\
            f(x) = x + \sqrt{x^2 - 4}, & \text{si } x \geq 0
        \end{array}
    \right.$
    
    \item $\left\{
        \begin{array}{ll}
            f(x) = \dfrac{3}{|x+1| - 2}, & \text{si } x \leq 1 \\
            f(x) = \sqrt{x - 3}, & \text{si } x > 1
        \end{array}
    \right.$
\end{enumerate}
\end{multicols}

\fbox{\textbf{Exercice 2}} \quad \textbf{Dans chacun des cas suivants, dites si $f$ est une application}

\bigskip

\noindent
\begin{multicols}{3}
\noindent
\renewcommand{\arraystretch}{1.3} % Augmente l'espacement vertical entre les lignes
\begin{tabular}{rcl}
    $f : \mathbb{R}$ & $\to$ & $\mathbb{R}$ \\
    $x$ & $\mapsto$ & $\dfrac{x^2 - 1}{x^2 + 1}$
\end{tabular}

\bigskip

\begin{tabular}{rcl}
    $f : \mathbb{R}$ & $\to$ & $\mathbb{R}$ \\
    $x$ & $\mapsto$ & $\sqrt{x^2 - 1} - 1$
\end{tabular}

\columnbreak

\begin{tabular}{rcl}
    $f : \mathbb{R}$ & $\to$ & $\mathbb{R}$ \\
    $x$ & $\mapsto$ & $\dfrac{x}{\sqrt{x^2+1}}$
\end{tabular}

\bigskip

\begin{tabular}{rcl}
    $f : [0; +\infty[$ & $\to$ & $\mathbb{R}$ \\
    $x$ & $\mapsto$ & $\dfrac{x}{\sqrt{x^2+1}}$
\end{tabular}

\columnbreak

\begin{tabular}{rcl}
    $f : \mathbb{R}$ & $\to$ & $\mathbb{R}$ \\
    $x$ & $\mapsto$ & $x - \sqrt{x}$
\end{tabular}

\bigskip

\begin{tabular}{rcl}
    $f : [2; +\infty[$ & $\to$ & $\mathbb{R}$ \\
    $x$ & $\mapsto$ & $\sqrt{x - 2}$
\end{tabular}

\bigskip

\begin{tabular}{rcl}
    $f : \mathbb{R}$ & $\to$ & $\mathbb{Z}$ \\
    $x$ & $\mapsto$ & $E(x)$
\end{tabular}

\end{multicols}

\fbox{\textbf{Exercice 3}} \quad \textbf{Égalité de deux fonctions}

\begin{enumerate}
    \item  Soit les fonctions \( f(x) = \dfrac{x^3 - x^2 + 2x - 2}{x^2 + 2} \) et \( g(x) = x - 1 \).

\begin{enumerate}
    \item Vérifier que :
\( (x^2 + 2)(x - 1) = x^3 - x^2 + 2x - 2 \)
    \item Montrer que \( f \) et \( g \) sont égales.
\end{enumerate}

\item Soient \( f(x) = 2x + 1 + \dfrac{20}{x - 1} \) et
\( g(x) = \dfrac{2x^2 - x + 1}{x - 1}. \)

\begin{enumerate}
\item Montrer que les fonctions \( f \) et \( g \) sont égales.
\item Dans chacun des cas suivants, déterminer les réels \( a \) et \( b \) pour que les fonctions \( f \) et \( g \) soient égales.
\( f(x) = -2x^2 - 12x - 16 \) et \( g(x) = -2(x - a)^2 + b \).

\( f(x) = \dfrac{x + 7}{x^2 + 2x - 3} \) et \( g(x) = \dfrac{a}{x - 1} + \dfrac{b}{x + 3} \).
\end{enumerate}
\end{enumerate}

\fbox{\textbf{Exercice 4}} \quad Soit \( f \) la fonction définie de \( \mathbb{R} \) vers \( \mathbb{R} \) par :  
\( f(x) = |x - 1| + 2 |3 - x| \).  

Déterminer l’application affine \( g \) qui a même restriction que \( f \) sur l’intervalle \([1;3]\).

\bigskip

\fbox{\textbf{Exercice 5}} \quad Soient \( f \) et \( g \) des fonctions définies par  
\( f(x) = \dfrac{x - 1}{x - 2} \), \( g(x) = \dfrac{2x - 1}{x - 1} \).

\begin{enumerate}
    \item Déterminer le domaine de définition et l’expression des fonctions suivantes : \( f + g \), \( fg \), \( \dfrac{f}{g} \) et \( 3f - 2g \).
    \item Déterminer le domaine de définition de \( f \circ g \). et \( g \circ f \) puis comparer \( f \circ g \) et \( g \circ f \) ; Que peut-on en déduire ?
\end{enumerate}

\fbox{\textbf{Exercice 6}} \quad \textbf{Composition et décomposition}

\bigskip

\noindent
\textbf{1.} On considère les fonctions suivantes :  
\( f(x) = \sqrt{x + 3} \) et \( g(x) = \dfrac{1}{x^2 - 4} \).

\begin{enumerate}
    \item Déterminer \( D_f \), \( D_g \), \( D_{fog} \) et \( D_{gof} \).
    \item Calculer \( f \circ g(x) \) et \( g \circ f(x) \).
\end{enumerate}

\bigskip

\noindent
\textbf{2.} Trouver deux fonctions \( f \) et \( g \) telles que  
\( h(x) = (f \circ g)(x) \).

\bigskip

\noindent
\(
\begin{array}{ll}
    \text{a)} \quad h(x) = \sqrt{x^2 + 3x + 1} & \quad
    \text{d)} \quad h(x) = \dfrac{x^2 + 7}{x^2 - 2} \\
    \text{b)} \quad h(x) = (3x + 1)^2 & \quad
    \text{e)} \quad h(x) = (x + 5)^2 + 4 \\
    \text{c)} \quad h(x) = \dfrac{3}{x^2 - 5x + 6} & \quad
    \text{f)} \quad h(x) = \dfrac{1}{x - 1}
\end{array}
\)

\fbox{\textbf{Exercice 7}} \quad \textbf{Étude de la parité d’une fonction}

\bigskip

Étudier la parité des fonctions suivantes :

\begin{enumerate}
    \item \( f(x) = 2^x + x^2 - 1 \);
    \item \( f(x) = x^3 + x \);
    \item \( f(x) = x^2 - 3|x| + 1 \);
    \item \( f(x) = \dfrac{|x|}{x^2 + 1} \);
    \item \( f(x) = \dfrac{x^3}{|x^4 - x^2 + 1|} \);
    \item \( f(x) = \dfrac{\sqrt{x^2 - 16}}{\sqrt{4 + x^2}} \);
    \item \( f(x) = \dfrac{\sin x}{2 + \sin^2 x} \);
    \item \( f(x) = \dfrac{|1 + x| - |1 - x|}{|1 + x| + |1 - x|} \);
    \item \( f(x) = \dfrac{x + 1}{1 - x^2} \).
\end{enumerate}

\fbox{\textbf{Exercice 8}}

\bigskip

\textbf{1.} Dans chacun des cas suivants, étudier la périodicité des fonctions numériques et préciser et d’en préciser la période.

\begin{itemize}
    \item[a)] \( f(x) = \sin x \);
    \item[b)] \( f(x) = \cos x \);
    \item[c)] \( f(x) = \dfrac{\sin x}{\cos x} \);
    \item[d)] \( f(x) = \cos 2x \sin 3x \);
    \item[e)] \( f(x) = x - E(x) \);
    \item[f)] \( f(x) = [2x - E(2x)] \sin 3\pi x \);
    \item[g)] \( f(x) = \dfrac{\cos \pi x}{x - E(x)} \).
\end{itemize}

\bigskip

\textbf{2.} Soit la fonction numérique \( f \) telle que :

\( f(x + 2) = \dfrac{1 + f(x)}{1 - f(x)}. \)

Calculer \( f(x + 4) \), \( f(x + 6) \), \( f(x + 8) \) en fonction de \( f(x) \).  
Quelle conclusion peut-on en tirer ?

\fbox{\textbf{Exercice 9}} \quad Étudier le sens de variations de \( f \).

\bigskip

\noindent
\begin{enumerate}
    \item \( f(x) = 2x^2 + 3x + 1 \) \quad ; \quad \( f(x) = \sqrt{x^2 - 1} \).
    \item \( f(x) = \dfrac{2x - 1}{x + 1} \) \quad ; \quad \( f(x) = \sqrt{x^3 + x} \).
    \item \( f(x) = x^3 - 3x \) \quad ; \quad \( f(x) = \sqrt{x^3 - x} \).
    \item \( f(x) = |x + 1| - |x - 1| + |2x| \).
\end{enumerate}

\bigskip

\fbox{\textbf{Exercice 10}} \quad \textbf{Élément de symétrie}

\bigskip

\begin{enumerate}
    \item  Dans chacun des cas, montrer que \( (C_f) \) admet la droite \( (\Delta) \) pour axe de symétrie.

\begin{enumerate}
    \item \( f(x) = x^2 + 2x - 3 \) et \( (\Delta) : x = -1 \).
    \item \( f(x) = -3x^2 + 4x + 1 \) et \( (\Delta) : x = \dfrac{2}{3} \).
    \item \( f(x) = \dfrac{-2x^2 + 4x - 1}{(x - 1)^2} \) et \( (\Delta) : x = 1 \).
    \item \( f(x) = \dfrac{x^2 + 4x + 3}{2x^2 + 8x + 9} \) et \( (\Delta) : x = -2 \).
\end{enumerate}

\item  Dans chacun des cas suivants, montrer que \( (C_f) \) admet le point \( K \) pour centre de symétrie.

\begin{enumerate}
    \item \( f(x) = \dfrac{x - 4}{x - 2} \) et \( K(2;1) \).
    \item \( f(x) = -x^3 + 3x + 4 \) et \( K(0;4) \).
    \item \( f(x) = \dfrac{x^2 - 5x + 7}{x - 2} \) et \( K(2;1) \).
    \item \( f(x) = \dfrac{x^3 - x^2 - x}{2x^2 - 4 + 1} \) et \( K(1;1) \).
    \item \( f(x) = \dfrac{1}{x+3} + \dfrac{1}{x+1} \) et \( K(-2;0) \).
\end{enumerate}

\end{enumerate}

\fbox{\textbf{Exercice 11}}

\bigskip

On considère la fonction \( f \) définie par :  
\( f(x) = x(1 - x) \) sur \( \mathbb{R} \).

\begin{enumerate}
    \item Démontrer que \( f(x) \leq \dfrac{1}{4} \) pour tout \( x \in \mathbb{R} \).
    \item En déduire que la fonction \( f \) admet un maximum en \( x = \dfrac{1}{2} \).
    \item Démontrer que \( f(x) = \dfrac{1}{4} - \left( x - \dfrac{1}{2} \right)^2 \).
\end{enumerate}

\fbox{\textbf{Exercice 12}}

\bigskip

Soit l’application \( f \) définie par sa représentation graphique ci-dessous

\bigskip

\begin{center}
    \includegraphics[width=0.6\textwidth]{graph1.png} % Remplacez "graph.png" par le nom de votre image
\end{center}

\bigskip

\begin{enumerate}
    \item Trouver l’image directe par \( f \) des intervalles suivants : \([ -3; -1 ]\) ; \( ] -1; 4] \) et \(\{ -3; -1 \}\).
    \item Trouver l’image réciproque par \( f \) des intervalles suivants : \( ]1;3[ \) ; \( ] -\infty; -3] \) et \([ -2; 3 ]\).
    \item Donner les formules explicites de \( f(x) \).
    \item Montrer que \( f \) est bijective.
\end{enumerate}

\fbox{\textbf{Exercice 13}}

\bigskip

Soient les fonctions \( f \) et \( g \) définies respectivement par :

\renewcommand{\arraystretch}{1.3} % Augmente l'espacement vertical entre les lignes
\begin{tabular}{rcl}
    $f : \mathbb{R} - \{1\}$ & $\to$ & $\mathbb{R} - \{1\}$ \\
    $x$ & $\mapsto$ & $\dfrac{x}{x - 1}$
\end{tabular}

\bigskip

\begin{tabular}{rcl}
    $g : \mathbb{R} - \{1\}$ & $\to$ & $\mathbb{R}^{*}_{+}$ \\
    $x$ & $\mapsto$ & $3x - 2$
\end{tabular}

\bigskip

\begin{enumerate}
    \item Les fonctions \( f \) et \( g \) sont-elles bijectives ?
    \item Calculer : \( f \circ g(0) \), \( g \circ f(0) \), \( f \circ g(x) \) et \( g \circ f(x) \).
    \item Trouver \( f([-2; -1]) \) ; \( g([-2;0]) \) ; \( f^{-1}([2;3[) \) et \( f^{-1}(] - \infty; 0[) \).
    \item Résoudre l’équation \( g(x) = g(x) \).
    \item Sur quel intervalle \( I \), \( f \) et \( g \) se rencontrent-elles ?
\end{enumerate}

\fbox{\textbf{Exercice 14}}

\bigskip

La courbe tracée ci-dessous est la représentation graphique d’une fonction \( f \) définie sur \( \mathbb{R} \) par :  
\[
f(x) = 3x^2 - x^3.
\]

\bigskip

\begin{center}
    \includegraphics[width=0.6\textwidth]{graph2.png} % Remplacez "graph.png" par le nom de votre image
\end{center}

\bigskip

\begin{enumerate}
    \item Dessiner la courbe représentative de la fonction \( g \) définie par : \( g(x) = f(x) - 1 \).
    
    \item Soit \( h \) la fonction définie par : \( h(x) = f(x-2) \).
    \begin{enumerate}
        \item Dessiner la courbe représentative de la fonction \( h \).
        \item Donner l’expression de \( h(x) \).
        \item Établir le tableau de variation de la fonction \( h \).
    \end{enumerate}
    
    \item Soit \( z \) la fonction définie par : \( z(x) = |f(x)| \).
    \begin{enumerate}
        \item Donner l’expression de \( z(x) \).
        \item Établir le tableau de variation de la fonction \( z \).
    \end{enumerate}
\end{enumerate}

\fbox{\textbf{Exercice 15}}

\bigskip

Dans chacun des cas suivants, dire si l’application \( f \) est injective, surjective ou bijective.

\bigskip

\renewcommand{\arraystretch}{1.3} % Augmente l'espacement vertical entre les lignes
\begin{tabular}{rcl}
    \( f : \mathbb{R} - \{2\} \) & \( \to \) & \( \mathbb{R} - \{2\} \) \\
    \( x \) & \( \mapsto \) & \( \dfrac{2x+1}{2x-4} \)
\end{tabular}

\bigskip

\renewcommand{\arraystretch}{1.3}
\begin{tabular}{rcl}
    \( f : [\frac{1}{2}; +\infty[ \) & \( \to \) & \( \mathbb{R} \) \\
    \( x \) & \( \mapsto \) & \( \sqrt{2x - 1} \)
\end{tabular}

\bigskip

\renewcommand{\arraystretch}{1.3}
\begin{tabular}{rcl}
    \( f : \mathbb{R} \) & \( \to \) & \( \mathbb{R} \) \\
    \( x \) & \( \mapsto \) & \( x^2 - 2x - 3 \)
\end{tabular}

\bigskip

\renewcommand{\arraystretch}{1.3}
\begin{tabular}{rcl}
    \( f : \mathbb{N} \) & \( \to \) & \( \mathbb{Z} \) \\
    \( x \) & \( \mapsto \) & \( 2x - 3 \)
\end{tabular}

\bigskip

\fbox{\textbf{Exercice 16}}

\bigskip

Soit la correspondance :

\bigskip

\renewcommand{\arraystretch}{1.3}
\begin{tabular}{rcl}
    \( f : \mathbb{R} \) & \( \to \) & \( \mathbb{R} \) \\
    \( x \) & \( \mapsto \) & \( \sqrt{|1 - x^2|} \)
\end{tabular}

\bigskip

\begin{enumerate}
    \item Justifier que \( f \) est une application.
    \item Soit \( g \) la restriction de \( f \) sur \([1; +\infty[\)
    \begin{enumerate}
        \item Montrer que \( g(x) = \sqrt{x^2 - 1} \).
        \item Déterminer l’image directe par \( g \) de \( A = \{1;2;3\} \).
        \item Déterminer l’image réciproque par \( g \) de \( B = ]1;4] \).
        \item  Montrer que \( g \) est une bijection de \([1; +\infty[\) vers un intervalle \( J \) à préciser.
        \item  Déterminer \( g^{-1}(x) \).
    \end{enumerate}
    \item On définit la fonction \( h \) par \( h(x) = \dfrac{x+1}{x-3} \).
    \begin{enumerate}
        \item Déterminer \( D_{g \circ h} \) et \( D_{h \circ g} \).
        \item Expliciter \( g \circ h(x) \).
    \end{enumerate}
\end{enumerate}

\fbox{\textbf{Exercice 18}}

\bigskip

Soit \( f \) et \( g \) les fonctions définies par :  

\( f(x) = -5x + 3 \quad \text{et} \quad g(x) = \dfrac{4x^2 + 1}{x^2 + 2}. \)

\begin{enumerate}
    \item Démontrer que \( \forall x \in \mathbb{R} \), on a :  
\( \frac{1}{2} \leq g(x) \leq 4. \)
    \item Démontrer que \( \forall x \in \mathbb{R} \), on a :  
    \(  -17 \leq f \circ g(x) \leq \frac{1}{2}. \)
\end{enumerate}

\bigskip

\fbox{\textbf{Exercice 19}}

\bigskip

Soient \( E \), \( F \) et \( G \) des ensembles non vides. On considère les applications suivantes  
\( f : E \to F \) et \( g : F \to G \).

\begin{enumerate}
    \item Montrer que \( g \circ f \) injective implique que \( f \) est injective.
    \item Montrer que \( g \circ g \) injective et \( f \) surjective implique que \( g \) est injective.
    \item Montrer que \( g \circ f \) surjective sur \( G \) implique que \( g \) est surjective sur \( G \).
    \item Montrer que \( g \circ f \) surjective sur \( G \) et \( g \) injective implique que \( f \) est surjective sur \( F \).
\end{enumerate}

\end{document}
