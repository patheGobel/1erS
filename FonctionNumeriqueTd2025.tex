\documentclass{article}
\usepackage{amsmath, amssymb}
\usepackage{geometry}
\usepackage{multicol}
\usepackage{fancybox}
\usepackage{relsize}
\usepackage{setspace}

\geometry{a4paper, margin=1in}

\begin{document}

\fbox{\textbf{Exercice 1}} \quad \textbf{Déterminer le domaine de définition des fonctions suivantes :}

\bigskip

\noindent
\begin{multicols}{3}
\smaller
\setstretch{2.5} % Augmente l'espacement entre les lignes
\begin{enumerate}
    \item $f(x) = \dfrac{2x^2 - 5x + 15}{8x^2 - 5x - 3}$
    \item $f(x) = \dfrac{4x^2 - 5x + 15}{x^3 + 6x}$
    \item $f(x) = \sqrt{x^2 - 3x - 18}$
    \item $f(x) = \sqrt{4x - x^3}$
    \item $f(x) = \sqrt{\dfrac{x^2 + x - 6}{-x^2 + x - 1}}$
    \item $f(x) = \dfrac{\sqrt{-4x+1}}{x+2}$
    \item $f(x) = \dfrac{x^2 + 5x - 7}{\sqrt{2x^2 + 3x - 2}}$
    \item $f(x) = \dfrac{\sqrt{x^2 - x}}{(x-2)(x+3)}$
    \item $f(x) = \sqrt{\dfrac{(1-x)(2+x)}{x^2 + x}}$
    \item $f(x) = \sqrt{|x-1|} - 2$
    \item $f(x) = \dfrac{\sqrt{x-1}}{|2x-5| -2}$
\end{enumerate}

\columnbreak

\begin{enumerate}
    \setcounter{enumi}{11}
    \item $f(x) = \sqrt{|1 - 3x| - x + 2}$
    \item $f(x) = \dfrac{4x^2 -1}{6x^2 - |13x - 5|}$
    \item $f(x) = \dfrac{\sqrt{2-3x+1}}{|x| - 1}$
    \item $f(x) = \dfrac{x^2 + 2x}{\sqrt{x+3} - \sqrt{2x-1}}$
    \item $f(x) = \dfrac{1}{|x| + x}$
    \item $f(x) = \sqrt{3 - |2x + 5|}$
    \item $f(x) = \sqrt{2x + 1} - \sqrt{7 - 6x}$
    \item $f(x) = \dfrac{x^2 + 1}{x - E(x)}$
    \item $\left\{
        \begin{array}{ll}
            f(x) = \dfrac{x}{x^2 - 1}, & \text{si } x \leq 0 \\
            f(x) = \sqrt{x - 1}, & \text{si } x > 0
        \end{array}
    \right.$
    
    \item $\left\{
        \begin{array}{ll}
            f(x) = \sqrt{x^2 - 2x} + x, & \text{si } x > 0 \\
            f(x) = \dfrac{x^2}{x+1}, & \text{si } x < 0
        \end{array}
    \right.$
\end{enumerate}

\columnbreak

\begin{enumerate}
    \setcounter{enumi}{21}
    \item $\left\{
        \begin{array}{ll}
            f(x) = |x + 1| - \dfrac{2}{x+2}, & \text{si } x < 0 \\
            f(x) = x^2 \sqrt{4 - x}, & \text{si } x \geq 0
        \end{array}
    \right.$
    
    \item $\left\{
        \begin{array}{ll}
            f(x) = x \sqrt{\left| \dfrac{x+1}{x} \right|}, & \text{si } x < 0 \\
            f(x) = \dfrac{x^3 - x^2}{x^2 + 1}, & \text{si } x \geq 0
        \end{array}
    \right.$
    
    \item $\left\{
        \begin{array}{ll}
            f(x) = x\sqrt{-x} + 1, & \text{si } x \leq 0 \\
            f(x) = \dfrac{1 - x}{1 + x^3}, & \text{si } x > 0
        \end{array}
    \right.$
    
    \item $\left\{
        \begin{array}{ll}
            f(x) = \dfrac{x(x-2)}{x-1}, & \text{si } x < 0 \\
            f(x) = x + \sqrt{x^2 - 4}, & \text{si } x \geq 0
        \end{array}
    \right.$
    
    \item $\left\{
        \begin{array}{ll}
            f(x) = \dfrac{3}{|x+1| - 2}, & \text{si } x \leq 1 \\
            f(x) = \sqrt{x - 3}, & \text{si } x > 1
        \end{array}
    \right.$
\end{enumerate}
\end{multicols}

\fbox{\textbf{Exercice 2}} \quad \textbf{Dans chacun des cas suivants, dites si $f$ est une application}

\bigskip

\noindent
\begin{multicols}{3}
\noindent
\renewcommand{\arraystretch}{1.3} % Augmente l'espacement vertical entre les lignes
\begin{tabular}{rcl}
    $f : \mathbb{R}$ & $\to$ & $\mathbb{R}$ \\
    $x$ & $\mapsto$ & $\dfrac{x^2 - 1}{x^2 + 1}$
\end{tabular}

\bigskip

\begin{tabular}{rcl}
    $f : \mathbb{R}$ & $\to$ & $\mathbb{R}$ \\
    $x$ & $\mapsto$ & $\sqrt{x^2 - 1} - 1$
\end{tabular}

\columnbreak

\begin{tabular}{rcl}
    $f : \mathbb{R}$ & $\to$ & $\mathbb{R}$ \\
    $x$ & $\mapsto$ & $\dfrac{x}{\sqrt{x^2+1}}$
\end{tabular}

\bigskip

\begin{tabular}{rcl}
    $f : [0; +\infty[$ & $\to$ & $\mathbb{R}$ \\
    $x$ & $\mapsto$ & $\dfrac{x}{\sqrt{x^2+1}}$
\end{tabular}

\columnbreak

\begin{tabular}{rcl}
    $f : \mathbb{R}$ & $\to$ & $\mathbb{R}$ \\
    $x$ & $\mapsto$ & $x - \sqrt{x}$
\end{tabular}

\bigskip

\begin{tabular}{rcl}
    $f : [2; +\infty[$ & $\to$ & $\mathbb{R}$ \\
    $x$ & $\mapsto$ & $\sqrt{x - 2}$
\end{tabular}

\bigskip

\begin{tabular}{rcl}
    $f : \mathbb{R}$ & $\to$ & $\mathbb{Z}$ \\
    $x$ & $\mapsto$ & $E(x)$
\end{tabular}

\end{multicols}

\fbox{\textbf{Exercice 3}} \quad \textbf{Égalité de deux fonctions}

\begin{enumerate}
    \item  Soit les fonctions \( f(x) = \dfrac{x^3 - x^2 + 2x - 2}{x^2 + 2} \) et \( g(x) = x - 1 \).

\begin{enumerate}
    \item Vérifier que :
\( (x^2 + 2)(x - 1) = x^3 - x^2 + 2x - 2 \)
    \item Montrer que \( f \) et \( g \) sont égales.
\end{enumerate}

\item Soient \( f(x) = 2x + 1 + \dfrac{20}{x - 1} \) et
\( g(x) = \dfrac{2x^2 - x + 1}{x - 1}. \)

\begin{enumerate}
\item Montrer que les fonctions \( f \) et \( g \) sont égales.
\item Dans chacun des cas suivants, déterminer les réels \( a \) et \( b \) pour que les fonctions \( f \) et \( g \) soient égales.
\( f(x) = -2x^2 - 12x - 16 \) et \( g(x) = -2(x - a)^2 + b \).
 \( f(x) = \dfrac{x + 7}{x^2 + 2x - 3} \) et \( g(x) = \dfrac{a}{x - 1} + \dfrac{b}{x + 3} \).
\end{enumerate}
\end{enumerate}

\fbox{\textbf{Exercice 4}} \quad Soit \( f \) la fonction définie de \( \mathbb{R} \) vers \( \mathbb{R} \) par :  
\( f(x) = |x - 1| + 2 |3 - x| \).  

Déterminer l’application affine \( g \) qui a même restriction que \( f \) sur l’intervalle \([1;3]\).

\bigskip

\fbox{\textbf{Exercice 5}} \quad Soient \( f \) et \( g \) des fonctions définies par  
\( f(x) = \dfrac{x - 1}{x - 2} \), \( g(x) = \dfrac{2x - 1}{x - 1} \).

\begin{enumerate}
    \item Déterminer le domaine de définition et l’expression des fonctions suivantes : \( f + g \), \( fg \), \( \dfrac{f}{g} \) et \( 3f - 2g \).
    \item Déterminer le domaine de définition de \( f \circ g \). et \( g \circ f \) puis comparer \( f \circ g \) et \( g \circ f \) ; Que peut-on en déduire ?
\end{enumerate}

\fbox{\textbf{Exercice 6}} \quad \textbf{Composition et décomposition}

\bigskip

\noindent
\textbf{1.} On considère les fonctions suivantes :  
\( f(x) = \sqrt{x + 3} \) et \( g(x) = \dfrac{1}{x^2 - 4} \).

\begin{enumerate}
    \item Déterminer \( D_f \), \( D_g \), \( D_{fog} \) et \( D_{gof} \).
    \item Calculer \( f \circ g(x) \) et \( g \circ f(x) \).
\end{enumerate}

\bigskip

\noindent
\textbf{2.} Trouver deux fonctions \( f \) et \( g \) telles que  
\( h(x) = (f \circ g)(x) \).

\bigskip

\noindent
\(
\begin{array}{ll}
    \text{a)} \quad h(x) = \sqrt{x^2 + 3x + 1} & \quad
    \text{d)} \quad h(x) = \dfrac{x^2 + 7}{x^2 - 2} \\
    \text{b)} \quad h(x) = (3x + 1)^2 & \quad
    \text{e)} \quad h(x) = (x + 5)^2 + 4 \\
    \text{c)} \quad h(x) = \dfrac{3}{x^2 - 5x + 6} & \quad
    \text{f)} \quad h(x) = \dfrac{1}{x - 1}
\end{array}
\)

\fbox{\textbf{Exercice 7}} \quad \textbf{Étude de la parité d’une fonction}

\bigskip

Étudier la parité des fonctions suivantes :

\begin{enumerate}
    \item \( f(x) = 2^x + x^2 - 1 \);
    \item \( f(x) = x^3 + x \);
    \item \( f(x) = x^2 - 3|x| + 1 \);
    \item \( f(x) = \dfrac{|x|}{x^2 + 1} \);
    \item \( f(x) = \dfrac{x^3}{|x^4 - x^2 + 1|} \);
    \item \( f(x) = \dfrac{\sqrt{x^2 - 16}}{\sqrt{4 + x^2}} \);
    \item \( f(x) = \dfrac{\sin x}{2 + \sin^2 x} \);
    \item \( f(x) = \dfrac{|1 + x| - |1 - x|}{|1 + x| + |1 - x|} \);
    \item \( f(x) = \dfrac{x + 1}{1 - x^2} \).
\end{enumerate}

\fbox{\textbf{Exercice 8}}

\bigskip

\textbf{1.} Dans chacun des cas suivants, étudier la périodicité des fonctions numériques et préciser et d’en préciser la période.

\begin{itemize}
    \item[a)] \( f(x) = \sin x \);
    \item[b)] \( f(x) = \cos x \);
    \item[c)] \( f(x) = \dfrac{\sin x}{\cos x} \);
    \item[d)] \( f(x) = \cos 2x \sin 3x \);
    \item[e)] \( f(x) = x - E(x) \);
    \item[f)] \( f(x) = [2x - E(2x)] \sin 3\pi x \);
    \item[g)] \( f(x) = \dfrac{\cos \pi x}{x - E(x)} \).
\end{itemize}

\bigskip

\textbf{2.} Soit la fonction numérique \( f \) telle que :
\[
f(x + 2) = \dfrac{1 + f(x)}{1 - f(x)}.
\]
Calculer \( f(x + 4) \), \( f(x + 6) \), \( f(x + 8) \) en fonction de \( f(x) \).  
Quelle conclusion peut-on en tirer ?

\end{document}
