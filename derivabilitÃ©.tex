\documentclass[12pt]{article}
\usepackage{stmaryrd}
\usepackage{graphicx}
\usepackage[utf8]{inputenc}

\usepackage[french]{babel}
\usepackage[T1]{fontenc}
\usepackage{hyperref}
\usepackage{verbatim}

\usepackage{color, soul}

\usepackage{pgfplots}
\pgfplotsset{compat=1.15}
\usepackage{mathrsfs}

\usepackage{amsmath}
\usepackage{amsfonts}
\usepackage{amssymb}
\usepackage{tkz-tab}
\author{Destinés à la 1\textsuperscript{ère}S2\\Au Lycée de Dindéferlo}
\title{\textbf{Dérivée et Application}}
\date{\today}
\usepackage{tikz}
\usetikzlibrary{arrows, shapes.geometric, fit}

% Commande pour la couleur d'accentuation
\newcommand{\myul}[2][black]{\setulcolor{#1}\ul{#2}\setulcolor{black}}
\newcommand\tab[1][1cm]{\hspace*{#1}}

\begin{document}
\maketitle
\newpage
\section*{\underline{\textbf{\textcolor{red}{I.Dérivabilité en un point}}}}
\subsection*{\underline{\textbf{\textcolor{red}{1.Définition}}}}
une fonction f définie sur $I$ contenant $x_{0}$ est dérivable en $x_{0}$ si\\ 
\[\lim_{x \to x_{0}}\frac{f(x)-f(x_{0})}{x-x_{0}}=\ell \] avec $(\ell \in\mathbb{R})$\\
$\ell$ est appelé nombre dérivée de la fonction $f$ en $x_{0}$ et est noté $f'(x_{0})$
\subsection*{\underline{\textbf{\textcolor{red}{Exemple}}}}
Soit $f(x)=x^{2}+1$  Etudier la dérivabilité de $f$ en $x_{0}=-2.$\\
\subsection*{\underline{\textbf{\textcolor{red}{Solution}}}}
\[\lim_{x \to -2}\frac{f(x)-f(-2)}{x-(-2)}=\lim_{x \to -2}\frac{x^{2}+1-5}{x+2}=
\lim_{x \to -2}\frac{(x-2)(x+2)}{x+2}=\lim_{x \to -2}\frac{(x-2)(x+2)}{x+2}=-4\]
Donc\[\lim_{x \to -2}\frac{f(x)-f(-2)}{x+2}=-4\]
Donc f est dérivable en $-2$ et sa dérivée en $-2$ est $-4$\\ On écrit $f'(-2)=-4$ 
\subsection*{\underline{\textbf{\textcolor{red}{2.Dérivabilté à gauche-Dérivabilté à droite}}}}
Soit f une fonction définie sur un intervalle I contenant $x_{0}$ et $\ell$ un réel.\\
$f$ est dérivable à gauche de $x_{0}$ si\[\lim_{x \to x_{0}^{-}}\frac{f(x)-f(x_{0})}{x-x_{0}}=\ell\]
$f$ est dérivable à droite de $x_{0}$ si\[\lim_{x \to x_{0}^{+}}\frac{f(x)-f(x_{0})}{x-x_{0}}=\ell\]
$f$ est dérivable en $x_{0}$ si\[\lim_{x \to x_{0}^{-}}\frac{f(x)-f(x_{0})}{x-x_{0}}=\lim_{x \to x_{0}^{+}}\frac{f(x)-f(x_{0})}{x-x_{0}}=\ell\]
\subsection*{\underline{\textbf{\textcolor{red}{Notation}}}}
Si f est dérivable à gauche de $x_{0}$  alors\[\lim_{x \to x_{0}^{-}}\frac{f(x)-f(x_{0})}{x-x_{0}}=f'_{g}(x)\](nombre dérivé de f à gauche de $x_{0}$)\\
Si f est dérivable à droite de $x_{0}$  alors\[\lim_{x \to x_{0}^{+}}\frac{f(x)-f(x_{0})}{x-x_{0}}=f'_{d}(x)\](nombre dérivé de f à droite de $x_{0}$)
\subsection*{\underline{\textbf{\textcolor{red}{Exemple}}}}
Etudier la dérivabilté de f en $x_{0}=0$\\
\[ f(x) = \begin{cases} 
  3x^{2}, & \text{si } x \leq 0 \\
  -2x^{2}, & \text{si } x > 0 
\end{cases} \]
\subsection*{\underline{\textbf{\textcolor{red}{Solution}}}}
Posons \[ f(x) = \begin{cases} 
  f(x)_{1}=3x^{2}, & \text{si } x \leq 0 \\
  f(x)_{2}=-2x^{2}, & \text{si } x > 0 
\end{cases} \]
$f$ est dérivable en 0 ssi \[\lim_{x \to 0^{-}}\frac{f(x)-f(x_{0})}{x-0}=\lim_{x \to 0^{+}}\frac{f(x)-f(x_{0})}{x-0}=\ell\]
En $0^{-}$, $f(x)=f(x)_{1}=3x^{2}$: \\
\[\lim_{x \to 0^{-}}\frac{f(x)-f(x_{0})}{x-0}=\lim_{x \to 0^{-}}\frac{f(x))}{x}=\lim_{x \to 0^{-}}3x=0\]
Ainsi \[\lim_{x \to 0^{-}}\frac{f(x)-f(x_{0})}{x-0}\]
Donc $f$ est dérivable en $0^{-}$\\
En $0^{-}$, $f(x)=f(x)_{1}=3x^{2}$: \\

\[\lim_{x \to 0^{-}}\frac{f(x)-f(x_{0})}{x-0}=\lim_{x \to 0^{-}}\frac{f(x))}{x}=\lim_{x \to 0^{-}}3x=0\]
\section*{\underline{\textbf{\textcolor{red}{II.Interprétation géométrique du nombre dérivé}}}}
\subsection*{\underline{\textbf{\textcolor{red}{1.Définition et Théorème}}}}
Soit $f$ une fonction dérivable en $x_0$. Notons $A(x_0, f(x_0))$ le point de coordonnées $(x_0, f(x_0))$ sur la courbe $\mathcal{C}_f$ de $f$. La tangente à la courbe $\mathcal{C}_f$ au point $A$ est la droite passant par $A$ et ayant pour coefficient directeur $f'(x_0)$.\\
\definecolor{qqwuqq}{rgb}{0,0.39215686274509803,0}
\begin{tikzpicture}[line cap=round,line join=round,>=triangle 45,x=1cm,y=1cm]
\begin{axis}[
  x=1cm, y=1cm,
  axis lines=middle,
  ymajorgrids=true,
  xmajorgrids=true,
  xmin=-7, xmax=7,
  ymin=-8, ymax=8,
  xtick={-2,-15,...,30},
  ytick={-18,-16,...,12},
]

\draw[line width=2pt,color=qqwuqq,smooth,samples=100,domain=-8:3] plot(\x,{(\x)^(2)});
\draw[line width=2pt,color=qqwuqq,smooth,samples=100,domain=-2:3] plot(\x,{(\x*2-1)});
\begin{scriptsize}
  \draw[color=qqwuqq] (-6,2) node {$f$};
\end{scriptsize}

\end{axis}
\end{tikzpicture}
L'éaqution réduite de cette tangeante est (T):$y=f'(x_{0})(x-x_{0})+f(x_{0})$
\subsection*{\underline{\textbf{\textcolor{red}{Exemple}}}}
Soit $f(x)=x^{2}+1$, Etudier la dérivabilité de f en 1\\
Determiner l'équation de la tangeante à $C_{f}$ au point d'abscisse 1.\\
\subsection*{\underline{\textbf{\textcolor{red}{2.Théorème}}}}
Toute Fonction dérivable en $x_{0}$ est continue en $x_{0}$
\subsection*{\underline{\textbf{\textcolor{red}{Remarque}}}}
La réciproque n'est pas toujours vraie c'est-à-dire, une fonction continue en $x_{0}$ peut ne pas être dérivable en $x_{0}$.
\subsection*{\underline{\textbf{\textcolor{red}{Exemple}}}}
La fonction $f(x)=|x|$ est-elle dérivable en $0$
\subsection*{\underline{\textbf{\textcolor{red}{Solution}}}}
\begin{itemize}
\item[•] Continuité de f en $0$\\
 \[\textbf{A-t-on}\lim_{x \to 0}f(x)=f(x_{0})?\]
 Oui donc f est continue en 0
 \item[•] Dérivabilité de f en $0$\\
 \[\textbf{A-t-on}\lim_{x \to 0}\frac{f(x)-f(x_{0})}{x-0}=\ell?\]
  \[\lim_{x \to 0}\frac{f(x)-f(x_{0})}{x-0}=\lim_{x \to 0}\frac{|x|}{x}\]
  Cherchons la limite a gauche et a droite de 0.
   \[\lim_{x \to 0}\frac{|x|}{x}=\lim_{x \to 0^{-}}\frac{-x}{x}=-1\]
     \[\lim_{x \to 0}\frac{|x|}{x}=\lim_{x \to 0^{+}}\frac{-x}{x}=1\]
     Comme \[\lim_{x \to 0^{-}}\frac{f(x)-f(0)}{x-0}\neq \lim_{x \to 0^{+}}\frac{f(x)-f(0)}{x-0}\] donc f n(est pas dérivable en 0
\end{itemize}
\section*{\underline{\textbf{\textcolor{red}{III.Fonction dérivée}}}}
\subsection*{\underline{\textbf{\textcolor{red}{1.définition}}}}
On dit que la fonction $f$ est dérivable sur I lorsqu'elle est dérivable en tout point de I.\\
La fonction qui a tout x élément de I associe le nombre dérivé de  f s'appele fonction dérivé de f et est notée\\
$f':x\mapsto f'(x)$
\subsection*{\underline{\textbf{\textcolor{red}{2.Fonction dérivée de quelques fonctions usuelles}}}}
\begin{center}
\begin{tabular}{|c|c|c|}
\hline
Fonction &Domaine de Dérivabilité &  Dérivée   \\
\hline
$f(x) = k$& $\mathbb{R}$ & $f'(x)=0$ \\
\hline
$f(x) = x$& $\mathbb{R}$ & $f'(x) = 1$ \\
\hline
$f(x) = x^{n}$& $\mathbb{R}$ & $f'(x) = nx^{n-1}$ \\
\hline
$f(x) = \frac{1}{x}$& $\mathbb{R}^{*}$ & $f'(x) = -\frac{1}{x^{2}}$ \\
\hline
$f(x) = \sqrt{x}$& $\mathbb{R}^{*}_{+}$ & $f'(x) = \frac{1}{2\sqrt{x}}$ \\
\hline
\end{tabular}
\end{center}
\subsection*{\underline{\textbf{\textcolor{red}{3.Opération sur les fonctions dérivables}}}}
Soient u et v deux fonctions dérivables sur I et dont les fonctions dérivées sont notées u' et v'.
\begin{center}
\begin{tabular}{|c|c|c|}
\hline
Fonction &Domaine de Dérivabilité &  Dérivée   \\
\hline
$ku;k\in \mathbb{R}$& $I$ & $ku'$ \\
\hline
$u^{n}, \forall n \in \mathbb{N}\setminus{0,1}$ & $I$ &$nu'u^{n-1}$\\
\hline
$u+v$& $I$ & $u'+v'$ \\
\hline
$uv$& $I$ & $u'v+v'u$ \\
\hline
$\frac{1}{v}$& $\forall \in I; v(x) \neq 0$ & $-\frac{v'}{v^{2}}$ \\
\hline
$\frac{u}{v}$& $\forall \in I; v(x) \neq 0$ & $\frac{u'v-v'u}{v^{2}}$ \\
\hline
$\sqrt{u}$& $\forall \in I; u(x) > 0$ & $\frac{u'}{2\sqrt{u}}$ \\
\hline
\end{tabular}
\end{center}
\subsection*{\underline{\textbf{\textcolor{red}{4.Interprétation graphique du nombre dérivée à gauche et à droite}}}}
\subsection*{\underline{\textbf{\textcolor{red}{5.Propriété}}}}
\subsection*{\underline{\textbf{\textcolor{red}{6.Dérivée d'une fonction trigonométrique}}}}
\subsection*{\underline{\textbf{\textcolor{red}{7.Dérivée d'une fonction composée}}}}
\end{document}