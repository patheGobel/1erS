\documentclass{article}
\usepackage{stmaryrd}
\usepackage{graphicx}
\usepackage[utf8]{inputenc}

\usepackage[french]{babel}
\usepackage[T1]{fontenc}
\usepackage{hyperref}
\usepackage{verbatim}

\usepackage{color, soul}
\usepackage{xcolor} % Ajout pour les couleurs

\usepackage{pgfplots}
\pgfplotsset{compat=1.15}
\usepackage{mathrsfs}

\usepackage{amsmath}
\usepackage{amsfonts}
\usepackage{amssymb}
\usepackage{tkz-tab}
\author{Destinés à la 1\textsuperscript{ère}S2\\Au Lycée de Dindéferlo}
\title{\textbf{Produit Scalaire}}
\date{\today}
\usepackage{tikz}
\usetikzlibrary{arrows, shapes.geometric, fit}

% Commande pour la couleur d'accentuation
\newcommand{\myul}[2][black]{\setulcolor{#1}\ul{#2}\setulcolor{black}}
\newcommand\tab[1][1cm]{\hspace*{#1}}

% Redéfinition des sous-sections avec une couleur
\makeatletter
\renewcommand{\@seccntformat}[1]{\color{blue}\csname the#1\endcsname\quad}
\renewcommand\subsection{\@startsection{subsection}{2}{0mm}%
    {-\baselineskip}{0.5\baselineskip}{\normalfont\bfseries\color{red}}}
% Redéfinition des sections avec une couleur
\renewcommand\section{\@startsection{section}{1}{0mm}%
    {-\baselineskip}{0.5\baselineskip}{\normalfont\LARGE\bfseries\color{blue}}}

\makeatother

\begin{document}
\maketitle
\newpage

\section*{I. Produit scalaire de deux vecteurs}
\subsection*{1. Définition}

Soient $\vec{u}$ et $\vec{v}$ deux vecteurs non nuls, on appelle \textbf{produit scalaire} du vecteur $\vec{u}$ par le vecteur $\vec{v}$ le réel noté $\vec{u} \cdot \vec{v}$ défini par :
\[
\vec{u} \cdot \vec{v} = \|\vec{u}\| \times \|\vec{v}\| \times \cos(\widehat{\vec{u}, \vec{v}})
\]

Si $\vec{u} = \vec{0}$ ou $\vec{v} = \vec{0}$, alors :
\(
\vec{u} \cdot \vec{v} = 0
\)

\textbf{Attention} : Le produit scalaire de deux vecteurs n'est \textbf{pas un vecteur} mais un \textbf{nombre réel}.
\section*{Remarque}

Si $A$, $B$, $C$ sont trois points du plan, en posant $\vec{u} = \overrightarrow{AB}$ et $\vec{v} = \overrightarrow{AC}$, on a :
\[
\overrightarrow{AB} \cdot \overrightarrow{AC} = \|\overrightarrow{AB}\| \times \|\overrightarrow{AC}\| \times \cos(\widehat{BAC})
\]

\subsection*{Cas particulier de deux vecteurs colinéaires $\vec{u}$ et $\vec{v}$ non nuls}

\begin{itemize}
    \item Si $\vec{u}$ et $\vec{v}$ sont colinéaires et de même sens, alors :
    \[
    \vec{u} \cdot \vec{v} = \|\vec{u}\| \times \|\vec{v}\| \quad \text{car} \quad \cos(\widehat{\vec{u}, \vec{v}}) = 1
    \]
    \item Si $\vec{u}$ et $\vec{v}$ sont colinéaires et de sens contraires, alors :
    \[
    \vec{u} \cdot \vec{v} = -\|\vec{u}\| \times \|\vec{v}\| \quad \text{car} \quad \cos(\widehat{\vec{u}, \vec{v}}) = -1
    \]
\end{itemize}
dans un repère orthonormé, alors :
\[
\vec{u} \cdot \vec{v} = xx' + yy'
\]

\textbf{Remarque} : Ce résultat est indépendant du repère choisi.
\subsection*{2) Propriété}

\begin{itemize}
    \item \textbf{P1 :} 
    Deux vecteurs $\vec{u}$ et $\vec{v}$ sont orthogonaux si l'un d'eux est nul ou si leurs directions sont orthogonales.
    
    \item \textbf{P2 :} 
    Le produit scalaire $\vec{u} \cdot \vec{v}$ est nul si et seulement si $\vec{u}$ et $\vec{v}$ sont orthogonaux.
    
    \item \textbf{P3 :} 
    Si $\vec{u}$ a pour composant $\vec{u}(x, y)$ et $\vec{v}(x', y')$, alors :
    \[
    \vec{u} \cdot \vec{v} = x \cdot x' + y \cdot y'.
    \]

\textbf{Remarque} : Ce résultat est indépendant du repère choisi.

    \item \textbf{P4 :} 
    Soient $\vec{u}$, $\vec{v}$, $\vec{w}$ trois vecteurs du plan et $k$ un réel :
\begin{itemize}
    \item \(
    \vec{u} \cdot \vec{v} = \vec{v} \cdot \vec{u} \quad \text{(symétrie ou commutativité)}
    \)
    \item \(
    \vec{u} \cdot (k \vec{v}) = k (\vec{u} \cdot \vec{v}) \quad \text{et} \quad \vec{u} \cdot (\vec{v} + \vec{w}) = \vec{u} \cdot \vec{v} + \vec{u} \cdot \vec{w}.
    \)
    On dit que le produit scalaire est linéaire.
\end{itemize}
    
    \item \textbf{P5 :} 
    $\vec{u} \cdot \vec{u} = \|\vec{u}\|^2$. Ce nombre est appelé le carré scalaire de $\vec{u}$ et est noté $\vec{u}^2$.

    \item \textbf{P6 :} 
    Si $\vec{u}(x_C, y_C)$ et $\vec{v}(x_B, y_B)$ dans un repère orthonormé, alors :
    \[
    \vec{u} \cdot \vec{v} = x_C \cdot x_B + y_C \cdot y_B, \quad AB = \sqrt{(x_B - x_A)^2 + (y_B - y_A)^2}.
    \]
\end{itemize}
\subsection*{Exerice d'application}

Soient $A(-2, -3)$, $B(1, 1)$, $C(-3, -1)$, $D(4, 2)$, $E(-4, -3)$ et $E(2, -1)$ dans un repère orthonormé.

En utilisant deux méthodes différentes (une méthode par triangle), montrer que le triangle $ABC$ est un triangle rectangle en $C$, mais que $FDE$ ne l'est pas en $E$.
\subsection*{Correction}
\textbf{Calculons $AB$, $AC$ et $BC$ :}

\[
AB = \sqrt{(1 + 2)^2 + (1 + 3)^2}
\]
\[
AB = \sqrt{3^2 + 4^2} = \sqrt{9 + 16} = 5
\]

\[
AC = \sqrt{(-3 + 2)^2 + (-1 + 3)^2}
\]
\[
AC = \sqrt{(-1)^2 + 2^2} = \sqrt{1 + 4} = \sqrt{5}
\]

\[
BC = \sqrt{(-3 - 1)^2 + (-1 - 1)^2}
\]
\[
BC = \sqrt{16 + 4} = \sqrt{20}
\]
\[
BC = 2\sqrt{5}
\]

\textbf{Calculons} $AB^2$ :
\[
AB^2 = 5^2 = 25
\]

\textbf{Calculons} $AC^2 + BC^2$ :
\[
AC^2 + BC^2 = (\sqrt{5})^2 + (2\sqrt{5})^2
\]
\[
AC^2 + BC^2 = 5 + 20 = 25
\]

Ainsi, on a :
\[
AB^2 = AC^2 + BC^2
\]

Comme $AB^2 = AC^2 + BC^2$, d'après la réciproque du théorème de Pythagore, le triangle $ABC$ est rectangle en $C$.

Le triangle $EDF$ n'est pas rectangle en $E$ si et seulement si :
\[
\widehat{FED} \neq 90^\circ
\]
C'est-à-dire que :
\[
\vec{EF} \cdot \vec{ED} \neq 0
\]

\textbf{les coordonnées des vecteurs $\vec{EF}$ et $\vec{ED}$ :}

\[
\vec{EF}(3, 2) \quad \text{et} \quad \vec{ED}(-3, 5)
\]

\[
\vec{EF} \cdot \vec{ED} = (3 \times -3) + (2 \times 5)
\]
\[
\vec{EF} \cdot \vec{ED} = 1 = 36 \neq 0
\]

Donc, $EDF$ n'est pas rectangle en $E$.

\subsubsection*{3) Identités remarquables :}

Soient $\vec{u}$ et $\vec{v}$ deux vecteurs du plan.

\[
I_1 : (\vec{u} + \vec{v})^2 = \|\vec{u}\|^2 + \|\vec{v}\|^2 + 2(\vec{u} \cdot \vec{v})
\]
\[
\iff \|\vec{u} + \vec{v}\|^2 = \|\vec{u}\|^2 + \|\vec{v}\|^2 + 2\|\vec{u}\|\|\vec{v}\| \cos(\theta)
\]

\[
I_2 : (\vec{u} - \vec{v})^2 = \|\vec{u}\|^2 + \|\vec{v}\|^2 - 2(\vec{u} \cdot \vec{v})
\]
\[
\iff \|\vec{u} - \vec{v}\|^2 = \|\vec{u}\|^2 + \|\vec{v}\|^2 - 2\|\vec{u}\|\|\vec{v}\| \cos(\theta)
\]

\[
I_3 : (\vec{u} + \vec{v})(\vec{u} - \vec{v}) = \|\vec{u}\|^2 - \|\vec{v}\|^2
\]

\end{document}
