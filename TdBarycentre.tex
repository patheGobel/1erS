\documentclass[12pt]{article}
\usepackage{lmodern} % Pour une police plus nette
\usepackage{stmaryrd}
\usepackage{graphicx} % Pour l'insertion d'images
\usepackage{float}    % Pour contrôler précisément le placement
\usepackage[utf8]{inputenc}
\usepackage[french]{babel}
\usepackage[T1]{fontenc}
\usepackage{hyperref}
\usepackage{verbatim}
\usepackage{color, soul}
\usepackage{pgfplots}
\pgfplotsset{compat=1.18} % Version plus récente de pgfplots
\usepackage{mathrsfs}
\usepackage{amsmath}
\usepackage{amsfonts}
\usepackage{amssymb}
\usepackage{tkz-tab}
%\author{Destiné aux élèves de Terminale S\\Lycée de Dindéfelo\\Présenté par M. BA}
%\title{\textbf{Rappels et compléments sur les fonctions numériques}}
%\date{\today}
\usepackage{tikz}
\usetikzlibrary{arrows, shapes.geometric, fit}
% Commande pour la couleur d'accentuation
\newcommand{\myul}[2][black]{\setulcolor{#1}\ul{#2}\setulcolor{black}}
\newcommand\tab[1][1cm]{\hspace*{#1}}
\usepackage[margin=2.5cm]{geometry} % Ajustement des marges
\usepackage{eso-pic} % Pour ajouter des éléments en arrière-plan

% Commande pour ajouter du texte en arrière-plan, centré au milieu de chaque page
\AddToShipoutPicture{
    \AtPageCenter{%
        \makebox(0,0)[c]{\rotatebox{60}{\textcolor[gray]{0.6}{\fontsize{2cm}{2cm}\selectfont PGB}}}
    }
}

\begin{document}

\noindent
\begin{minipage}[t]{0.48\textwidth}
\raggedright
\textbf{Ministère de l'Éducation Nationale}\\
Inspection Académique de Kédougou\\
Lycée Dindéfelo\\
Cellule de Mathématiques
\end{minipage}
\hfill
\begin{minipage}[t]{0.48\textwidth}
\raggedleft
\textbf{Année scolaire 2024-2025}\\
Date : 07/01/2025\\
Classe : Terminale S2\\
Professeur : M. BA
\end{minipage}
\vspace{1cm}
\begin{center}
\textbf{\textcolor{red}{Barycentre-Vecteur}}
\end{center}
\vspace{1cm}

\section*{\underline{Exercice 1}}

Soient \( A \), \( B \), \( C \) points du plan \( \mathcal{P} \).

\begin{enumerate}
    \item[1°)] Déterminer l'ensemble des points \( M \) de \( \mathcal{P} \) tels que :
    \[
    \| 3 \overrightarrow{MA} - \overrightarrow{MB} - \overrightarrow{MC} \| = \| - \overrightarrow{MA} + 3 \overrightarrow{MB} - \overrightarrow{MC} \|.
    \]
    
    \item[2°)] Existe-t-il un point \( M \) de \( \mathcal{P} \) tel que :
    \[
    \| 3 \overrightarrow{MA} - \overrightarrow{MB} - \overrightarrow{MC} \| = \| - \overrightarrow{MA} + 3 \overrightarrow{MB} - \overrightarrow{MC} \| = \| - \overrightarrow{MA} - \overrightarrow{MB} + 3 \overrightarrow{MC} \| ?
    \]

    \item[3°)] Déterminer l'ensemble des points \( M \) de \( \mathcal{P} \) tels que :
    \[
    \| \overrightarrow{MA} + \overrightarrow{MB} + \overrightarrow{MC} \| =  \| 2\overrightarrow{MA} - \overrightarrow{MB} - \overrightarrow{MC} \|.
    \]
\end{enumerate}

\section*{\underline{Exercice 2}}

Étant donné un triangle \( ABC \), soient les points \( M \) et \( N \) définis par :
\[
\overrightarrow{AM} = \frac{1}{2} \overrightarrow{AB} - \overrightarrow{AC} \quad \text{et} \quad \overrightarrow{AN} = -\overrightarrow{AB} + \frac{1}{2} \overrightarrow{AC}.
\]

\begin{enumerate}
    \item[1°)] Montrer que \( (MN) \) est parallèle à \( (BC) \).
    \item[2°)] Donner les coordonnées de \( M \) et \( N \) dans les repères \( (A, \overrightarrow{AB}, \overrightarrow{AC}) \) puis \( (B, \overrightarrow{BA}, \overrightarrow{BC}) \).
    \item[3°)] On définit maintenant les points \( M \) et \( N \) par :
    \[
    \overrightarrow{AM} = \frac{1}{2} \overrightarrow{AB} + (1 - k) \overrightarrow{AC} \quad \text{et} \quad \overrightarrow{AN} = (1 - k) \overrightarrow{AB} + \frac{1}{2} \overrightarrow{AC}, \quad (k \in \mathbb{R}).
    \]
    \begin{enumerate}
        \item[a)] Exprimer \( \overrightarrow{MN} \) en fonction de \( \overrightarrow{BC} \).
        \item[b)] Déterminer \( k \) pour que \( BCMN \) soit un parallélogramme.
    \end{enumerate}
\end{enumerate}

\section*{\underline{Exercice 3}}

On considère \( A \), \( B \), \( C \) trois points distincts du plan, \( a \), \( b \) et \( c \) trois réels non nuls. 

On donne \( \overrightarrow{AI} = -2 \overrightarrow{AB} \) et \( \overrightarrow{BJ} = 3 \overrightarrow{BC} \).

\begin{enumerate}
    \item[1°)] Déterminer \( a \) et \( b \) tels que 
    \[
    I = \text{bar}\left\lbrace  \left( A, a \right), \left( B, b \right) \right\rbrace  \quad \text{et} \quad J = \text{bar}\left\lbrace  \left( B, b \right), \left( C, c \right) \right\rbrace .
    \]
    
    \item[2°)] Montrer que \( I = \text{bar}\left\lbrace \left( A, 18 \right), \left( B, -12 \right) \right\rbrace \) 
    
    \item[3°)] Déterminer et construire l'ensemble des points \( M \) du plan tels que :
    \begin{enumerate}
        \item[a)] \( \| 3 \overrightarrow{MA} - 2 \overrightarrow{MB} \| = \| - 2 \overrightarrow{MB} + 3 \overrightarrow{MC} \| \).
        \item[b)] \( \| 18 \overrightarrow{MA} - 12 \overrightarrow{MB} \| = \| \overrightarrow{MA} -  \overrightarrow{MB} \| \).
    \end{enumerate}
\end{enumerate}

\section*{\underline{Exercice 4}}

Soit \( ABC \) un triangle tel que \( AB = 4 \, \text{cm} \) et \( BC = 5 \, \text{cm} \). 

On donne \( I = \text{bar} \left\lbrace  (A, 3), (B, -2) \right\rbrace  \) et \( J = \text{bar} \left\lbrace  (B, -2), (C, 3) \right\rbrace  \).

Soit \( G \) le point défini par \( \overrightarrow{3GA} - \overrightarrow{2GB} + \overrightarrow{3GC} = \overrightarrow{O} \).

\begin{enumerate}
    \item[1)] Montrer que \( G = \text{bar} \left( (I, 1),(C, 3) \right) \).
    
    \item[2)] Déterminer et construire l'ensemble des points \( G \) du plan tels que :
    \begin{enumerate}
        \item[a)] \( \| 3 \overrightarrow{MA} - 2 \overrightarrow{MB} + 3 \overrightarrow{MC} \| = AB \)
        \item[b)] \( 18 \overrightarrow{MA} - 12 \overrightarrow{MB} = \| \overrightarrow{MA} - \overrightarrow{MB} \| \)
        \item[c)] \( \| 3 \overrightarrow{MA} - 2 \overrightarrow{MB} \| = \| -2\overrightarrow{MB} +3 \overrightarrow{MC} \| \)
    \end{enumerate}
\end{enumerate}

\section*{\underline{Exercice 5}}

\( ABCD \) est un rectangle.

On donne \( G = \text{bar} \left\lbrace (A, 1), (B, 1), (C, 1), (D, 1) \right\rbrace  \).

\begin{enumerate}
    \item[1)] Justifier l'existence de \( G \).
    
    \item[2)] Montrer que \( A \), \( G \), \( C \) sont alignés.
    
    \item[3)] Soit \( I \) le milieu de \( [AB] \), et \( J \) celui de \( IC \). 
    
    Montrer que \( G \), \( J \), \( D \) sont alignés.
    
    \item[4)] Déterminer et construire l'ensemble des points \( M \) du plan tels que :
    \begin{enumerate}
        \item[a)] \( \| \overrightarrow{MA} + \overrightarrow{MB} + 2\overrightarrow{MC} + \overrightarrow{MD} \| = AB \)
        \item[b)] \( \| \overrightarrow{MA} + \overrightarrow{MB} + 2\overrightarrow{MC} + \overrightarrow{MD} \| = \| \overrightarrow{AB} \| \)
        \item[c)] \( \overrightarrow{MA} + \overrightarrow{MB} + 2\overrightarrow{MC} + \overrightarrow{MD}  \) soit colinéaire à \( \overrightarrow{AB} \)
    \end{enumerate}
\end{enumerate}

\end{document}