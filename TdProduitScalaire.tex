\documentclass[12pt]{article}
\usepackage{lmodern} % Pour une police plus nette
\usepackage{stmaryrd}
\usepackage{graphicx} % Pour l'insertion d'images
\usepackage{float}    % Pour contrôler précisément le placement
\usepackage[utf8]{inputenc}
\usepackage[french]{babel}
\usepackage[T1]{fontenc}
\usepackage{hyperref}
\usepackage{verbatim}
\usepackage{color, soul}
\usepackage{pgfplots}
\pgfplotsset{compat=1.18} % Version plus récente de pgfplots
\usepackage{mathrsfs}
\usepackage{amsmath}
\usepackage{amsfonts}
\usepackage{amssymb}
\usepackage{tkz-tab}
%\author{Destiné aux élèves de Terminale S\\Lycée de Dindéfelo\\Présenté par M. BA}
%\title{\textbf{Rappels et compléments sur les fonctions numériques}}
%\date{\today}
\usepackage{tikz}
\usetikzlibrary{arrows, shapes.geometric, fit}
% Commande pour la couleur d'accentuation
\newcommand{\myul}[2][black]{\setulcolor{#1}\ul{#2}\setulcolor{black}}
\newcommand\tab[1][1cm]{\hspace*{#1}}
\usepackage[margin=2.5cm]{geometry} % Ajustement des marges
\usepackage{eso-pic} % Pour ajouter des éléments en arrière-plan

% Commande pour ajouter du texte en arrière-plan, centré au milieu de chaque page
\AddToShipoutPicture{
    \AtPageCenter{%
        \makebox(0,0)[c]{\rotatebox{60}{\textcolor[gray]{0.6}{\fontsize{2cm}{2cm}\selectfont PGB}}}
    }
}

\begin{document}

\noindent
\begin{minipage}[t]{0.48\textwidth}
\raggedright
\textbf{Ministère de l'Éducation Nationale}\\
Inspection Académique de Kédougou\\
Lycée Dindéfelo\\
Cellule de Mathématiques
\end{minipage}
\hfill
\begin{minipage}[t]{0.48\textwidth}
\raggedleft
\textbf{Année scolaire 2024-2025}\\
Date : 07/01/2025\\
Classe : Terminale S2\\
Professeur : M. BA
\end{minipage}
\vspace{1cm}
\begin{center}
\textbf{\textcolor{red}{Produit Scalaire}}
\end{center}
\vspace{1cm}
\section*{Exercice 1}

$ABC$ est un triangle isocèle de sommet principal $A$ et de base $[BC]$ de $2\ \text{cm}$. $I$ est le milieu de $[BC]$. Tracer une figure et calculer en utilisant les projections orthogonales :
\[
\text{a) } \overrightarrow{BI} \cdot \overrightarrow{BC}; \quad
\text{b) } \overrightarrow{AI} \cdot \overrightarrow{IC}; \quad
\text{c) } \overrightarrow{BA} \cdot \overrightarrow{BI}; \quad
\text{d) } \overrightarrow{AC} \cdot \overrightarrow{CI}; 
\]
\[
\text{e) } \overrightarrow{AI} \cdot \overrightarrow{BC}; \quad
\text{f) } \overrightarrow{IB} \cdot \overrightarrow{IC}; \quad
\text{g) } \overrightarrow{BC} \cdot \overrightarrow{CI}.
\]

\section*{Exercice 2}

Soit un carré $ABCD$ de centre $O$ et de côté $a$. Calculer en fonction de $a$, les produits scalaires suivants :
\[
\text{a) } \overrightarrow{AB} \cdot \overrightarrow{AC}; \quad
\text{b) } \overrightarrow{BC} \cdot \overrightarrow{BA}; \quad
\text{c) } \overrightarrow{OC} \cdot \overrightarrow{OB}; 
\]
\[
\text{d) } \overrightarrow{AC} \cdot \overrightarrow{AO}; \quad
\text{e) } \overrightarrow{OB} \cdot \overrightarrow{OD}; \quad
\text{f) } \overrightarrow{AD} \cdot \overrightarrow{OB}.
\]

\section*{Exercice 3}

$ABCD$ est un parallélogramme tel que $AB = 4\ \text{cm}$ ; $AD = 3\ \text{cm}$ et $\overrightarrow{AB} \cdot \overrightarrow{AD} = 6$.
\begin{enumerate}
    \item Déterminer une mesure en degré de l’angle géométrique $\widehat{BAD}$.
    \item Construire le parallélogramme $ABCD$ et déterminer les produits scalaires $\overrightarrow{AB} \cdot \overrightarrow{AC}$ et $\overrightarrow{AB} \cdot \overrightarrow{BC}$.
\end{enumerate}
\section*{Exercice 4}

\begin{enumerate}
    \item Montrer qu’un triangle $ABC$ est isocèle de sommet principal $A$ si et seulement si
    \[
    \overrightarrow{BA} \cdot \overrightarrow{BC} = \overrightarrow{CA} \cdot \overrightarrow{CB}.
    \]

    \item Quelle est la nature d’un triangle $ABC$ qui vérifie
    \[
    \overrightarrow{BA} \cdot \overrightarrow{BC} = \overrightarrow{CA} \cdot \overrightarrow{CB} = \overrightarrow{AC} \cdot \overrightarrow{AB}?
    \]
\end{enumerate}

\section*{Exercice 5}

$ABC$ est un triangle. $A'$, $B'$ et $C'$ sont les pieds des hauteurs issues respectivement de $A$, $B$ et $C$. Le point $H$ est l’orthocentre du triangle $ABC$.

\begin{enumerate}
    \item Justifier que 
    \[
    \overrightarrow{HA} \cdot \overrightarrow{HA'} = \overrightarrow{HA} \cdot \overrightarrow{HB} = \overrightarrow{HA} \cdot \overrightarrow{HC}.
    \]

    \item Déduire que 
    \[
    \overrightarrow{HA} \cdot \overrightarrow{HA'} = \overrightarrow{HB} \cdot \overrightarrow{HB'} = \overrightarrow{HC} \cdot \overrightarrow{HC'}.
    \]
\end{enumerate}

\section*{Exercice 6}

Soit $ABCD$ un parallélogramme.

\begin{enumerate}
    \item Exprimer $\overrightarrow{AC}$ en fonction de $\overrightarrow{AB}$ et $\overrightarrow{AD}$.

    \item Exprimer $\overrightarrow{DB}$ en fonction de $\overrightarrow{AB}$ et $\overrightarrow{AD}$.

    \item Utiliser les expressions précédentes pour montrer que 
    \[
    AC^2 + BD^2 = 2(AB^2 + AD^2).
    \]

    \item Peut-on construire un parallélogramme $ABCD$ tel que $AB = 3\ \text{cm}$, $AD = 4\ \text{cm}$ et $AC = 1\ \text{cm}$ ?
\end{enumerate}
\section*{Exercice 7}

Soit un rectangle $ABCD$. On considère :
\begin{itemize}
    \item Sur la demi-droite $[AD)$, le point $B'$ tel que $AB' = AB$.
    \item Sur la demi-droite opposée à $[AB)$, le point $D'$ tel que $AD' = AD$.
\end{itemize}
\begin{enumerate}
    \item Démontrer que $(BD) \perp (B'D')$.
    \item Que représente $D$ pour le triangle $BB'D'$ ?
    \item En déduire que $(DD') \perp (BB')$.
\end{enumerate}
\section*{Exercice 8}

Déterminer l’équation du cercle $(C)$ de centre $O(a; b)$ et de rayon $R$.
\begin{enumerate}
    \item $a = -3$ ; $b = 2$ et $R = \sqrt{2}$.
    \item $a = \frac{1}{2}$ ; $b = -\frac{1}{3}$ et $R = \frac{3}{2}$.
\end{enumerate}

\section*{Exercice 9}

Déterminer l’équation du cercle $C$ de diamètre $[AB]$.
\begin{enumerate}
    \item $A(1; 1)$ et $B(5; 3)$.
    \item $A\left(\frac{1}{2}; \frac{1}{2}\right)$ et $B\left(\frac{3}{2}; -\frac{1}{2}\right)$.
\end{enumerate}

\section*{Exercice 10}

Déterminer l’ensemble des points $M$ dont les coordonnées vérifient les équations suivantes :
\begin{enumerate}
    \item $x^2 + y^2 - 6x + 2y + 5 = 0$.
    \item $x^2 + y^2 - x + 2y + 2 = 0$.
    \item $5x^2 + 5y^2 + 15x - 5y + 10 = 0$.
\end{enumerate}
\section*{Exercice 11}

Soit $(O; \vec{i}, \vec{j})$ un repère orthonormé et $(\Delta)$ la droite d’équation $ax + by + c = 0$ de vecteur normal $\vec{n}\left(\begin{array}{c}a \\ b\end{array}\right)$.  
$M$ est le point de coordonnées $(\alpha; \beta)$ et $H$ est le projeté orthogonal de $M$ sur $(\Delta)$.

\begin{enumerate}
    \item Vérifier que pour tout point $P$ de $(\Delta)$, on a $\vec{n} \cdot \overrightarrow{OP} = -c$.

    \item Calculer $\vec{n} \cdot \overrightarrow{OM}$ en fonction de $a$, $b$, $\alpha$ et $\beta$.

    \item Déduire des questions 1. et 2. que $\vec{n} \cdot \overrightarrow{HM} = a\alpha + b\beta + c$.

    \item Justifier l’égalité $\|\vec{n} \cdot \overrightarrow{HM}\| = \|\vec{n}\| \times HM$.

    \item Conclure que 
    \[
    HM = \frac{|a\alpha + b\beta + c|}{\sqrt{a^2 + b^2}}
    \]
    où $HM$ est la distance du point $M$ à la droite $(\Delta)$ et est notée $d((\Delta), M)$.

    \item \textbf{Application} : Soit $(\Delta) : x - 2y + 2 = 0$.
    \begin{enumerate}
        \item Déterminer la distance de $M(1; 4)$ à $(\Delta)$.
        \item Soit $N(0; 1)$. Déterminer $d((\Delta), N)$ et conclure.
    \end{enumerate}
\end{enumerate}
\section*{Exercice 12}

\begin{enumerate}
    \item \textbf{Les théorèmes de la médiane} : $MAB$ étant un triangle ; $I$ milieu de $[AB]$, montrer que :
    \begin{enumerate}
        \item $MA^2 + MB^2 = 2IM^2 + \frac{AB^2}{2}$.
        \item $MA^2 - MB^2 = 2\overrightarrow{IM} \cdot \overrightarrow{AB}$.
        \item $\overrightarrow{MA} \cdot \overrightarrow{MB} = IM^2 - \frac{AB^2}{4}$.
    \end{enumerate}

    \item \textbf{Le théorème d’Alkashi} : Si $ABC$ est un triangle tel que $AB = c$ ; $AC = b$ et $BC = a$. Montrer que :
    \[
    a^2 = b^2 + c^2 - 2bc\cos\widehat{A}.
    \]

    \item \textbf{Le théorème des sinus} : Si $ABC$ est un triangle, établir les relations :
    \[
    \frac{a}{\sin\widehat{A}} = \frac{b}{\sin\widehat{B}} = \frac{c}{\sin\widehat{C}} = \frac{abc}{2S},
    \]
    où $S$ est l’aire du triangle.
\end{enumerate}
\section*{Exercice 13}

Soit $A$ et $B$ deux points du plan tels que $AB = 4$.

\begin{enumerate}
    \item Déterminer l’ensemble des points $M$ du plan vérifiant :
    \begin{enumerate}
        \item $\overrightarrow{MA} \cdot \overrightarrow{MB} = 0$.
        \item $\overrightarrow{MA} \cdot \overrightarrow{MB} = -9$.
        \item $\overrightarrow{MA} \cdot \overrightarrow{MB} = -30$.
        \item $\overrightarrow{MA} \cdot \overrightarrow{MB} = 5$.
    \end{enumerate}

    \item Déterminer l’ensemble des points $M$ du plan vérifiant :
    \begin{enumerate}
        \item $MA^2 - MB^2 = 0$.
        \item $MA^2 - MB^2 = 10$.
        \item $MA^2 - MB^2 = -30$.
    \end{enumerate}

    \item Déterminer l’ensemble des points $M$ du plan vérifiant :
    \begin{enumerate}
        \item $MA^2 + MB^2 = 0$.
        \item $MA^2 + MB^2 = 22,5$.
        \item $MA^2 + MB^2 = 7$.
    \end{enumerate}
\end{enumerate}
\begin{enumerate}
    \setcounter{enumi}{3}
    \item Déterminer l’ensemble des points $M$ du plan vérifiant :
    \begin{enumerate}
        \item $\overrightarrow{AB} \cdot \overrightarrow{AM} = 0$.
        \item $\overrightarrow{AB} \cdot \overrightarrow{AM} = 5$.
        \item $\overrightarrow{AB} \cdot \overrightarrow{AM} = -2$.
    \end{enumerate}

    \item Déterminer l’ensemble des points $M$ du plan vérifiant :
    \begin{enumerate}
        \item $\frac{MA}{MB} = 2$.
        \item $\frac{MA}{MB} = 5$.
    \end{enumerate}
\end{enumerate}
\section*{Exercice 14}

Soit $ABC$ un triangle isocèle tel que $AB = AC = 5$ et $BC = 6$.

\begin{enumerate}
    \item Montrer que $\overrightarrow{AC} \cdot \overrightarrow{AB} = 7$.

    \item Soit $G = \text{bar}\{(A, 2); (B, 3); (C, 3)\}$. Construire $G$ et montrer que $AG = 3$.

    \item Soit $f$ l’application qui à tout point $M$ du plan, associe :
    \[
    f(M) = 2\overrightarrow{MB} \cdot \overrightarrow{MC} + \overrightarrow{MC} \cdot \overrightarrow{MA} + \overrightarrow{MA} \cdot \overrightarrow{MB}.
    \]
    \begin{enumerate}
        \item Calculer $f(A)$ et $f(G)$.
        \item Déterminer l’ensemble des points $M$ tels que $f(M) = f(A)$ et représenter cet ensemble.
    \end{enumerate}
\end{enumerate}
\section*{Exercice 15}

Soit $ABC$ un triangle tel que $AB = 6$, $AC = 8$ et $\widehat{CAB} = \frac{\pi}{3}$.

\begin{enumerate}
    \item 
    \begin{enumerate}
        \item Calculer $\overrightarrow{AC} \cdot \overrightarrow{AB}$.
        \item En déduire $BC = 2\sqrt{13}$.
    \end{enumerate}

    \item Soit $H$ le projeté orthogonal de $B$ sur $(AC)$.
    \begin{enumerate}
        \item Montrer que $AH = 3$ et vérifier que $H$ est le barycentre des points $A$ et $C$ affectés des coefficients que l’on déterminera.
        \item Montrer que pour tout point $M$ du plan :
        \[
        5MA^2 + 3MC^2 = 8MH^2 + 120.
        \]
    \end{enumerate}

    \item 
    \begin{enumerate}
        \item Déterminer l’ensemble $(C)$ des points $M$ du plan vérifiant $5MA^2 + 3MC^2 = 336$.
        \item Vérifier que $B \in (C)$ et construire $(C)$.
    \end{enumerate}
\end{enumerate}
\section*{Exercice 16}

On considère le triangle $ABC$ tel que $AB = a$, $AC = 3a$ (où $a$ est un réel strictement positif) et $\widehat{BAC} = \frac{2\pi}{3}$.  
Soient $H$ le projeté orthogonal de $C$ sur $(AB)$ et le point $O$ milieu de $[BC]$.

\begin{enumerate}
    \item Faire une figure.

    \item 
    \begin{enumerate}
        \item Calculer $\overrightarrow{AB} \cdot \overrightarrow{AC}$.
        \item En déduire $AH$, $CB$ et $CH$ en fonction de $a$.
    \end{enumerate}

    \item 
    \begin{enumerate}
        \item Calculer $\overrightarrow{BA} \cdot \overrightarrow{BC}$.
        \item En déduire $AO$ en fonction de $a$.
    \end{enumerate}

    \item 
    \begin{enumerate}
        \item Soit $I$ milieu de $[AO]$ ; montrer que pour tout point $M$ du plan, on a 
        \[
        \overrightarrow{MA} \cdot (\overrightarrow{MB} + \overrightarrow{MC}) = 2(MI^2 + IA^2).
        \]

        \item Déterminer l’ensemble des points $M$ du plan tels que 
        \[
        \overrightarrow{MA} \cdot (\overrightarrow{MB} + \overrightarrow{MC}) = \frac{a^2}{4}.
        \]
    \end{enumerate}
\end{enumerate}
\section*{Exercice 17}

Soit $ABC$ un triangle tel que $AB = 4$, $AC = 6$ et $BC = 8$. On désigne par $I$ le milieu de $[AB]$ et $J$ le milieu de $[AC]$.

\begin{enumerate}
    \item Montrer que 
    \[
    \overrightarrow{AB} \cdot \overrightarrow{AC} = \frac{1}{2}(AB^2 + AC^2 - BC^2).
    \]

    \item Calculer $\overrightarrow{AB} \cdot \overrightarrow{AC}$, puis déduire $\cos\widehat{BAC}$.

    \item Soit $H$ le projeté orthogonal de $B$ sur $(AC)$, calculer $AH$.

    \item Calculer $\overrightarrow{BA} \cdot \overrightarrow{BC}$, en déduire $BJ$.

    \item 
    \begin{enumerate}
        \item Montrer que pour tout point $M$, on a :
        \[
        MA^2 + MB^2 = 2MI^2 + 8.
        \]

        \item Calculer $CI$.

        \item Déterminer l’ensemble 
        \[
        E = \{M \in P : MA^2 + MB^2 = 100\}.
        \]
    \end{enumerate}

    \item Montrer que pour tout point $M$ du plan, on a :
    \[
    \overrightarrow{MA} \cdot \overrightarrow{MC} = MJ^2 - 9.
    \]

    \item Calculer $\overrightarrow{IA} \cdot \overrightarrow{IC}$. En déduire l’ensemble 
    \[
    E' = \{M \in P : \overrightarrow{MA} \cdot \overrightarrow{MC} = 7\}.
    \]

    \item Soit $O$ le milieu de $[IJ]$.
    \begin{enumerate}
        \item Montrer que $MI^2 - MJ^2 = 2\overrightarrow{IJ} \cdot \overrightarrow{OM}$.
        \item Déterminer l’ensemble $E''$ des points $M$ du plan tels que 
        \[
        MA^2 + MB^2 - 2\overrightarrow{MA} \cdot \overrightarrow{MC} = -6.
        \]
    \end{enumerate}
\end{enumerate}


\end{document}