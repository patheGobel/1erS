\documentclass[12pt]{article}
\usepackage{stmaryrd}
\usepackage{graphicx}
\usepackage[utf8]{inputenc}

\usepackage[french]{babel}
\usepackage[T1]{fontenc}
\usepackage{hyperref}
\usepackage{verbatim}

\usepackage{color, soul}

\usepackage{pgfplots}
\pgfplotsset{compat=1.15}
\usepackage{mathrsfs}

\usepackage{amsmath}
\usepackage{amsfonts}
\usepackage{amssymb}
\usepackage{tkz-tab}
\author{Destinés à la 1\textsuperscript{ère}S2\\Au Lycée de Dindéferlo}
\title{\textbf{Primitive}}
\date{\today}
\usepackage{tikz}
\usetikzlibrary{arrows, shapes.geometric, fit}

% Commande pour la couleur d'accentuation
\newcommand{\myul}[2][black]{\setulcolor{#1}\ul{#2}\setulcolor{black}}
\newcommand\tab[1][1cm]{\hspace*{#1}}

\begin{document}
\maketitle
\newpage
\section*{\underline{\textbf{\textcolor{red}{Définition}}}}
Soit $F$ une fonction dérivable sur un intervalle $I$ et soit $f$ une fonction définie et continue sur $I$.

On dit que $F$ est une primitive de $f$ sur $I$ si pour tout $x$ élément de $I$\\
On a : $F'(x)=f(x)$
\subsection*{\underline{\textbf{\textcolor{red}{Autrement dit :}}}}
$F$ est une primitive de $f$ sur $I$ si $f$ est la dérivée de $F$ sur $I$.
\subsection*{\underline{\textbf{\textcolor{red}{Théorème}}}}
Toute fonction continue sur un intervalle I admet une primitive sur I.
\subsection*{\underline{\textbf{\textcolor{red}{Remarque :}}}}
Si f admet une primitive sur I alors elle en admet une infinité.
\subsection*{\underline{\textbf{\textcolor{red}{Théorème}}}}
Deux primitives d'une même fonction sur un même intervalle diffèrent d'une constante.
 
Donc si $F$ et $G$ sont des primitives de $f$ sur $I$ alors il existe une constante réelle $k$ telle que pour tout $x$ élément de $I$ $F(x)=G(x)+k$

Pour la preuve voir la variation des fonctions avec l'utilisation des dérivées.
 
On en déduit que : 
 
Chacune des primitives de $f$ sur $I$ est déterminée par sa valeur en un point de $I$.
\subsection*{\underline{\textbf{\textcolor{red}{Remarque}}}}
Toute primitive de $f$ sur $I$ est dérivable sur $I$.
 
Il serait utile de connaitre les résultats figurant sur le tableau suivant :
\begin{center}
\begin{tabular}{|c|c|c|c|}
\hline
I=Intervalle &Remarques ou restrictions & Fonction $f$ & Primitive F où k une constante \\
\hline
$I \subset \mathbb{R}$& $x\mapsto 0$ & $x\mapsto k$ &\\
\hline
$I \subset \mathbb{R}$ & & &\\
\hline
$I \subset \mathbb{R}$& & &\\
\hline
$I \subset \mathbb{R}$& & &\\
\hline
$I \subset \mathbb{R}$& & &\\
\hline
$I \subset \mathbb{R}$& & &\\
\hline
$I \subset \mathbb{R}$& & &\\
\hline
$I \subset \mathbb{R}$ & & &\\
\hline
$I \subset \mathbb{R}$& & &\\
\hline
$I \subset \mathbb{R}$& & &\\
\hline
$I \subset \mathbb{R}$& & &\\
\hline
$I \subset \mathbb{R}$& & &\\
\hline
$I \subset \mathbb{R}$& & &\\
\hline
\end{tabular}
\end{center}

Primitive d'une de quelque fonction usuelles

\begin{center}
\begin{tabular}{|c|c|c|c|}
\hline
I=Intervalle &Remarques ou restrictions & Fonction $f$ & Primitive F où k une constante \\
\hline
$I \subset \mathbb{R}$& & &\\
\hline
$I \subset \mathbb{R}$ & & &\\
\hline
$I \subset \mathbb{R}$& & &\\
\hline
$I \subset \mathbb{R}$& & &\\
\hline
$I \subset \mathbb{R}$& & &\\
\hline
$I \subset \mathbb{R}$& & &\\
\hline
$I \subset \mathbb{R}$& & &\\
\hline
$I \subset \mathbb{R}$ & & &\\
\hline
$I \subset \mathbb{R}$& & &\\
\hline
$I \subset \mathbb{R}$& & &\\
\hline
$I \subset \mathbb{R}$& & &\\
\hline
$I \subset \mathbb{R}$&  & &\\
\hline
$I \subset \mathbb{R}$&  &  &\\
\hline
\end{tabular}
\end{center}
\end{document}