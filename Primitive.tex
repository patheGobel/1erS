\documentclass[12pt]{article}
\usepackage{stmaryrd}
\usepackage{graphicx}
\usepackage[utf8]{inputenc}

\usepackage[french]{babel}
\usepackage[T1]{fontenc}
\usepackage{hyperref}
\usepackage{verbatim}

\usepackage{color, soul}

\usepackage{pgfplots}
\pgfplotsset{compat=1.15}
\usepackage{mathrsfs}

\usepackage{amsmath}
\usepackage{amsfonts}
\usepackage{amssymb}
\usepackage{tkz-tab}
\author{Destiné à la 1\textsuperscript{ère}S2\\Au Lycée de Dindéfelo}
\title{\textbf{Primitives}}
\date{\today}
\usepackage{tikz}
\usetikzlibrary{arrows, shapes.geometric, fit}

% Commande pour la couleur d'accentuation
\newcommand{\myul}[2][black]{\setulcolor{#1}\ul{#2}\setulcolor{black}}
\newcommand\tab[1][1cm]{\hspace*{#1}}

\begin{document}
\maketitle
\newpage
\section*{\underline{\textbf{\textcolor{red}{Définition}}}}
Soit $F$ une fonction dérivable sur un intervalle $I$ et soit $f$ une fonction définie et continue sur $I$.

On dit que $F$ est une primitive de $f$ sur $I$ si pour tout $x$ élément de $I$\\
On a : $F'(x)=f(x)$
\subsection*{\underline{\textbf{\textcolor{red}{Autrement dit :}}}}
$F$ est une primitive de $f$ sur $I$ si $f$ est la dérivée de $F$ sur $I$.
\subsection*{\underline{\textbf{\textcolor{red}{Théorème}}}}
Toute fonction continue sur un intervalle $I$ admet une primitive sur $I$.
\subsection*{\underline{\textbf{\textcolor{red}{Remarque :}}}}
Si $f$ admet une primitive sur $I$ alors elle en admet une infinité.
\subsection*{\underline{\textbf{\textcolor{red}{Théorème}}}}
Deux primitives d'une même fonction sur un même intervalle diffèrent d'une constante.
 
Donc si $F$ et $G$ sont des primitives de $f$ sur $I$ alors il existe une constante réelle $k$ telle que pour tout $x$ élément de $I$ $F(x)=G(x)+k$

Pour la preuve voir la variation des fonctions avec l'utilisation des dérivées.
 
On en déduit que : 
 
Chacune des primitives de $f$ sur $I$ est déterminée par sa valeur en un point de $I$.
\subsection*{\underline{\textbf{\textcolor{red}{Remarque}}}}
Toute primitive de $f$ sur $I$ est dérivable sur $I$.
 

\begin{center}
\textbf{Tableau des primitives usuelles}
\begin{tabular}{|c|c|c|}
\hline
Fonction $f$ & Primitive F où k une constante &Intervalles \\
\hline
$f(x)=0$ &$F(x)=k$ & $\mathbb{R}$\\
\hline
$f(x)=a$ & $F(x)=ax+k$ & $\mathbb{R}$\\
\hline
$f(x)=x$ & $F(x)=\frac{1}{2}x^{2}+k$ & $\mathbb{R}$\\
\hline
$f(x)=ax+b$ & $F(x)=\frac{1}{2}ax^{2}+bx+k$ & $\mathbb{R}$\\
\hline
$f(x)=x^{n}, n\neq -1$  & $F(x)=\frac{x^{n+1}}{n+1}+k$ & $\mathbb{R}$,si $n>0.\mathbb{R}\setminus,$si $n\leq -2$ \\
\hline
$fx)=\frac{1}{x^{2}}$ & $F(x)=-\frac{1}{x}+k$ & $\mathbb{R}^{*}$ \\
\hline
$f(x)=\frac{1}{\sqrt{x}}$ & $F(x)=2\sqrt{x}+k$ & $\mathbb{R}^{*}_{+}$\\
\hline
$f(x)=x^{\alpha}\quad \alpha \neq -1$  & $F(x)=\frac{1}{\alpha+1}x^{\alpha+1}+k$ & Selon les valeurs de $\alpha$\\
\hline
$f(x)= \sin(x)$ & $F(x)= -\cos(x)+k$ & $\mathbb{R}$\\
\hline
$f(x)= \cos(x)$ & $F(x)= \sin(x)+k$ & $\mathbb{R}$\\
\hline
$f(x)= \sin(ax+b)$ & $F(x)= -\frac{1}{a}\cos(ax+b)+k$ & $\mathbb{R}$\\
\hline
$f(x)= \sin(ax+b)$ & $F(x)= -\frac{1}{a}\cos(ax+b)+k$ &$\mathbb{R}$\\
\hline
$f(x)= \frac{1}{\cos^{2}(x)}$ & $F(x)= \tan(x)+k$ & $\left]-\frac{\pi}{2};\frac{\pi}{2}\right[,[2\pi] $\\
\hline
$f(x)= \sqrt{x}$ & $F(x)= \frac{2}{3}x\sqrt{x}+k$ & $\mathbb{R}^{*}_{+}$\\
\hline
$f(x)= (ax+b)$ & $F(x)= \frac{1}{a}\frac{(ax+b)^{n+1}}{n+1}+k$ & $\mathbb{R}$\\
\hline
\end{tabular}
\end{center}
\subsection*{\underline{\textbf{\textcolor{red}{Exercice d'application 1}}}}
\begin{center}
\textbf{Primitives et Opérations}
\begin{tabular}{|c|c|c|}
\hline
Fonction $f$ & Primitive F où k une constante &Intervalles \\
\hline
$f=u+v$ &$F=U+V$ & $\mathbb{R}$\\
\hline
$f=ku$ & $F=kU$ & $\mathbb{R}$\\
\hline
$f=u'u^{n}$ ($n\neq -1$) & $F=-\frac{1}{n+1}u^{n+1}$ & $\mathbb{R}$\\
\hline
$f=\frac{u'}{u^{2}}$ & $F=-\frac{1}{u}$ & u ne s'annule pas sur I\\
\hline
$f=\frac{u'}{\sqrt{u}}$ & $F=2\sqrt{u}$ & $u>0$\\
\hline
$f(x)=u'\sin(u)$ & $F(x)= -\cos(u)+k$ & $\mathbb{R}$\\
\hline
$f(x)=u'\cos(u)$ & $F(x)= \sin(u)+k$ & $\mathbb{R}$\\
\hline
\end{tabular}
\end{center}
\subsection*{\underline{\textbf{\textcolor{red}{Exercice d'application 2}}}}
\end{document}