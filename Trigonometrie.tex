\documentclass[12pt]{article}
\usepackage{stmaryrd}
\usepackage{graphicx}
\usepackage[utf8]{inputenc}

\usepackage[french]{babel}
\usepackage[T1]{fontenc}
\usepackage{hyperref}
\usepackage{verbatim}

\usepackage{color, soul}

\usepackage{amsmath}
\usepackage{amsfonts}
\usepackage{amssymb}
\usepackage{tkz-tab}
\author{Destinés à la 1\textsuperscript{ère}S2\\Au Lycée de Dindéferlo}
\title{\textbf{Fonctions Numériques}}
\date{\today}
\usepackage{tikz}
\usetikzlibrary{arrows, shapes.geometric, fit}

% Commande pour la couleur d'accentuation
\newcommand{\myul}[2][black]{\setulcolor{#1}\ul{#2}\setulcolor{black}}
\newcommand\tab[1][1cm]{\hspace*{#1}}

\begin{document}
\maketitle
\newpage

\section*{\underline{\textbf{\textcolor{red}{I.Angles orientés}}}}

\subsection*{\underline{\textbf{\textcolor{blue}{1.1 Orientation du plan}}}}
Considérons un cercle C dans le plan muni d'un repère orthonormé. Soient A, B, et C des points appartenant à C. À partir du point  A, on peut se déplacer le long du cercle en direction de B ou de C.
Le sens positif est défini comme étant celui qui va à l'encontre des aiguilles d'une montre, tandis que le sens négatif est défini comme celui qui suit le sens des aiguilles d'une montre.
\underline{\textbf{\textcolor{red}{NB}}}\\

Sur un cercle, on appelle \textcolor{green}{sens direct}, \textcolor{green}{sens positif} ou \textcolor{green}{sens trigonométrique} le sens contraire des aiguilles d'une montre.
\subsection*{\underline{\textbf{\textcolor{blue}{1.2 Angles orientés de vecteurs}}}}
Soient $\vec{u}$ et $\vec{v}$ deux vecteurs, $O$ un point du plan, $[Ox)$ et $[Oy)$ deux demi-droites telles que $\vec{u}$ soit un vecteur directeur de $[Ox)$ et $\vec{v}$ celui de $[Oy)$.
On a : $(\vec{u}, \vec{v}) = ([Ox),[Oy))$
\subsection*{\underline{\textbf{\textcolor{blue}{1.2 Propriétés}}}}
\begin{itemize}
\item[1] $(\vec{u}, \vec{v})= -(\vec{v}, \vec{u})$
\end{itemize}
\section*{\underline{\textbf{\textcolor{red}{II.Trigonométrie}}}}
\subsection*{\underline{\textbf{\textcolor{red}{2.1 Cercle trigonométrique}}}}
Dans le plan muni d'un repère orthonormé $(O$ ; $\vec{i}$ ; $\vec{j})$ et orienté dans le sens direct, le cercle trigonométrique est le cercle de centre O et rayon 1.
L'idée est de pouvoir se reperer sur le celcle trigonométrique c'est-dire mettre une graduation bien régulière sur le cercle.

\begin{tikzpicture}[scale=1.5]
  % Repère avec les axes Ox (i) et Oy (j)
  \draw[->] (-1,0) -- (1,0) node[right] {$i$};
  \draw[->] (0,-1) -- (0,1) node[above] {$j$};
  
  % Dessiner un cercle centré à l'origine avec un rayon de 1
  \draw (0,0) circle (1) node[below left] {$O$};
  
  % Placer le nombre "1" sur l'axe i
  \node at (1,0) [below left] {1};
  \node at (-1,0) [below left] {-1};
  % Placer le nombre "1" sur l'axe j
  \node at (0,1) [above left] {1};
  \node at (0,-1) [above left] {-1};
\end{tikzpicture}

\begin{tikzpicture}[scale=2]
  % Repère avec les axes Ox (i) et Oy (j)
  \draw[->] (-1.5,0) -- (1.5,0) node[right] {$i$};
  \draw[->] (0,-1.5) -- (0,1.5) node[above] {$j$};
  
  % Dessiner un cercle centré à l'origine avec un rayon de 1
  \draw (0,0) circle (1);
  
  % Marquer les angles principaux et relier les points
  \foreach \angle/\label in {30/\frac{\pi}{6}, 45/\frac{\pi}{4}, 60/\frac{\pi}{3}, 90/\frac{\pi}{2}, 120/\frac{2\pi}{3}, 135/\frac{3\pi}{4}, 150/\frac{5\pi}{6}, 180/\pi, 210/\frac{7\pi}{6}, 225/\frac{5\pi}{4}, 240/\frac{4\pi}{3}, 270/\frac{3\pi}{2}, 300/\frac{5\pi}{3}, 315/\frac{7\pi}{4}, 330/\frac{11\pi}{6}, 360/2\pi}
  {
    \draw ({cos(\angle)}, {sin(\angle)}) -- ({cos(\angle)*1.15}, {sin(\angle)*1.15});
    \node[anchor=\angle-90] at ({cos(\angle)*1.3}, {sin(\angle)*1.3}) {$\label$};    
  }
\end{tikzpicture}
\underline{\textbf{\textcolor{red}{Exercices1}}}\\
On considère un triangle ABC, rectangle en A, de sens direct, tel que $\frac{\pi}{6}$ est une mesure de $\widehat{(\vec{BC}, \vec{BA})}$. Donner trois mesures de chacun des angles orientés suivants:\\
$\widehat{(\vec{AB}, \vec{AC})}$, $\widehat{(\vec{AB}, \vec{CB})}$, $\widehat{(\vec{AC}, \vec{AB})}$, $\widehat{(\vec{AC}, \vec{BC})}$, $\widehat{(\vec{BC}, \vec{AB})}$, 
$\widehat{(\vec{CB}, \vec{AB})}$, $\widehat{(\vec{BC}, \vec{AC})}$, $\widehat{(\vec{BC}, \vec{CA})}$.\\
\underline{\textbf{\textcolor{red}{Exercices2}}}\\
$\vec{u}$ et $\vec{v}$ sont deux vecteurs non nuls els que $\widehat{(\vec{u},\vec{v})}=\frac{2\pi}{3}$.
Donner trois mesures de chacun des angles orientés suivants:\\
$\widehat{(-\vec{u},\vec{v})}$ , $\widehat{(\vec{u},-\vec{v})}$ , $\widehat{(-\vec{u},-\vec{v})}$ , $\widehat{(\vec{v},-\vec{u})}$
\subsection*{\underline{\textbf{\textcolor{red}{2.2 Mesure principale d'un angle}}}}
\underline{\textbf{\textcolor{red}{Approche de la notion}}}\\

\begin{tikzpicture}
  % Repère (O, I, J)
  \draw[->] (-2.5,0) -- (2.5,0) node[right] {$I$};
  \draw[->] (0,-2.5) -- (0,2.5) node[above] {$J$};
  \node at (0,0) [below left] {$O$};

  % Dessiner un cercle
  \draw (0,0) circle (2cm);
  
  % Marquer un point sur le cercle
  \fill (160:2) circle (2pt) node[above right] {$A$};
    % Tracer la droite OA
  \draw[->, thick] (0,0) -- (160:2) node[midway, above left] {$O$};
\end{tikzpicture}

Considérons un point mobile M, qui , partant de I, se déplace sur le cercle trigonométrique $(\mathcal{C})$.\\
A chaque déplacement de M, on associe un nombre réel égal à:\\
\\
La distance parcourue si le déplacement se fait dans le sens direct\\
L'opposé de la distance parcourue si le déplacement se fait dans le sens indirect.\\
Rappelons que le périmettre du cercle trigonométrique est $2\pi$.\\
Déterminons les nombres réels associés aux différents passages de M en A et plaçons ces nombres réels sur une droite graduée $(D)$ (l'unité étant celle des axes du repère orthonormé direct (O,I,J).\\
\\
\begin{tikzpicture}
  % Repère (O, I)
  \draw[->] (-2.5,0) -- (5,0) node[right] {$I$};
  
  % Tracer la droite graduée
  \draw[|-|] (-2,0) -- (4,0) node[below right] {$x$};
  
  % Graduations
  \foreach \x in {-2,-1,0,1,2,3,4}
    \draw (\x,0.1) -- (\x,-0.1) node[below] {$\x$};
\end{tikzpicture}\\
$\bullet$ On montre par ce schéma que :\\
à tout angle orienté $(\hat{\alpha})$ de mesure principale $\alpha$ on peut associer une infinité de nombre réels:\\
..$\alpha - 8\pi $ ; $\alpha - 6\pi $ ; $\alpha - 4\pi $; $\alpha - 2\pi $; $\alpha + 2\pi $ ; $\alpha + 4\pi $ ; $\alpha + 6\pi $ $\alpha + 8\pi $...\\

Ces nombres sont de la forme: $\alpha+k \times 2\pi$, k étant un élément de $\mathbb{Z}$\\
On écrira plus simplement: $\alpha+k \times 2\pi$ $[k\in\mathbb{Z}]$\\
$\bullet$ On remarque qu'un seul d'entre eux, $\alpha$, est élément de l'intervalle $]-\pi ; \pi]$.\\
C'est pouquoi $\alpha$ est appelée la mesure principale de l'angle orienté 
$(\hat{\alpha})$.\\
\underline{\textbf{\textcolor{red}{2.3 Définition}}}\\
\textbf{La} mesure principale d'un angle est comprise entre dans l'intervalle $]-\pi ; \pi]$
\subsection*{\underline{\textbf{\textcolor{red}{2.4 Détermination de la mesure principale d'un angle orienté}}}}
\underline{\textbf{\textcolor{red}{Exemple d'approche}}}\\
$(\hat{\alpha})$ est un angle orienté dont une mesure est $\frac{27\pi}{5}$. Nous allons déterminer la mesure principale de $(\hat{\alpha})$.\\
\underline{\textbf{\textcolor{red}{Résolution}}}\\
Nous allons soustraire $2\pi$ à $\frac{27\pi}{5}$ de proche ne proche jusqu'a obtenir un angle appartenenat à $]-\pi ; \pi]$\\
En effet,\\ posons $\frac{27\pi}{5} \notin ]-\pi ; \pi]$\\
$\frac{27\pi}{5}-2\pi=\frac{17\pi}{5} \notin ]-\pi ; \pi]$\\
$\frac{17\pi}{5}-2\pi=\frac{7\pi}{5} \notin ]-\pi ; \pi]$\\
$\frac{7\pi}{5}-2\pi=\frac{-3\pi}{5} \notin ]-\pi ; \pi]$\\
$\frac{7\pi}{5}-2\pi=\frac{-3\pi}{5} \in ]-\pi ; \pi]$\\
\underline{\textbf{\textcolor{red}{NB}}}\\
Si $(\hat{\alpha})$ est négatif, on ajoute $-2\pi$ de proche en proche jusqu'a ce que l'on trouve un angle appartennat en $]-\pi ; \pi]$\\
\underline{\textbf{\textcolor{red}{Exercice}}}\\
Détermine la mesure principale de $\frac{-17\pi}{3}$\\
\underline{\textbf{\textcolor{red}{Correction}}}\\
\underline{\textbf{\textcolor{red}{Autre approche}}}\\
Déterminons la mesure principale de $\frac{25\pi}{3}$\\
Soit $\alpha$ la mesure principale alors, $\alpha \in ]-\pi ; \pi]$ et soit $k\in\mathbb{Z}$\\
Ponsons $\frac{25\pi}{3}=\alpha+2k\pi \Longrightarrow $ $\alpha=\frac{25\pi}{3}-2k\pi$ alors on a:\\
$-\pi<\frac{25\pi}{3}-2k\pi\leq \pi$\\
\\
$3.66 \leq k < 4.66$\\
On obtient : $k=4 \Longrightarrow \alpha = \frac{\pi}{3}$\\
\underline{\textbf{\textcolor{red}{Exercice}}}\\
Détermine la mesure principale de $\frac{77\pi}{4}$\\
\underline{\textbf{\textcolor{red}{Correction}}}\\
On obtient : $k=10 \Longleftrightarrow \alpha=\frac{-3\pi}{4}$\\
\underline{\textbf{\textcolor{red}{Remarque}}}\\ 
Si deux réels $\alpha$ et $\beta$ sont les mesures d'un même point alors, $\alpha$ et $\beta$ différent d'un multiple entier de $2\pi$; $\alpha-\beta = 2k\pi$
\section*{\underline{\textbf{\textcolor{red}{III. Angles associés et relations trigonométriques}}}}
Soit le cercle trigonométrique $\mathcal{C}(O,1)$\\
\begin{center}
\underline{\textbf{\textcolor{red}{Figure}}}\\
\end{center}
\subsection*{\underline{\textbf{\textcolor{red}{3.1 Expressions en fonction de $cos\alpha$ ou $sin\alpha$}}}}
$cos(\alpha+2k\pi)=cos\alpha$\\
$sin(\alpha+2k\pi)=sin\alpha$\\
$cos(-\alpha)=cos\alpha$\\
$sin(-\alpha)=-sin\alpha$\\

\underline{\textbf{\textcolor{red}{Exercice d'application }}}\\
\underline{\textbf{\textcolor{red}{Résolution }}}
\subsection*{\underline{\textbf{\textcolor{red}{3.2 Quelques relations trigonométriques}}}}
\underline{\textbf{\textcolor{red}{$\ast$Formules d'addition}}}\\

\end{document}
